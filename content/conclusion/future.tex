\subsection*{Future Work}

Throughout this thesis, multiple research questions arose, paving the way for
interesting future directions. The model of Chapter~\ref{chap:hardware-wallets}
highlighted the need for the operator of a hardware wallet to faithfully follow
the prescribed protocol; future work could evaluate the error probability,
identify the causes of mistakes, and propose techniques to reduce such risk.
Building on the results of Chapter~\ref{chap:delegation}, future work could
explore efficient (\ie short) non-malleable address schemes, as well as explore
the implications of incorporating higher levels of anonymity and privacy. The
collective stake pool of Chapter~\ref{chap:collective-pools} currently requires
closing and re-creating a pool, in order to update its parameters and add new
members; a more efficient design could expand this scheme to enable such
changes in a dynamic manner. The analysis of Chapter~\ref{chap:utxo-growth} is
focused on UTxO-based ledgers and size (as a cost function); future research
could propose an adapted state efficiency framework, which incorporates
account-based ledgers and explores its behavior under varying cost models. In
Chapter~\ref{chap:compliance}, we consider only two
infraction predicates and rewards that stem only from the system itself;
future work could explore alternative infraction predicates, \eg to take into account selfish mining,
how reward transfers between parties may affect the compliance analysis, as
well as possible correlations between the exchange rate and protocol rewards,
to produce an analysis for settings closer to the real-world. The
crypto-egalitarianism property of Section~\ref{sec:egalitarianism} is defined
over the, rather simplistic, economic model of Bitcoin (and its disciples);
further research is needed, both to evaluate various existing systems and
identify whether economies of scale in various parameters
exacerbate the identified gap between PoW and PoS. Finally,
Section~\ref{sec:taxation} offered a glimpse of how ledgers can
solve real-world problems that tax authorities face; further research could
enable collaboration between tax authorities of different
countries, incorporate tax gap identification mechanisms in
anonymous ledgers, and offer incentives to motivate
adoption, instead of depending on law enforcement.
