The goal of this thesis was to expand the horizon of how to build secure and
efficient applications on top of distributed ledgers. This goal was approached
via a series of research works, each building on and expanding the existing
literature, from and towards multiple directions. Our research evolved
around three axes:
\begin{inparaenum}[i)]
    \item cryptography;
    \item distributed systems;
    \item game theory and economics.
\end{inparaenum}

On the cryptographic side, we both analyzed pre-existing protocols and proposed
constructions of our own. Specifically, in focusing on the security and safety
of distributed ledger-based systems, we analyzed hardware wallets
(Chapter~\ref{chap:hardware-wallets}), describing a formal model to capture
their necessary properties and evaluate real-world implementations. Following
this, we proposed a holistic model of wallets for Proof-of-Stake-based systems
(Chapter~\ref{chap:delegation}). This line of work brought about a formal
definition of a PoS wallet's core, a novel security notion (address
malleability), and a rigorous description of incorporating the wallet's core in
a PoS distributed ledger system. Notably, our model highlighted a number
of disadvantages, perhaps most intesting of which is the tendency for
centralization around a few participants. After identifying this drawback, we
devised a collective stake pool protocol (Chapter~\ref{chap:collective-pools}),
which enables a group of people to form a coalition in a trustless manner
and be as competitive as centrally-controlled participating entities.
Crucially, across the thesis, our security analyses employed simulation-based
proofs of security, striving to offer the highest level of guarantees possible.

On the systems' side, we focused on improving the efficiency and scalability of
distributed ledger-based systems. Motivated by the ever-growing (and eventually
unsustainable) amount of data published in existing distributed ledgers, we
devised a framework for improving transaction state efficiency
(Chapter~\ref{chap:utxo-growth}). Our framework is inspired by similar
techniques in traditional database systems and consists of multiple layers,
each enabling incremental improvements towards minimizing a ledger's state.

On the economics' side, we analyzed distributed ledger systems first from a
micro-economic perspective. Chapters~\ref{chap:collective-pools}
and~\ref{chap:utxo-growth} offer high-level descriptions of incentivizing users
to behave honestly, when participating in collective stake pools and creating
efficient transactions respectively. More importantly,
Chapter~\ref{chap:compliance} engages in a thorough analysis of the Nash dynamics of
blockchain-based financial systems and introduces the notion of compliance. The results
of our analysis highlight core differences between PoW and PoS-based
systems, formalized the necessity of penalizing misbehaving parties in the
latter, and showcased that the market does not often respond rationally, thus
cannot (on its own) offer protection against economic attacks.

Finally, Chapter~\ref{chap:macroeconomics} explored macroeconomic properties of distributed ledgers.
Section~\ref{sec:egalitarianism} introduced crypto-egalitarianism, a property that
expresses the reward rate of participants, with respect to their
capital investment in the system. We offered a formal definition of
crypto-egalitarianism, which boils it down to a single number,
enabling a precise comparison, in contrast to existing ad
hoc arguments. Crucially, our research proved that wealth redistribution in
favor of the poor is impossible in completely decentralized systems;
consequently, some level of central control is needed to enforce any such
policy in a democratically-mandated manner. Building on this necessity,
Section~\ref{sec:taxation} explored how to reduce tax gaps, thus helping a
government to efficiently enforce its intended tax policies via a distributed
ledger-based monetary and payment system.
