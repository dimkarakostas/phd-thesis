\subsection*{Epilogue}

During the $4$-odd years of working on this thesis, a chasm grew in me. On
the one hand, I devoted large amounts of time and energy in understanding,
building, and fixing decentralized ledgers. On the other hand, my
initial interest in Bitcoin and cryptocurrencies gradually turned to
disillusionment and eventual distaste. Akin to an atheist monk, I kept working
day in day out on beautiful designs, which were used by ideologies that I
detested. I believe that this thesis manifests this internal conflict, with
technical contributions existing side by side with snarky comments.

Two questions emerged from this conflict. Are decentralized ledgers
merely tools, possibly usable in societally beneficial applications,
instead of speculative financial products of late capitalism? Should
intellectual work always serve a societal purpose, or is science for science's
sake good enough? Neither question is new. The former is expressed with the
classic parable, that a knife can be used both to slice bread and as a murder
instrument. The latter came to the spotlight during the $19$th and $20$th
centuries in science, due to nuclear power, and in art, with the emergence of
aestheticism.

I don't have an answer to either question. I don't know if a definitive answer
\emph{can} exist. Nonetheless, I do know that acknowledging these questions may
lead to the same consequences as acknowledging Camus's absurd: revolt, freedom,
and passion.\footnote{\emph{The Myth of Sisyphus}, Albert Camus}
