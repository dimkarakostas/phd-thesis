\section{Literature Overview}\label{sec:related}

We now overview notable works that pertain to the entire scope of the thesis.
Following, each individual chapter focuses on the body of literature related to
its specific research questions.

\subsection{Bitcoin Formal Models}
The importance of formal methods for analyzing Bitcoin is well understood, with
existing literature showcasing different approaches. Garay
\etal~\cite{EC:GarKiaLeo15,C:GarKiaLeo17}, after extracting and analyzing the
core Bitcoin blockchain protocol, presented a formal abstraction to prove that
Bitcoin satisfies a set of security and quality properties. Pass
\etal~\cite{EC:PasSeeShe17} analyzed the consistency and liveness properties of
the consensus protocol in an asynchronous setting, proving Bitcoin secure
assuming an upper bound on the network delay. Badertscher \etal~\cite{C:BMTZ17}
suggested a Universally Composable treatment of the Bitcoin ledger, defining
Bitcoin's goals and proving that their model is securely realized in the UC
framework. Transactions, being a core part of Bitcoin, have also attracted
attention; notably, Atzei \etal~\cite{FC:ABLZ18} proposed a formal model of
Bitcoin transactions, to prove security \eg against double-spending attacks.

\subsection{Proof-of-Stake Protocols}
The cryptographic literature has seen a number of PoS protocols in the past
years. A family of such protocols, which was initiated with
Ouroboros~\cite{C:KRDO17} and continued with Ouroboros Praos~\cite{EC:DGKR18},
Ouroboros Genesis~\cite{CCS:BGKRZ18}, and Ouroboros
Crypsinous~\cite{SP:KKKZ19}, provides a wide range of threat model
considerations, relevant to PoS systems, and offers eventual guarantees of
liveness and persistence, similar to Bitcoin.
Algorand~\cite{EPRINT:CGMV18,EPRINT:GHMVZ17} is a protocol which employs
Byzantine Agreement to achieve the necessary properties of a PoS setting, as
well as transaction finality in (expected) constant time. Snow
White~\cite{FC:DaiPasShi19,AC:PasShi17} similarly uses the notion of
``robustly reconfigurable consensus'', which is specially designed to cope with
the lack of participation of users in the consensus protocol.

Real-world PoS implementations often opt for stake representation and
delegation, similar to our work in Chapter~\ref{chap:delegation}.  Systems like
Cardano~\cite{cardano}, EOS~\cite{eosWhitepaper}, and (to
some extent) Tezos~\cite{goodman2014tezos}, employ different consensus
protocols, but all enforce that a (relatively small) subset of representatives
is elected to participate.  Decred~\cite{decred} takes a somewhat different
approach, where stakeholders buy a ticket for participation, akin to PoS with
optional participation. However, these systems typically assume single parties
that act as delegates, either individually or as pool operators, a restriction
that Chapter~\ref{chap:collective-pools} directly aims at relaxing.

\subsection{Blockchain Incentives}
The seminal work of Selfish
Mining~\cite{FC:EyaSir14,FC:SapSomZoh16,kiayias2016blockchain} showed that
honest behavior is not incentive-compatible. However, alternative reward
sharing mechanisms in the PoS setting may make it feasible to perform better in
terms of incentive compatibility. For instance, Ouroboros~\cite{C:KRDO17} can
be designed from the ground up to be a Nash equilibrium under certain plausible
conditions. The question of how to incentivize parties in PoS systems to form a
desired number of stake pools was further studied
in~\cite{DBLP:conf/eurosp/BrunjesKKS20}. The problem that these works study,
and which we also explore in Chapter~\ref{chap:compliance}, is designing the
incentives of blockchain systems from the designer's point of view, so that
participants do not deviate from the prescribed protocol. One related question
is how fair the protocol is to participants themselves, particularly to honest
participants. The Bitcoin Backbone and Selfish Mining works include attacks in
which an adversary can strategically the ledger's performance, causing the
number of blocks and, in turn, the respective rewards, to be disproportionate
to their contributed computational power, thereby harming fairness towards
honest participants. In the PoW setting, Fruitchains~\cite{PODC:PasShi17}
proposes a protocol which solves this problem via the notion of ``fair''
rewards, \ie rewards in exact proportion to the computational power of each
party.
