\section{Distributed Ledgers}

We now overview the problem of constructing a secure distributed ledger.
Specifically, starting from the problem of consensus, a fundamental problem of
distributed computing on which distributed ledgers are based, we provide formal
definitions for the problem space that blockchain protocols aim to solve.

\subsection{Consensus}

The consensus problem, introduced in the seminal work of Shostak, Pease, and
Lamport~\cite{lamport1982byzantine,pease1980reaching}, is the setting where a
set of parties, each with its own input, need to reach agreement and output the
same value. A protocol that solves the consensus problem demonstrates three
core properties~\cite{koulouris2002distributed}, namely termination, agreement,
and validity (Definition~\ref{def:consensus}).

\begin{definition}[Consensus]\label{def:consensus}
    A consensus protocol $\proto$, which is performed among $\totalParties$
    parties $\party_i$, each with input $v_i$, satisfies the following
    properties:
    \begin{itemize}
        \item \emph{Termination:} eventually each correct process outputs a
            single value;
        \item \emph{Agreement:} all correct processes output the same value;
        \item \emph{Validity:} if all correct processes start $\proto$
            with the same input $v$, then every correct process outputs $v$.
    \end{itemize}
\end{definition}

\subsection{Reliable Broadcast}

A problem close, but not identical, to consensus is that of Reliable Broadcast
(RBC), which will prove useful in the construction of the collective pool of
Chapter~\ref{chap:collective-pools}. Briefly, a RBC protocol ensures that a
message output by an honest party is eventually broadcast and output by all
other honest parties. Definition~\ref{def:broadcast} describes RBC, following
the formalization of~\cite{PODC:CJKR12,cryptoeprint:2021:671}.

\begin{definition}[Reliable Broadcast]\label{def:broadcast}
    Let $\party_s$ be a designated sender of its input $b_{in}$. A Reliable
    Broadcast (RBC) protocol, which is performed among $\totalParties$ parties,
    satisfies the following properties;
    \begin{itemize}
        \item \emph{Safety}:
            \begin{inparaenum}[i)]
                \item \emph{Consistency}: if two honest parties output values
                    $b, b'$ respectively, then $b = b'$;
                \item \emph{Integrity}: if $\party_s$ is honest, no honest
                    party outputs a value $b \neq b_{in}$.
            \end{inparaenum}
        \item \emph{Liveness}:
            \begin{inparaenum}[i)]
                \item \emph{Validity}: if $\party_s$ is honest, then all honest
                    parties output some value;
                \item \emph{Totality}: if an honest party commits a value,
                    then all honest parties output some value.
            \end{inparaenum}
    \end{itemize}
\end{definition}

\subsection{Distributed Ledger}\label{subsec:distributed-ledger}

A ledger is an append-only list of ordered transactions: $\ledger = [\tx_0,
\dots, \tx_j]$. A distributed ledger is a ledger that is maintained in a
decentralized manner, \ie by multiple parties without a single central
authority. Intuitively, a ledger can be seen as a repetitive execution of a
consensus protocol, where the input is a transaction. As
Definition~\ref{def:secure-ledger} shows, a distributed ledger is secure if all
parties agree on the transaction ordering and the ledger expands over time.

\begin{definition}[Distributed Ledger]\label{def:secure-ledger}
    Let $\proto$ be a protocol that is run by $\totalParties$ parties; $\proto$
    implements a secure distributed ledger if the following properties are
    satisfied:
    \begin{itemize}
        \item \emph{Safety}: If two honest parties output $[\tx_0, \dots,
            \tx_j]$ and $[\tx_0', \dots, \tx_j']$ respectively, then $\forall i
            \in [0, \min(j, j')]: \tx_i = \tx_i'$.
        \item \emph{Liveness}: If a transaction $\tx$ is provided as input to
            at least one honest party, eventually all honest parties output a
            ledger containing $\tx$.
    \end{itemize}
\end{definition}
