\section{Cryptographic Primitives}

Across the thesis we assume a peer-to-peer network, where parties use a gossip
\emph{diffuse} functionality~\cite{EC:GarKiaLeo15}, without the need of a
fully-connected graph and point-to-point connections. Consequently, when a
party receives a message, they cannot know which party the message originated
from.

\paragraph{Synchronous Network}
Under a synchronous network, a message produced by an honest party at round
$\round$ is delivered as input to all other parties by at most round $\round +
t$, where $t$ is a threshold known a priori to a protocol's designer. In the
simplest case, when $t = 1$, each protocol round simulates a message passing
round, \ie the maximum time required to deliver a message at any computer in
the world; in blockchain analyses, the message passing round is typically equal
to $10-20$ seconds~\cite{EC:GarKiaLeo15,wood2014ethereum}.

\paragraph{Adversary}
The adversary $\adversary$ is an algorithm that aims at breaking the security
of the protocol under review, \ie violate one of its security properties.
$\adversary$ is \emph{adaptive}, \ie can corrupt parties dynamically, and
\emph{rushing}, \ie can decide its strategy after receiving, and possibly
delaying, the honest parties' messages.

\subsection{Cryptographic Hash Functions}

Cryptographic hash functions, as in Damg{\aa}rd~\cite{EC:Damgaard87},
exhibit the properties outlined in Definition~\ref{def:hash}.

\begin{definition}[Hash Function]\label{def:hash}
    A cryptographic hash function $\hash: \{0, 1\}^* \rightarrow \{0, 1\}^l$ is
    a function that, for some $l$ which is the length of the hash values,
    presents the following properties:
    \begin{itemize}
        \item \emph{Collision Resistance}:
            Given $h \leftarrow \{0, 1\}^l$ it should be computationally
            infeasible for a probabilistic polynomial algorithm to find a value
            $x$ such that $h = \hash(x)$.
        \item \emph{Pre-image resistance}:
            It should be computationally infeasible for a probabilistic
            polynomial algorithm to find two values $x, y$ where $x \neq y$
            such that $\hash(x) = \hash(y)$.
        \item \emph{Second pre-image resistance}:
            Given a value $x$, it should be computationally infeasible for a
            probabilistic polynomial algorithm to find a value $y \neq x$ such
            that $\hash(x) = \hash(y)$.
    \end{itemize}
\end{definition}

\subsection{Digital Signatures}\label{sec:secure_sig}

A digital signature scheme $\sigscheme$, as in
Canetti~\cite{EPRINT:Canetti03} and Goldwasser \emph{et
al.}~\cite{C:GolMicRiv84}, is a triple of algorithms $\sigscheme = \langle
\algokeygen, \algoverify, \algosign \rangle$, as described in Definition~\ref{def:digsign}.
Following Definition~\ref{def:eufcma} describes a core property of signatures,
which is resistance to Existential Unforgeability under Adaptive Chosen Message
Attacks (\eufcma).

\begin{definition}[Digital Signature]\label{def:digsign}
    For a security parameter $\secparam$, a digital signature scheme $\sigscheme$
    is a tuple $\digsign$:
    \begin{itemize}
        \item $\algokeygen(1^\secparam) \rightarrow (\keyverify, \keysign)$: a
            randomized algorithm that, given the security parameter
            $\secparam$, outputs a pair of keys, the verification key
            $\keyverify$ and the $\secparam$-bit long private key $\keysign$;

        \item $\algosign(\mesg, \keysign) \rightarrow \signature$: (possibly)
            randomized algorithm that, given a message $\mesg$ and the private key
            $\keysign$, outputs a signature $\signature$;

        \item $\algover(\mesg, \keyverify, \signature) \rightarrow \{0,1\}$: a
            deterministic algorithm that, given a message $\mesg$, a public key
            $\keyverify$, and a signature $\signature$ outputs $1$ if a
            signature is valid \wrt message $m$ and verification key
            $\keyverify$ (respectively $0$ if the signature is invalid).
    \end{itemize}
\end{definition}

\begin{definition}[EUFCMA]\label{def:eufcma}
    A digital signature scheme $\sigscheme$ is Existentially Unforgeable under
    Adaptive Chosen Message Attacks (\eufcma) if it presents the following
    properties:

    \begin{itemize}
        \item \emph{Completeness}:
            For any message $\mesg$, it holds:
            \begin{align}
                \Pr[(\keyverify, \keysign) \leftarrow \algokeygen(1^\secparam), \signature \leftarrow \algosign(\mesg, \keysign): 0 \leftarrow \algover(\mesg, \keyverify, \signature)]
                \leq \negl \nonumber
            \end{align}
            where all probabilities are computed over the random coins of
            the generation and sign algorithms.
        \item \emph{Consistency}:
            For any message $\mesg$, the probability that two independent
            executions of $\algover(\mesg, \keyverify, \signature)$ for a key pair
            $(\keyverify, \keysign) \leftarrow \algokeygen(1^\secparam)$,
            output two different outcomes is smaller than $\negl$.
        \item \emph{Unforgeability}:
            For any PPT algorithm $\adv_{forger}$, which can query the
            signature oracle $\algosign(\cdot, \keysign)$ for signatures on a
            polynomial number of messages $\mesg_i$, it holds:
            \begin{align}
                \Pr\left[
                (\keyverify, \keysign) \leftarrow \algokeygen(1^\secparam):
                \begin{tabular}{c}
                    $(\mesg, \signature) \leftarrow \adv_{forge}^{\algosign(\cdot, \keysign)} \wedge$ \\
                    $\mesg \neq \mesg_i \wedge$ \\
                    $\algover(\mesg, \keyverify, \signature) = 1]$ \\
                \end{tabular}
                \right]
                \leq \negl(\secparam) \nonumber
            \end{align}
            where all the probabilities are computed over the random coins of
            the generation algorithm and the adversary.
    \end{itemize}
\end{definition}

\subsection{The Universal Composability Framework}

The Universal Composability (UC) Framework by Canetti~\cite{FOCS:Canetti01} is
a tool that enables us to capture the security properties of a distributed
protocol. As a preparation for presenting the framework, consider two ensembles
$X=\{X_{\secparam,z}\}_{\secparam \in \N, z\in\bits^*}$ and
$Y=\{Y_{\secparam,z}\}_{\secparam \in \N, z\in\bits^*}$ of binary random
variables. $X$ and $Y$ are said to be \emph{computationally indistinguishable},
denoted by $X \cind Y$, if for all $z$ it holds that $\mid
\Pr[\mc{D}(X_{\secparam,z})=1]-\Pr[\mc{D}(Y_{\secparam,z})=1]\mid$ is
negligible in $\secparam$, \ie $\negl$, for every probabilistically
polynomial-time (PPT) distinguishing algorithm $\mc{D}$.

The main idea of security proofs under the UC framework relies on the
comparison between the execution of a concrete protocol, say $\pi$, and a
security definition, named the \emph{ideal functionality}. These two executions
are, respectively, the \emph{real world} and the \emph{ideal world}. Both are
controlled by an entity called the \emph{environment}, denoted by $\env$, which
can submit actions and observe outputs from the executions. The environment
controls the execution of $\pi$, through choosing the inputs of its
participants, and also the actions of the adversary $\adv$ in the real world.
We note that our work is restricted on executions where every party $\party \in
\partySet$ is activated on each time slot. $\env$ also controls the activation
schedule and the inputs of each party. The adversary $\adv$ can read the
messages exchanged between the protocol players and even delay them, to a
degree that depends on the network model. Moreover, it is allowed to corrupt
players per the environment's instructions, in which case the player's secret
state is compromised and is available to the adversary.

More formally, every entity is modeled as a PPT Interactive Turing Machine
(ITM), and the real world and ideal executions are respectively represented by
the ensembles:
\begin{align*}
    \real=\ensereal
\end{align*}
and
\begin{align*}
    \ideal=\enseideal
\end{align*}
and uniform randomly chosen value $r$. We use $\real(\secparam,z,r)$ to denote
the output of the environment $\env$ in the real-world execution of a protocol
$\pi$ and the adversary $\adv$ under security parameter $\secparam$, input $z$
and randomness $r$. Analogously, we denote by $\ideal(\secparam,z,r)$ the
output of the environment in the ideal interaction between the simulator
$\Iadv$ and the ideal functionality $\Fuc$ under security parameter
$\secparam$, input $z$ and randomness $r$. It is said that \emph{the protocol
$\pi$ securely realizes the functionality $\mc{F}$} when the environment
cannot distinguish between the two worlds, \ie for every $\adv$ exists a
simulator $\Iadv$ such that for every PPT $\env$ we have that
$\real\cind\ideal$.
