\begin{acknowledgements}
    First and foremost, this thesis owes its existence to my advisor, Prof.
    Aggelos Kiayias. His calming presence, steady guidance, profound advices,
    and contagious passion for the scientific process have been staples of my
    journey through academia. He allowed me to work independently and on my
    preferred pace, putting trust in my abilities, always offering
    opportunities to better myself and never making me feel pressured. For all
    these and more, I am forever grateful.

    Completing this thesis would have been impossible without the continuous
    discussions and exchange of ideas with multiple people. My co-authors and
    collaborators Nikos Karayannidis, Thomas Zacharias, Andriana Gkaniatsou,
    Myrto Arapinis, Christos Nasikas, and Kostis Karantias have been extremely
    helpful in tackling the questions we faced and helping me learn in practice
    how science is conducted. I am also grateful to Drs. Vesselin Velichkov and
    Arthur Gervais, for taking the time to review this thesis, posing
    intriguing questions during the viva voce examination, and providing me
    with constructive feedback. I consider myself lucky to have been member of
    a research group alongside Markulf Kohlweiss, Vassilis Zikas, Michele
    Ciampi, Thomas Kerber, Orfeas Stefanos Thyfronitis Litos, Giorgos
    Panagiotakos, Hendrik Waldner, Christian Badertscher, Lamprini Georgiou,
    Misha Volkhov, Lorenzo Martinico, Muhammad Ishaq, Yun Lu, Aydin Abadi, and
    Yiannis Tselekounis, who were daily companions in discussions that, more
    often than not, culminated in undeniably entertaining arguments. I am
    particularly thankful to IOHK, for funding my numerous conference travels
    all around the globe, and Mirjam Wester, for helping me tackle the oh-so
    dreaded bureaucracy. Finally, I feel the need to give special thanks to
    Mario Larangeira, who was a constant collaborator for the duration of my
    thesis and never failed to lift my spirits, and Dionysis Zindros, for
    teaching me the marvels of technology and motivating me by example.

    To the extent that this thesis is a product of mine, I am a product of the
    relationships with the people closest to me. My mom, dad, and sister have,
    simply put, shaped who I am today more than any other people. Christos,
    Panagiotis, and Tea have been part of my life for more years than not; in
    our relationship, lies my definition of friendship. I was fortunate enough
    to have shared my adult life with friends in Thodoris, Konstantina,
    Nikolas, Fenia, Christos, Thanos, Foteini, Georgia, and Dorothea, who
    always made my living in Greece and abroad as relaxing and fun as I could
    hope for.

    In culminating this journey, any attempt to compress in a few sentences
    what Mirella means to my life is futile; all I can offer is a humble thank
    you.
\end{acknowledgements}
