Atop Castle Rock, a land which has been occupied by humans since the Iron Age,
stands Edinburgh's castle. The most famous attraction in a city full of those,
the castle used to serve as the royal residence and repository of Scotland's
official documents, since as early as the High Middle Ages. Initially, it
comprised of only a fortified keep. This inner sanctum, a small and square
stone building, was the place where, at each point in time, the ruler decided
the fates of Scotland and its people.
Centuries passed and towers were added, houses were
constructed, extra lines of fortification were built. The confines of the
castle now housed the king's advisors, that is the people who had gained his
trust and helped him rule the land. Outside the castle's gates, a society
formed, as potters supplied the castle with ceramics and stoneware,
masons built or restored walls and roads, peasants and merchants travelled from
across the land to trade in food and other products. These people lived on the
periphery of the castle and were allowed to enter its premises by permission
only, either from the king or his closed circle of confidants.

Eventually, the castle, both as a physical space and an idea, became too limited
for the modern times. With the degradation of feudalism
and the industrial revolution, the castle was no longer the fortified home of
the ruler. More and more people gradually started participating in
the governing of the country. First, the king was the lone ruler; then, he
presided an inner council of hand-picked members; following, this council
comprised of various lords and members of the upper class. Starting from
$1802$ and the first British general elections, a few thousand
aristocrats would elect a government. This was followed by
granting voting rights to all male home owners and, eventually, universal
suffrage for all citizens. Nowadays, the Scottish Government is elected by an
open, fluid body of people, consisting of all \emph{residents} of
Scotland aged $16$ and above.\footnote{The above timeline should be perceived
more as an allegory for the paragraphs to follow, rather than a scientific
exploration of Scotland's history and politics; for the latter, the reader may advise
textbooks and works specializing on the matter, such
as~\cite{maclean2019scotland,bambery2014people}.}
