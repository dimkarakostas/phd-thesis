\section{Motivation and Contributions}

Evidently, although the distributed ledger used by Bitcoin
is a marvelous \emph{technical} achievement, the application built on top of it
is rather flawed. Specifically, as Bitcoin is a somewhat simple protocol, aimed
at working on a basic level for its core application, \ie transacting, there
are various avenues of research on the usability, efficiency, and ecological
and economic sustainability of distributed ledger-based applications.

Regarding usability, Bitcoin users control their assets via a cryptographic
key. Using this key, they can only perform limited actions, such as
transferring assets between addresses, possibly under some very basic
conditions. Importantly, the ledger is immutable; once a transaction is
complete, it is impossible to revert it. As a result, if the controlling key is
compromised, the assets are at risk. This makes Bitcoin and cryptocurrencies a
prime target for criminals, as is evident by the --- almost daily --- reports
of thefts that are worth thousands of dollars.

Regarding efficiency, a decentralized system should be carefully designed to
enable hundreds or thousands of participants to quickly process data under
reasonable hardware requirements. Bitcoin requires of users and miners to store
two data objects. First, the log of transactions, an ever-increasing list of
historical data. Second, the state of the systems, \ie a mapping of addresses
and the amount of bitcoins each owns. As time passes, both objects increase.
Therefore, if this trend continues, the system is bound to be unmaintainable
without using specialized, expensive hardware.

From an ecological point of view, Bitcoin is clearly unsustainable. In the past years,
this has become common knowledge, with alternative, more sustainable designs
being proposed, of which Proof-of-Stake (PoS) is the most prominent. PoS is a
variation of PoW where, instead of computational power, users are identified by
stake in the system, \ie the assets that they own. Consequently, the energy
footprint of PoS-based ledgers is minuscule, offering a rather appealing
alternative to Bitcoin.

From an economic point of view, both in theory and in practice, Bitcoin has
proven unable to sustain a real-world, productive economy. Instead, as shown
above, it has been used mostly in gray areas of speculation and dubious
transactions. A particularly interesting question then is how distributed ledgers could be
better utilized, by applications that solve real problems and which are managed
in an open, democratic manner by the whole of society, instead of a small group
of insiders and early adopters.

Based on these principles, this thesis makes incremental steps in improving how
digital assets, which are maintained via a distributed ledger, are designed,
managed, and used. The contributions are split into the following main
chapters, each based on one of the research papers output during the
composition of the thesis:
\begin{itemize}
    \item \textbf{Formalization of Hardware Wallets}:
        Chapter~\ref{chap:hardware-wallets} analyzes hardware wallets, \ie
        hardware modules that offer state-of-the-art security in storing and
        managing cryptocurrencies. We present a formal model of expressing the
        necessary properties of hardware wallets, which is then used to analyze
        a number of commercial products, identifying, in some cases, potential
        hazards.
    \item \textbf{Account Management in Proof-of-Stake Ledgers}:
        Chapter~\ref{chap:delegation} articulates the necessary properties of
        managing assets on PoS-based ledgers. In doing so, it first identifies
        a malleability attack on cryptocurrency addresses, which is applicable
        against real-world deployed systems, and then proposes a formal model
        that captures the security of Proof-of-Stake wallets, which enables
        participation in the ledger, either as a user or a maintainer of the
        system.
    \item \textbf{Collective Stake Pools}:
        The model of Chapter~\ref{chap:delegation} defines participation in the
        consensus mechanism of a Proof-of-Stake ledger via stake pools, \ie
        collaborative entities of multiple parties. However, these pools are
        presumably operated by a single party.
        Chapter~\ref{chap:collective-pools} relaxes this centralization
        assumption by introducing \emph{Conclave}, a design that enables a
        group of parties to jointly manage a stake pool, in a competitive and
        highly efficient manner.
    \item \textbf{Efficient Global State Management}:
        The transactions that are created by a wallet are stored in a global
        state, which is shared across all participants. As such, efficient
        state management is imperative, if the system is to scale to thousands
        or millions of users. Chapter~\ref{chap:utxo-growth} introduces a
        framework for constructing such efficient transactions and describes
        how to incentivize both users (\eg operators of hardware wallets, as in
        Chapter~\ref{chap:hardware-wallets}) and maintainers (\eg stake pool
        operators, as in Chapters~\ref{chap:delegation}
        and~\ref{chap:collective-pools}) to avoid the unnecessary bloating of the
        shared state.
    \item \textbf{Blockchain Nash Dynamics}:
        Cryptographic treatment assumes that parties act either faithfully or
        arbitrarily and (possibly) maliciously; to evaluate whether it is in
        the parties' best interest to follow a protocol, we turn to game
        theory. Chapter~\ref{chap:compliance} builds on the results of
        Chapter~\ref{chap:utxo-growth} and explores under which conditions
        parties remain compliant, \ie do not exhibit a well-defined problematic
        behavior, even if slightly diverging from the prescribed protocol. We
        evaluate large families of deployed protocols and offer both positive
        and negative results, which showcase the differences between PoW and
        PoS, as well as the limits of system design in the presence of external
        market factors.
    \item \textbf{Macroeconomic Principles}:
        Chapter~\ref{chap:macroeconomics} explores some macroeconomic properties of
        blockchain-based financial systems. First, we show that wealth redistribution from large to
        small capital owners is impossible in anonymous decentralized financial
        systems. Building on this result, we define crypto-egalitarianism, a
        metric which identifies the rate at which wealthy investors accumulate
        capital and quantifies the identified limitation, with the best
        possible scenario being a linear reward rate with respect to the
        invested capital.
        Second, we consider how a taxation policy can be enforced in such
        environment, under such limitations. Although the first and, till now,
        primary application of blockchains has been hosting decentralized
        financial systems, an evolving line of research looks into integrating
        this technology in traditional systems, to create centrally-controlled
        digital cash.  We builds on this idea by exploring how distributed
        ledgers can help solve a widespread problem in real-world economies,
        \emph{tax gaps}. We contribute by presenting two ideas, via which a tax
        authority can identify differences between the assets reported by
        citizens and the actual assets these citizens own.
\end{itemize}

In addition, Chapter~\ref{sec:preliminaries} reviews necessary
background material and Chapter~\ref{chap:conclusion} offers concluding remarks
and ties together the thesis's core results.
