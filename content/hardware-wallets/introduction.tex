Bitcoin, being the most successful cryptocurrency, has been repeatedly attacked
with many users losing their funds.  As wallets are the only way for a user to
access their funds, they are repeatedly targeted for attacks that aim to access
the account's keys or redirect the payments, ranging from clipboard
hijacking~\cite{clipboardHijack} and malware~\cite{NDSS:HDMDGM14} to
implementation bugs, \eg the Parity hack in Ethereum~\cite{alois2017ethereum},
and more specific attacks, \eg brain wallets~\cite{FC:VBCKM16}.  In order to
address such threats, different ways to harden the wallet's security have been
proposed, with the most notable one being the utilization of cryptographic
hardware, \ie \emph{hardware wallets}. These specialized devices currently
dominate the market and, as demand grows, so does the number of commercially
available products. However, they are also arguably the least formally studied
part of cryptocurrency ecosystems, with their specifications and security goals
remaining unclear and understudied.

The module known as a hardware wallet is responsible for the account's key
management and the execution of the required cryptographic operations. The
remaining operations are completed by a dedicated software, either provided
together with the hardware or by a third party, with which the hardware
communicates. However, incorporating expensive hardware as a wallet is bound to
bring some security guarantees. Currently, the security of commercially
available products can only be checked through manual inspection of their
implementation, a process that requires a strong engineering and technical
background, and a significant effort and time commitment. In turn, proprietary
assumptions and lack of a universal threat model frequently lead to
implementations prone to attacks.

Our work aims at bridging the gap between formally modeling and verifying the
wallet's properties and claimed specifications. In this chapter, we devise a
formal model which defines the characteristics, specifications, and security
requirements of hardware wallets. Instead of capturing a hardware wallet as a
single module, we conceptualize it as a system of different modules that
communicate with each other to perform the wallet's operations. This approach
allows us to identify a greater set of potential attacks and the conditions for
them to be successful. To that end, we manually inspect commercial products
and extract their implementations, which we then map to our model. As we show,
hardware wallets are prone to a set of attacks and are secure only under
specific, well-defined assumptions. Therefore, our model can not only prove the
security of existing implementations, but also act as a reference guide for
future implementations.

\paragraph{Related work}
Until now, Bitcoin hardware wallets have only been empirically studied, with
research focusing on the integrity of transactions. Gentilal
\etal~\cite{gentilal2017trustzone} stressed the necessity of separating the
wallet into two environments, the trusted and the non-trusted, and proposed
that a wallet remains secure against attacks by isolating the sensitive
operations in the trusted environment. Similarly, Lim
\etal~\cite{lim2014analysis} and Bamert \etal~\cite{bamert2014bluewallet},
argue that security in Bitcoin wallets equals with tamper-resistance and
propose the use of cryptographic hardware. Hardware wallets have not yet been
extensively studied, since no formal attempt to specify the functionalities and
the security properties of such wallets exists so far. As of September 2018,
research  has only focused on attacking commercially available implementations.
Gkaniatsou \etal~\cite{ISC:GkaAraKia17} showed that the low-level communication
between the hardware and its client is vulnerable to attacks which escalate to
the account management.  Their research concluded to a set of attacks on the
Ledger wallets, which allowed to take control of the account's funds.  Hardware
wallets have also been studied against physical attacks.
Volotikin~\cite{volotikin2018bh} showed that specific parts of the Ledger's
flash memory are accessible, exposing the private keys used for the second
factor verification mechanism.  Datko \etal presented fault injection, timing
and power analysis attacks on KeepKey~\cite{keepkey} and Trezor~\cite{trezor},
which allowed them to extract the private key.

\paragraph{Contributions}
This chapter provides a holistic formal treatment of hardware wallets,
identifying their core specifications and security properties.  The proposed
model allows to reason about the offered security of existing wallets and acts
as the foundation for designing and implementing new ones. In particular, this
chapter:
\begin{inparaenum}[i)]
 \item defines the properties and requirements of hardware wallets;
 \item provides a formal model and security guarantees of such wallets;
 \item evaluates the security of commercial products under a formal model.
\end{inparaenum}

In more detail, we identify the properties and security parameters of a Bitcoin
hardware wallet and formally define them in the Universal Composition (UC)
Framework. We present a modular treatment of a hardware wallet, by realizing
the wallet as a set of protocols.  This approach allows us to capture in detail
the wallet's components and interactions, and the potential threats. We deduce
the wallet's security by proving that it is secure under common cryptographic
assumptions, provided that there is no deviation in the protocol execution.
Finally, we define the attacks that are successful under a protocol deviation,
and analyze the security of commercially available wallets.
