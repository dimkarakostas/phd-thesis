\section{Product evaluation}\label{sec:evaluation}

The evaluation of commercial hardware wallet products was performed on September 2018.
At that time, the hardware wallets suggested by
\href{https://bitcoin.org/}{bitcoin.org} were Digital Bitbox, KeepKey, Ledger,
and Trezor. The latter $3$ had an embedded screen to present information to the
user, so we focus on them. We manually inspected these wallets, identified
their protocols, and mapped them to our model.

These implementations bare significant similarities to
each other and our model. Although the wallets did have different low-level
implementations, the protocols that they execute are captured by the hybrid
setting of Section~\ref{sec:hybrid-protocol}. This similarity, between our
model and the actual implementations, reaffirmed the correctness of previous
empirical studies, which suggest that the Ledger wallets are prone to the
payment~\cite{ISC:GkaAraKia17} and address generation~\cite{receiveAttack}
attacks. The wallets were subject to these attacks when the client is dishonest
and secure only if the cryptographic primitives are secure \emph{and} the user
does not deviate from the defined protocol, \ie successfully identifies
tampered data. Moreover, the instantiation of our model to the three
implementations suggests that the wallets were prone to chain attacks, as
discussed in Section~\ref{sec:theorem}.

For each implementation we focused on the two core wallet operations:
\emph{address generation} and \emph{transaction issuing}. Since all
implementations were susceptible to chain attacks, we focused on the viability
of payment and address attacks in each case. We showed that Trezor and KeepKey
were secure against payment and address attacks, as long as the user follows
the protocol and verifies the data, whereas Ledger wallets were prone to
address attacks, due to divergence from our model. We expect this type of
evaluation to become an industry standard for hardware wallets, so that vendors
can improve the security and performance of their products by employing formal
verification methods, instead of empirical techniques.

\paragraph{Trezor and KeepKey}
We investigated the implementation of the Trezor Model T and KeepKey hardware
wallets. Both products are implemented similarly, so we focused only on Trezor.
Trezor provides a touch screen for both displaying information and receiving
input from the user. Based on the developer's guide~\cite{trezor-dev}, we next
describe an abstraction of Trezor's behavior under our model.
During address generation, Trezor requires that the user connects the token to
the client and unlocks it, \ie the user \emph{initiates a session} similar to
our model definition. The client then retrieves the address from the hardware
token and displays it to the user. The hardware also displays the address, as
long as the ``Show on Trezor'' option is enabled. If this option is disabled,
then the user cannot verify the client's address and is prone to an address
attack, \ie the client might display a malicious address which the user cannot
cross-check with the hardware wallet. However, the user manual does urge the
user to always check the two addresses~\cite{trezor-user} to avoid such attack
scenarios.
During transaction issuing, the user again connects the device to the client
and unlocks it. Then they initiate a transaction by giving to the client the
recipient's address, and the payment and fee amounts, similarly to our hybrid
model setting. The client initiates the transaction signing process with the
hardware by providing this data, which the token then displays to the user for
verification. After the user has verified the transaction, the hardware
communicates with the client and signs the needed data. Again, given our high
level investigation, this process matches the communication steps that our
model describes.

\paragraph{Ledger}
We investigate the implementation of Ledger Nano S according to the user
manual~\cite{ledger-manual} and our own analysis. Like Trezor, before
performing any operation the user is required to initiate a session by
connecting the hardware to the client and unlocking it. The hardware provides a
small screen for displaying information and a pair of two buttons for receiving
commands from the user.
During address generation, the client displays the newly generated address to
the user. However, there is no option for the hardware wallet to also display
the address,\footnote{Ledger issued a firmware update to address this issue
and allow both the client and the hardware to generate and display the address;
however, the update process is manual and often
neglected by users.} so that the user can cross-check and verify the
two. This is a clear divergence from our model and allows for address attacks,
\eg by a corrupted client that displays a malicious address to the user.
Transaction issuing is also similar to Trezor and captured by our model. The
user inputs the transaction's data to the client, \ie the recipient's address
and the payment and fee amounts. The client forwards this data to the hardware,
which displays it to the user for verification. After receiving the user's
confirmation, the hardware interacts with the client to sign and publish the
transaction on the blockchain.
