\chapter{
    Macroeconomic Principles
}\label{chap:macroeconomics}

Chapter~\ref{chap:compliance} investigated distributed ledger systems from a
microeconomic perspective, focusing on the incentives and strategic choices of
individual parties. Nonetheless, if such systems are to support real-world
economies, a macroeconomic treatment is also imperative. In this chapter, we
provide some preliminary results on this line of research.

Section~\ref{sec:egalitarianism} explores the limitations in enforcing
macroeconomic policies in distributed ledgers. Importantly, we show that, in
decentralized anonymous systems, no macroeconomic policy which redistributes
wealth from the larger to the smaller parties can be applied. Instead, the best
one can hope for is a linear increase in each party's wealth, proportionally to
their capital. To quantify how different systems fare in this regard, we
introduce the notion of ``cryptocurrency egalitarianism'', a quantitative
metric that helps compare systems \wrt how much they favor wealthy investors.

Next, Section~\ref{sec:taxation} explores how taxation could be enforced. Given
the previous impossibility result, we assume the existence of a centralized
taxation authority. Our goal is to enable the authority to correctly identify
the users' assets, in order to enforce its taxation policy, in a
privacy-preserving and efficient manner. In that direction, we propose two
schemes based on programmable money, \ie currency which is transferable as long
as certain preconditions are met.

\section{
    Cryptocurrency Egalitarianism
}\label{sec:egalitarianism}

In almost all blockchain cryptocurrency systems, block generators are
incentivized to participate via block rewards, \ie for each block they
successfully produce and which is subsequently adopted by all other
participants. In many cryptocurrencies, the rewards serve a dual purpose:
incentivise the the miners/minters but also create and distribute the
underlying cryptocurrency to the system's maintainers. These rewards follow
various schedules that are designed based on the macroeconomic desiderata
envisioned by the architects of the cryptocurrency. For example, the rate of
coin production is halved every $210,000$ blocks in Bitcoin. Ethereum and
Litecoin follow similar schedules.  On the contrary, Monero has a smooth
emission schedule in which the rewards are gradually reduced at every new block
generated. The question of what this schedule should be can have significant
impact on the variance of stake ownership after an execution of a sufficient
number of protocol rounds~\cite{FC:FKORVW19}. Taking this into account, in this
chapter we consider the block generators as investors and focus on the
comparison of the expected returns of investors with different purchasing
power.

A central economic property that arises from this line of thought is
\emph{cryptocurrency egalitarianism} (also ``crypto-egalitarianism''). This
property states that rewards should be proportional to the invested capital.
Therefore, wealthy investors should not be disproportionately rewarded, but
everybody should have equal opportunity to both participate and earn rewards.
Until now, the term crypto-egalitarianism has been left undefined, although
several cryptocurrencies claim to be more egalitarian than
others~\cite{van2013cryptonote,mcmillan2013}. However, lacking a quantifiable
metric, the discussion around egalitarianism remains ill-posed.  The core
contribution of this chapter is to put forth the first concrete definition of
egalitarianism, in a way which is generic, practically measurable, and
applicable to any cryptocurrency. Additionally, we show that wealth
redistribution, from the rich to the poor, is impossible in decentralized,
anonymous (or pseudonymous) systems; thus, rewarding everybody proportionately
to their capital is the best achievable setting.

\paragraph{Related work}
The macro and microeconomics of blockchain design have been studied from
several perspectives, but remain an active area of research with a number of
open questions. \emph{Egalitarianism} in particular has been studied in PoW
systems from the perspective of \emph{memory-hard
functions}~\cite{EC:AlwBloPie17,USENIX:BirKho16}. These works operate under the
premise that memory hardness provides egalitarianism, in the sense that the
cost of one computational step is roughly the same irrespective of the
underlying computational platform (typically ASIC vs. generic). In this chapter
we generalize this question, by asking whether computational power grows
proportionally to capital invested, \ie whether larger wealth results
disproportionately more rewards. Additionally, a notable work by Fanti
\etal~\cite{FC:FKORVW19} introduces the complementary notion of
\emph{equitability}.  That work studies the evolution of a system after a
series of rounds, putting forth the property that stake ownership remains in
proportion \emph{before} and \emph{after} rewards have been awarded. By
studying the behavior of the returns' \emph{variance} under the randomness of
executions, they show that the distribution of capital follows a Pólya process.
This chapter can be seen as complementary to their results, by quantifying the
\emph{expectation} of rewards and then studying the variance under the
randomness of initial capital allocation. Therefore, a cryptocurrency can be
perfectly egalitarian and poorly equitable and vice versa; notably, it is
possible to obtain a cryptocurrency both egalitarian and equitable, by adopting
correctly parameterized PoS under a geometric reward function.

\subsection{PoW vs. PoS}

Before studying the egalitarianism of different cryptocurrency consensus
mechanisms, we consider the leader election process, to establish an
understanding of the differences in egalitarianism between the two models of
PoW and PoS.

\paragraph{Proof-of-Work}
The number of hash evaluations is one of the several critical parameters to
consider when purchasing mining hardware. Other important parameters include
the price of a mining unit, as well as its electricity consumption. Mining
hardware is divided in various tiers based on performance, namely CPU miners,
GPU miners, FPGA miners, and specialized ASIC miners~\cite{taylor2013bitcoin}.
Although the pricing of such devices may be similar, the hashing rate and, in
turn, the Return on Investment, is highly dependent on the hardware's tier.
For example, as of December 2018, the mining hardware ``Whatsminer M10''
produced by the company ``MicroBT'' cost $\$1{,}022.00$ per unit and produces
$\$0.104266$ per hour of operation in net gains, \ie average mined Bitcoins per
hour denominated in US dollars minus the electricity costs. On the other hand,
the mining hardware ``8 Nano Pro'', produced by the company ``ASICMiner'', cost
$\$6{,}000.00$ per unit, but produces $\$0.315327$ per hour of operation in net
gains, \ie almost three times the hourly net gains of its cheaper competitor.
Thus, if one can afford to purchase the more expensive hardware, each of their
subsequent dollar invested in electricity returns more mined coins.

It has long been folklore knowledge in the blockchain community that mining
becomes more egalitarian by using a memory-hard PoW function. This intuition is
correct, the core reason being the difficulty to construct specialized hardware
for memory-hard functions. For example, no ASICs currently exist for Monero
mining. Therefore, the only way to scale mining operations is by purchasing
more general purpose hardware. However, since the mining hardware in this case
varies little, both in terms of cost and performance, scaling returns become
proportional to investments.

In this chapter, we only analyze the scaling of the economics of mining with
respect to hardware. We also do not take into account basic costs such as
shipping and the availability of a basic machine to co-ordinate mining (such as
a personal computer not performing mining itself). A multitude of additional
factors play important roles for mining operations, such as space rental costs,
machine cooling and maintenance costs, or bulk electricity purchase. As is
common in economies of scale, these relative costs are reduced for large-scale
operations, although they are similar for all PoW cryptocurrencies and thus do
not affect relative comparisons between them. We also remark that we analyze
mining costs for small capital investments. If larger capital, \eg above a few
million US dollars, is available, corporations can develop their own
specialized hardware and gain a competitive advantage by treating it as a trade
secret~\cite{taylor2013bitcoin}. These details make the comparison in favour of
PoS \emph{more pronounced}, as PoS operations do not incur such types of costs
and do not lend themselves to specialized mining hardware research.

\paragraph{Proof-of-Stake}
PoS is often criticized for its lack of egalitarianism. The rationale is that,
in PoS, the more money one stakes, the more money one generates. Thus, ``the
rich get richer'', which is precisely the opposite of egalitarianism.
Additionally, in PoS systems, the money owners could constitute a \emph{closed,
rich club}, refusing to share the assets with any outsiders. In contrast, this
argument claims, PoW is naturally egalitarian; everyone is paid based not on
the money they own, but on the computational power they put to work. In this
case, since computational power is a \emph{natural} resource and cannot be
exclusively owned, a closed rich club cannot be formed. Although this argument
seems agreeable at first, the results of this chapter contradict it. In fact,
correctly parameterized stake-based systems are much more egalitarian than
work-based ones.\footnote{Variations of PoS, such as
delegated PoS, may not be perfectly egalitarian, since the delegates, \ie the
leaders of the stake pools, typically earn extra profits for managing the stake
pools~\cite{DBLP:conf/eurosp/BrunjesKKS20}.}

It is instructive to dispel the above argument intuitively, before we support
our position with data. First, the argument that money can be exclusively
owned, but computational power cannot, is rather misguided. Indeed, this may be
true in the case of a peculiar oligopoly, where a small faction of parties
mutually agrees to never sell to outsiders, despite external demand. However,
in an open market, both money and computational power can be freely purchased
and, in fact, any non-negligible amount of computational power must be
necessarily purchased that way.  In this work, we assume an open market for
both mining hardware and financial capital, which allows open participation.
Therefore, given that both money and computational power are purchasable, we
consider the funds one invests, either in technology or in financial capital,
in order to maximize the returns from a cryptocurrency's block generation
mechanisms. The amount of cryptocurrency generated by a given investment can be
concretely measured and compared, thus the question can now be analyzed
quantitatively and answered concretely.

\subsection{A Formal Model of Crypto-Egalitarianism}\label{sec:definition}

The core contribution of this chapter is a formal definition of an economic
measure of \emph{egalitarianism} in cryptocurrencies.

Before we present our definition, let us first state the \emph{desiderata} of
such a definition. First, we want to enable concrete measurements on
cryptocurrencies, in a manner that is quantitative and not vague. Thus far,
egalitarianism claims have been rather informal, failing to include exact
data~\cite{van2013cryptonote,mcmillan2013}. As such, different cryptocurrencies
claim egalitarianism over the others, without demonstrating the claims or
provide conclusive arguments. Second, an egalitarianism definition must measure
the protocol maintenance returns of smaller vs. larger investors.  We thus
desire a metric which extracts a smaller value to indicate a \emph{lack of
egalitarianism} (\eg when large wealth generates blocks disproportionately
faster than small wealth), or a larger value to indicate \emph{perfect
egalitarianism} (\ie when every invested dollar has exactly equal power in
terms of cryptocurrency generation).

The first step in establishing our crypto-egalitarianism definition is to
define the \emph{egalitarian curve} $f$. The horizontal axis of this curve
plots the financial capital, which is available for investment, denominated in
a fiat currency (USD).\footnote{Given that we explore a small investment
duration, it makes little difference whether these are nominal USD or real USD,
as long as they are the same when applying comparisons.} The vertical axis
plots the Return On Investment (ROI), which measures the amount of
cryptocurrency that is created during the investment period and remains unspent
at the end of it, given an optimal allocation of the initial capital. We
require that ROI is computed over \emph{freshly generated} cryptocurrency;
thus, it must be newly-mined or minted, and not purchased from existing
investors. Finally, the curve is plotted with a fixed investment duration in
mind; naturally, curves of different cryptocurrencies can be compared only if
they use the same duration.

\begin{definition}[Egalitarian curve]
    Given a cryptocurrency $c$ and the set of all possible investment
    strategies $\mbb{B}$, we define the \emph{egalitarian curve} $f_{c,d}:
    \mbb{R}^+ \longrightarrow \mbb{R}^+$ of $c$ for an investment period $d$
    as:
    \begin{align}
        f_{c, d}(v) = \frac{\underset{B \in \mbb{B}}{\max}{E[B(v)]} - v}{v}
    \end{align}
\end{definition}

The value $\underset{B \in \mbb{B}}{\max}{E[B(v)]}$ identifies the maximum
expectation of returns across all investment strategies $\mbb{B}$, \ie the
amount of returns which the \emph{optimal} strategy ensures for a given initial
capital $v$. The blockchain execution is modeled as a random variable, since
returns vary by execution; specifically, the randomness of the execution can
affect returns, as a participant may be ``lucky'', \ie produce more blocks than
expected~\cite{FC:FKORVW19}.

We remark that we \emph{do} allow strategies to reinvest capital. For instance,
returns earned from mining rewards can be reinvested in electricity costs for
future mining. Furthermore, for unit consistency, we assume the strategy $B(v)$
returns the freshly generated coins denominated in the same units as the
capital $v$. Second, we assume participants act independently and follow the
protocol according to its specifications.

Using our definition of the egalitarian curve, we now define
(Definition~\ref{def:egalitarianism}) egalitarianism as a concrete number.
Considering the initial capital $v$ as a random variable, which follows a
certain distribution $\mc{D}$, egalitarianism is the variance of the expected
ROI, when the capital is chosen from the given distribution.

\begin{definition}[Egalitarianism]\label{def:egalitarianism}
    Given a cryptocurrency $c$ and its egalitarian curve $f$, we define the
    \emph{egalitarianism} $e$ of $c$, for investment duration $d$ under initial
    capital distribution $\mc{D}$, as follows:
    \begin{align}
      e_{c, d, \mc{D}} = -\msf{Var}_{v \gets \mc{D}}[f_{c, d}(v)]
    \end{align}
\end{definition}

The intuition behind this definition is that, to have egalitarianism, the ROI
must remain the same across different capital investments. As such, any
deviation from the mean is non-egalitarian. Naturally, if a system's
egalitarianism is \emph{higher} than another, we say that the former is
\emph{more egalitarian} than the latter. Of course, to be accurate, such
comparisons must be made after fixing the parameters $c$ and $d$, as well as
the initial capital distribution $\mc{D}$. Fixing $\mathcal{D}$ to be the
uniform distribution between a minimum and a maximum capital, the returns are
the same for all initial capitals alike.

Based on the above, we can define the \emph{ideal egalitarian curve}. First,
as an interesting thought experiment, we consider the egalitarian curve which
is decreasing (and is, arguably, \emph{the} ideal curve). In this case, small
investors would receive proportionally more newly created cryptocurrencies for
every dollar they invest, \ie the system would redistribute wealth from the
rich to the poor. However, one can quickly see that, in decentralized
cryptocurrencies where the identities of the participants are unknown, this is
impossible. Indeed, the fact that decentralized cryptocurrencies allow
anonymous generation of new identities~\cite{douceur2002sybil} allows a wealthy
investor to split their capital into smaller ones. Thus, if the curve were ever
to have a negative slope, the sum of the smaller splits of the rich investment
would achieve a higher gain. By the definition of the curve, which mandates
that it depicts the ROI of an \emph{optimal} investment, this would be a
contradiction. Corollary~\ref{cor:sybil} makes this intuition more precise.

\begin{corollary}[Sybil strategies]\label{cor:sybil}
    Fix a cryptocurrency $c$ and an investment period interval $d$. Given
    capital $v$, for every natural number $i \in \mathbb{N}^\star$, it holds
    that $f_{c,d}(v) \leq f_{c, d}(i \cdot v)$.
\end{corollary}
\begin{proof}
    We prove the statement by contradiction. Assume that, for some capital $v$,
    there exists a natural number $i \in \mathbb{N}^\star$ such that
    $f_{c,d}(v) > f_{c,d}(i \cdot v)$. Also assume that, for capital $v$, the
    optimal strategy is $B'$, so $\underset{B \in
    \mathbb{B}}{\max}{\mathbb{E}[B(v)]} = \mathbb{E}[B'(v)]$. For capital $i
    \cdot v$, there exists a strategy $B''$, such that the capital is split
    into $i$ equally-sized parts, with the strategy $B'$ applied on each part.
    Given that the execution of each such sub-strategy is independent, the
    expected returns for $B''$ are:
    \begin{align}\label{eq:break-strategy}
        \mathbb{E}[B''(i \cdot v)] = i \cdot \mathbb{E}[B'(v)]  = i \cdot \underset{B \in \mathbb{B}}{\max}{\mathbb{E}[B(v)]}
    \end{align}
    Additionally, $B''$ is at best the optimal strategy, so:
    \begin{align}\label{eq:multi-strategy}
        \underset{B \in \mathbb{B}}{\max}{\mathbb{E}[B(i \cdot v)]} \geq \mathbb{E}[B''(i \cdot v)] \xRightarrow{\text{(\ref{eq:break-strategy})}}
        \underset{B \in \mathbb{B}}{\max}{\mathbb{E}[B(i \cdot v)]} \geq i \cdot \underset{B \in \mathbb{B}}{\max}{\mathbb{E}[B(v)]}
    \end{align}
    However, it should also hold that:
    \begin{alignat}{2}
        f_{c,d}(v) &> f_{c,d}(i \cdot v) \Rightarrow \notag\\
        \frac{\underset{B \in \mathbb{B}}{\max}{\mathbb{E}[B(v)]} - v}{v} &> \frac{\underset{B \in \mathbb{B}}{\max}{\mathbb{E}[B(i \cdot v)]} - i \cdot v}{i \cdot v} \xRightarrow{\text{(\ref{eq:multi-strategy})}} \notag\\
        \frac{\underset{B \in \mathbb{B}}{\max}{\mathbb{E}[B(v)]} - v}{v} &> \frac{i \cdot \underset{B \in \mathbb{B}}{\max}{\mathbb{E}[B(v)]} - i \cdot v}{i \cdot v} \Rightarrow \notag\\
        \frac{\underset{B \in \mathbb{B}}{\max}{\mathbb{E}[B(v)]} - v}{v} &> \frac{\underset{B \in \mathbb{B}}{\max}{\mathbb{E}[B(v)]} - v}{v} \notag
    \end{alignat}
    which is impossible.
\end{proof}

Corollary~\ref{cor:sybil} shows that, in purely decentralized systems, a
decreasing egalitarian curve is impossible. Therefore, the next-best ideal
curve is a constant one, where the ROI is stable regardless of capital
invested. Under this condition, the amount of freshly generated cryptocurrency
is exactly proportional to the money invested. Consequently, a cryptocurrency
with an ideal egalitarian curve is perfectly egalitarian
(Definition~\ref{def:perfect-egalitarianism}).

\begin{definition}[Perfect egalitarianism]\label{def:perfect-egalitarianism}
    A cryptocurrency $c$ is \emph{perfectly egalitarian}, for investment
    duration $d$ and initial capital distribution $\mc{D}$, if $e_{c, d, \mc{D}}
    = 0$.
\end{definition}

\subsection{Discussion}

In this chapter, in providing a concrete definition of crypto-egalitarianism,
we enable an evidence-based discussion to substitute folklore arguments.
The first application of this metric was provided
in~\cite{karakostas2019cryptocurrency}, which compared some of the largest
cryptocurrency systems to date.  Using our model, the egalitarianism of four
indicative PoW-based cryptocurrencies (Bitcoin,
Litecoin~\cite{lee2011litecoin}, Ethereum~\cite{buterin,wood2014ethereum}, and
Monero~\cite{van2013cryptonote}) was measured.  The assessed claims of these
projects were found in agreement with our data, thus presenting for the first
time economic comparisons which quantify them precisely. On the pure PoS side,
it was shown that egalitarian behavior is similar across all coins,
independently of externalities such as hardware characteristics. Therefore, it
suffices to perform a case study of an indicative PoS protocol (in this case,
Ouroboros~\cite{C:KRDO17}).  It was then shown that pure PoS coins can be
perfectly egalitarian, contrary to their PoW counterparts.
These results were very optimistic in terms of usability of our metric, as they
provide concrete figures which measure the egalitarianism of several popular
cryptocurrencies. The most unexpected result arised from the comparison between
the PoW and PoS mechanisms. Although blockchain folklore argued in favour of
PoW systems in terms of egalitarianism, these results show that, in fact, it is
PoS systems which are more egalitarian under our proposed model.

Another interesting property that arises from this work is the impossibility of
wealth redistribution in cryptocurrency systems. As shown in
Corollary~\ref{cor:sybil}, in a purely decentralized setting, where no
real-world identity checks exist (and, arguably, cannot exist), a wealthy
participant can always pose as multiple poor users. Therefore, any attempt to
redistribute wealth from the rich to the poor, based on entirely technological
tools and without taking into account real-world social structures, seems
doomed to fail. In conclusion, all decentralized cryptocurrencies are ``rich
get richer'' schemes; the pertinent question, which this chapter aimed at
resolving, is \emph{how fast} this takes place.


\section{
    Tax Applications of Programmable Money
}\label{sec:taxation}

A tax gap~\cite{comission2018taxgaps} is a difference between the reported and
the real tax revenue, for a given jurisdiction and period of time. Research
estimated that the tax gap in the USA was $16.4$\% of revenue
owed~\cite{internal2016federal} between 2008-2010, the total loss throughout
the EU due to the tax gap to €$151.5$ billion in 2015~\cite{murphy2018resources}, while
$\frac{1}{3}$ of taxpayers in the UK under-report their
earnings~\cite{advani2020does} (albeit half of UK's lost taxes are product of a
small, wealthy fraction of misbehaving taxpayers). Therefore, reducing the tax
gaps can significantly enhance the efforts of tax-collecting authorities.

Central bank digital currencies (CBDC) have also come to prominence in recent
years. In the past decade, distributed ledger-based financial systems, which
were kick-started with the creation of Bitcoin~\cite{nakamoto2008bitcoin}, were
accompanied by the increasing digitalization of payments~\cite{bis2011digital}.
CBDCs are the culmination of these trends, enabling fast, cheap, and safe
transactions in fiat assets. Crucially though, although still mostly on a
research stage,\footnote{\url{https://cbdctracker.org} [July 2021]} CBDCs have
caused great concerns on citizens regarding transaction
privacy~\cite{ecb2021cbdcprivacy}.

This chapter offers two mechanisms that facilitate tax auditing and the
identification of tax gaps in distributed ledger-based currency systems. The
first is a wrapper around a generic distributed ledger, which enables taxpayers
to declare their assets directly to the authorities, while undeclared assets
are frozen. The second is a proof mechanism that enables the sender of some
assets to prove, in a privacy-preserving manner, whether the transferred assets
have been taxed. Both mechanisms are examples of programmable money (also
referred to as smart money~\cite{AHA}), where currency is programmed to be
transferable under a suitable set of  circumstances or its transfer has
specific implications.

\paragraph{Related work}
Literature offers various works on auditing of distributed ledger-based assets.
A holistic approach is taken in zkLedger~\cite{narula2018zkledger}, which
combines a permissioned ledger with zero-knowledge proofs to create a
tamper-resistant, verifiable ledger of transactions.
PRCash~\cite{EPRINT:WKCC18} also employs a permissioned ledger and offers a
regulation mechanism that restricts the total amount of assets a user can
receive anonymously for a period of time. Also Garman
\etal~\cite{FC:GarGreMie16} propose an anonymous ledger, which can enforce
specific transaction policies. Following, Section~\ref{sec:taxchain} aims at
offering a simpler design, which can be more easily integrated in existing
pseudonymous distributed ledgers, compared to the aforementioned works. Another
interesting research thread considers proofs of solvency. The first such scheme
for Bitcoin exchanges, proposed by Maxwell~\cite{wilcox2014proving}, leaks the
total amount of both assets and liabilities of the exchange; more importantly,
it enables an attack that allows the exchange to hide assets, as detailed by in
Zeroledge~\cite{doernerzeroledge}, which also proposed a privacy-preserving
system that allows exchanges to prove properties about their holdings.
Provisions~\cite{CCS:DBBCB15} is a zero-knowledge proof of solvency mechanism
for Bitcoin exchanges, based on Sigma protocols \ie without the need to reveal
the addresses or the amount of assets that an exchange controls. Similarly,
Agrawal \etal~\cite{C:AgrGanMoh18} describe a proof of solvency which achieves
better performance compared to Provisions, although assuming a trusted setup.
The mechanism of Section~\ref{sec:provisions-extension} extends
Provisions and is also applicable to~\cite{C:AgrGanMoh18}.

\subsection{Desiderata}\label{sec:taxation-desiderata}

In distributed ledger-based currency systems, a user $\user$ manages their
assets via addresses. Each address $\addr$ is associated with a key pair
$\keypair$, such that the private key $\privkey$ is used to claim ownership of
the assets, \eg by signing special messages; typically $\addr =
\hash(\pubkey)$ for some hash function $\hash$. Each address $\addr$ is
associated with a (public) balance $\balance(\addr)$ so, given a list
$[\addr_1, \dots, \addr_n]$ of all addresses that $\user$ controls,
$\user$'s total assets are $\assets = \sum_{i=1}^n \balance(\addr_i)$. Our
goal will be to retain as much privacy as possible, so $\assets$ should be the
only information that is leaked to $\taxAuth$, without
de-anonymization of individual transaction data.

To showcase the limitations of current systems, consider the following example.
Assume that Alice tax evades, \ie creates a secret, undeclared address
$\addr$ and controls some assets $\asset$ in it. Given the pseudonymous
nature of the ledger, $\addr$ cannot be correlated with Alice, until she
uses it. Following, Alice issues a transaction $\tau$ which sends $\theta$
assets from $\addr$ to Bob. If Bob suspects that Alice evaded taxation for
these $\theta$ assets, they might want to report her to the authorities for
inspection. However, the complaint should be accompanied by a proof that
$\addr$ is controlled by Alice, \ie a proof that Alice knows the private key
associated with $\addr$. This is necessary as $\taxAuth$ needs to
distinguish between two scenarios:
\begin{inparaenum}[i)]
    \item Alice controls $\addr$ and tax evades;
    \item Bob is lying about Alice owning $\addr$.
\end{inparaenum}
In the first scenario, Bob \emph{does} know that $\addr$ is controlled by
Alice, but $\tau$ is not sufficient to prove it.
Instead, Bob needs a proof which can only be supplied by Alice, \eg a signature
from Alice which acknowledges $\tau$ or $\addr$. However, if Alice tax
evades, naturally she would not create such incriminating proof.

It is important that we retain as many good features of existing ledger systems
as possible. The most notable such feature is transaction privacy, thus our
work considers pseudonymous, Bitcoin-like levels of privacy, and minimizes the
information leaked to the authorities during a tax auditing. Another important
aspect is the mechanism's performance. A fundamental ingredient of payment
systems is the seamless transaction experience, so it is important to allow
users to transact at all times, while also avoiding significant strain during
taxation periods. Finally, our mechanisms aim to minimize the amount of
(additional) published data, since storage in
distributed ledgers is particularly costly.

In summary, the desiderata of our mechanisms are as follows:
\begin{itemize}
    \item \emph{Tax gap identification and counterincentive}: Tax evasion, \ie failure of a user
        $\user$ to declare the amount of assets they own, should be either
        detectable by a tax authority $\taxAuth$, with access to the
        ledger, or render the assets unusable.
    \item \emph{High level of privacy}: $\taxAuth$ should --- at most ---
        learn the total amount of assets owned by each taxpayer at the end of a
        fiscal year; this information should be leaked only to $\taxAuth$ and
        no additional information should be leaked to any other party, apart
        from the information already published on the ledger.
    \item \emph{Unobstructed operation}: The introduction of a taxation
        mechanism should not result in any period during which the --- tax
        compliant --- users are prohibited from transacting.
    \item \emph{Low performance overhead}: The taxation mechanism should not
        introduce a major performance overhead, in terms of computation and
        storage requirements from the users and the taxation authority.
    \item \emph{Balanced load}: The computation and storage overhead of
        taxation should be spread over a period of time, rather than introduce
        performance spikes.
\end{itemize}

\subsection{Tax Auditable Distributed Ledger}\label{sec:taxchain}

In this section we describe a ledger with a built-in tax auditing mechanism.
Our design is generic, such that existing ledgers can incorporate it with
minimal changes in the underlying consensus protocol. An \emph{auditable
ledger} enforces a user $\user$ to declare the amount of assets they own to a
taxation authority $\taxAuth$, with failure to do so rendering the assets
unusable. We achieve this while leaking to $\taxAuth$ only the total amount of
assets that $\user$ owns at a specific point in time, \eg the end of a fiscal
year. We note that we consider only pseudonymous ledgers, so potentially
de-anonymizable data may be published on the ledger, \eg addresses which may be
linked to the user who controls them.

We assume that $\taxAuth$ holds a list of all taxpayers and is identified by a
key $(\privkey_{\taxAuth}, \pubkey_{\taxAuth})$. Also there exist taxation
periods, which last for a pre-specified amount of time $d$. For example, a
taxation period may last $1$ calendar year, at the end of which taxpayers need
to declare their assets to the authorities.

The core idea is that assets unaccounted for, at the end of the taxation
period, are frozen, until their owners declare them to the authority.
Specifically, at the end of a taxation period, all assets are frozen. To
unfreeze an asset, a taxpayer $\user$ declares it to $\taxAuth$ as follows.

First, $\user$ creates a new key pair $(\privkey_{\user}, \pubkey_{\user})$
and the corresponding address $\addr_{\user}$ and sends $\addr_{\user}$
to $\taxAuth$ as part of a KYC process.  Next, $\taxAuth$ certifies $\addr_{\user}$ by issuing the
signature $\sig = \algosign(\addr_{\user}, \privkey_{\taxAuth})$, which it
gives to $\user$. The tuple $\addr_{\user}^{t} = \langle \addr_{\user},
\sig \rangle$ is the \emph{certified address}, which is used by the user to
transact with frozen assets. $\taxAuth$ maintains a mapping of taxpayers and
certified addresses, \ie for every taxpayer $\user$ it holds a list $A_{\user}$
of all certified taxation addresses that $\user$ requested.

A transaction $\tau = \langle \addr_{s}, \addr_{d}, \assets \rangle$,
which moves $\assets$ frozen assets from an address $\addr_{s}$, is valid
only if $\addr_{d} = \langle \addr, \sig \rangle \land
\algoverify(\addr, \sig, \pubkey_{\taxAuth}) = 1$. Consequently, miners
accept transactions that unfreeze assets only as long as said assets are
transferred to a certified address. Therefore, $\taxAuth$ can compute the
amount of $\user$'s assets as $\assets_{\user} := \sum_{i=1}^{n}
\balance(\addr_{\user}[i])$, $n$ being the total number of $\user$'s
certified addresses.

We note that the system can accommodate multiple taxation authorities from
different countries. In that case, $\taxAuth$ is a federation of authorities,
each identified by a single key. Each authority's key is published on the
ledger and a taxpayer can certify their addresses and declare their assets to
the respective authorities.

Naturally, this mechanism introduces some challenges. Although standard
pay-to-public-key-hash addresses are $25$ bytes, certified addresses may be
significantly larger, due to the certification signature of $\taxAuth$. For
instance, ECDSA signatures in the DER format result in $72$ additional bytes,
thus making certified addresses $99$ bytes long. Nevertheless, certified
addresses are expected to be used only once, to declare the assets, thus the
overall storage cost should not be significant. Another important consideration
regards to the private state of the taxation authority; given the statute of
limitations, $\taxAuth$ might need to maintain its taxation private key and the
mapping of certified addresses for a significant period,
possibly resulting in significant maintenance costs.

We showcase our design via an auditable variation of Bitcoin ledger, denoted as
$\taxBtc$. $\taxBtc$ is initially parameterized by the public key of the
authority $(\privkey_{\taxAuth}, \pubkey_{\taxAuth})$, which is
part of the ledger's genesis block. During the execution, $\taxAuth$ can update
its key by simply signing a new key $\pubkey_{\taxAuth}'$ with
$\privkey_{\taxAuth}$ and publishing it on the ledger. A taxation period lasts
$52560$ blocks, \ie roughly $1$ calendar year, so block $52560$ and its
multiples are ``tax-auditing'' blocks.  When a tax-auditing block is issued,
all assets on $\taxBtc$ which are controlled by non-certified addresses are
frozen. To transact with assets from a frozen address, a user sends them to a
certified address, as described above.

Freezing complicates the system in a number of ways. First, the liveness
of a transaction~\cite{EC:GarKiaLeo15} may be affected. For
instance, a transaction which spends from a non-certified address will be
rejected, if it is created before but published after a tax-auditing block. We
sidestep this issue by enabling users to use certified addresses before the
freezing period, hence the liveness guarantees of the ledger apply
unconditionally on certified addresses. Second, it is possible that multiple
competing tax-auditing blocks are created, \eg multiple blocks which extend the
tax-auditing block. Therefore, $\taxAuth$ needs to pick one and certify it.
Afterwards, this certified block cannot be reverted and acts as a
``checkpoint''.

We note that $\taxBtc$ covers the desiderata proposed in
Section~\ref{sec:taxation-desiderata}. Regarding privacy, although $\taxAuth$ can
de-anonymize the set of $\taxBtc$ users at a specific point in time, \ie when
the assets freeze, the users can employ standard Bitcoin addresses and
transactions outwith this period. Additionally, as with standard Bitcoin
addresses, third parties cannot obtain information regarding the identity of a
certified address's owner (as long as the signature itself does not leak it).
In terms of performance, a user can transact with their assets effortlessly, as
long as they use certified addresses to receive or unfreeze assets around the
taxation period. Importantly, users can certify their addresses ahead
of the freezing time, thus the additional load can be spread over a period
of a few days or weeks.

\subsection{A Tax-Auditing Extension for Provisions}\label{sec:provisions-extension}

We now build a tax auditing mechanism for existing ledgers, which is based on
Provisions~\cite{CCS:DBBCB15}. The goal of this mechanism is to enable all payment recipients to verify whether the assets used by a sender $\exchange$ in a transaction have been
properly declared to the authority $\taxAuth$. This is achieved in two stages, first with an asset declaration stage that involves $\taxAuth$ and second with a payer address auditing protocol, which is created
in tandem with the transaction that pays a recipient,
and after $\exchange$ commits to owning the assets. If $\exchange$ fails to provide such proof, the implication is that $\exchange$ performs tax evasion.
To build this proof mechanism we rely on Provisions~\cite{CCS:DBBCB15},
particularly its \emph{proof of assets}. Our scheme comprises of two simple
protocols, which $\exchange$ runs with the taxation authority and their
counter-party respectively. As we show, our protocols retain the privacy
guarantees of Provisions.

Provisions is a privacy-preserving auditing mechanism for Bitcoin exchanges.
Using Provisions a party can verify that a (cooperating) Bitcoin exchange is
solvent, \ie possesses enough assets to cover the liabilities towards its
users. In order to achieve this, Provisions defines three protocols:
\begin{inparaenum}[i)]
    \item proof of assets,
    \item proof of liabilities, and
    \item proof of solvency.
\end{inparaenum}
The first protocol commits the exchange --- in a zero-knowledge fashion --- to
the total amount of assets it possesses. The second commits it to the
liabilities towards its clients, such that each client can verify that the
exchange has included his/her deposits in the collective proof. Finally, the
proof of solvency proves that the exchange's assets are equal or surpass its
liabilities.
Our work is only concerned in the assets owned by the exchange, thus we focus
on the proof of assets. All proofs are considered under a group $G$ of prime
order $q$ with fixed public generators $g, h$. The proof of assets considers
the following:
\begin{itemize}
    \item $\text{PK} = \{y_1, \dots, y_n \}$: the total (anonymity) set of public keys;
    \item $s_i$: a bit such that, if the exchange controls $y_i$, \ie if it possesses the private key of $y_i$, then $s_i = 1$, otherwise $s_i = 0$;
    \item $\balance(y_i)$: the amount of assets that the address corresponding to $y_i$ controls;
    \item $\assets = \sum_{i = 1}^n s_i \cdot \balance(y_i)$: the amount of assets that the exchange controls;
    \item $b_i = g^{\balance(y_i)}$: a biding (but not hiding) commitment  to the balance of $y_i$.
\end{itemize}
The exchange publishes the Pedersen commitments~\cite{C:Pedersen91} for each $s_i \cdot
\balance(y_i), s_i$:
\begin{align}
    p_i = b_i^{s_i} \cdot h^{v_i} = g^{\balance(y_i) \cdot s_i} \cdot h^{v_i} \label{eq:balance-commit} \\
    l_i = y_i^{s_i}h^{t_i} =  g^{\hat{x}_i}h^{t_i} \label{eq:ownership-commit}
\end{align}
where $v_i, t_i \in \mathbb{Z}_q$ are chosen at random,
$x_i$ is the private key for $y_i$, and $\hat{x}_i = x_i \cdot s_i$.
Then, the exchange proves knowledge of values $s_i, v_i, t_i, \hat{x}_i$ for every $i
\in [1, n]$ via a $\Sigma$-protocol, such that conditions
(\ref{eq:balance-commit}), (\ref{eq:ownership-commit}) are satisfied.

\paragraph{Asset Declaration}\label{subsec:tax-authority-proto}
In our case, $\exchange$ declares the total amount of assets
it controls, \ie the value $\assets$,
to  $\taxAuth$ who verifies  that $\exchange$'s commitments
correspond to $\assets$. We obtain the condition
$Z_\assets = \prod_{i = 1}^n p_i = g^{\assets} \cdot h^v$,
where $v = {\sum_{i = 1}^n v_i}$, is a (publicly
computable) Pedersen commitment to $\exchange$'s assets. Given that $\taxAuth$
knows $\assets$, $\exchange$ needs only to prove knowledge of a value $v$, such
that this condition is satisfied. This is done via the Schnorr
protocol~\cite{C:Schnorr89} of Figure~\ref{fig:taxation_auth_proto}, which
guarantees privacy as described in Lemma~\ref{thm:tax-auth-proto}.

\myhalfbox{Asset Declaration Protocol $\taxationProto$}{white!40}{white!10}{
    Public data: $g, h, Z_\assets = \prod_{i = 1}^n p_i$

    Verifier's input from prover: $\assets$

    Prover's input: $v = \sum_{i = 1}^n v_i$
    \begin{enumerate}
        \item The prover ($\exchange$) chooses $r \xleftarrow{\$} \mathbb{Z}_q$
            and sends $\lambda = h^r$ to the verifier ($\taxAuth$).
        \item The verifier replies with a challenge $c \xleftarrow{\$} \mathbb{Z}_q$.
        \item The prover responds with $\theta = r + c \cdot v$.
        \item The verifier accepts if $h^\theta \stackrel{?}{=} \lambda \cdot (Z_\assets \cdot g^{-\assets})^c$.
    \end{enumerate}
}{\label{fig:taxation_auth_proto} Tax-auditing between $\exchange$ (prover) and $\taxAuth$ (verifier).}


\begin{lemma}\label{thm:tax-auth-proto}
    For public values $g, h$ and $Z_\assets$, the protocol $\taxationProto$ is an
    honest-verifier zero-knowledge argument of knowledge of quantity $v$
    satisfying
    $Z_\assets = \prod_{i = 1}^n p_i = g^{\assets} \cdot h^v$ for $i \in [1, n]$.
\end{lemma}

\paragraph{Payer Address Auditing}\label{subsec:user-verification-proto}
The second part of our taxation proof enables the tax auditing of a specific
address used by a payer $\exchange$ whenever a payment is made to an arbitrary  user $\user$. $\exchange$ will prove two conditions to
$\user$:
\begin{inparaenum}[i)]
    \item for some $i \in [1, n]$, the public key $y_i$ (which is published as
        part of the Provisions scheme) corresponds to the address from which
        $\user$ receives their assets;
    \item the corresponding bit $s_i$ for $y_i$ in the commitment condition
        (\ref{eq:ownership-commit}) is $s_i = 1$.
\end{inparaenum}
The first condition can be easily proven by providing $\user$ with an index
$i$, such that $\user$ confirms that the address in question is equal to the
hash of $y_i$. To prove the second condition, we observe that, for $s_i = 1$,
$p_i = g^{\balance(y_i)}h^{v_i}$ and
$l_i = y_ih^{t_i}$.
Therefore, $\exchange$ needs only to prove knowledge of $t_i$ and $v_i$, such that this
statement is satisfied, which can be achieved via the Schnorr protocol
of Figure~\ref{fig:taxation_verification_proto}, its privacy properties formalized in
Lemma~\ref{thm:user-proto}.

\myhalfbox{Address Verification Protocol $\taxationAddressProto$}{white!40}{white!10}{
    Public data: $h$, $(y_i, l_i), \balance(y_i)$ for $i \in [1, n]$

    Verifier's input from prover: $i$

    Prover's input: $t_i$
    \begin{enumerate}
        \item The prover ($\exchange$) chooses $r_1, r_2 \xleftarrow{\$} \mathbb{Z}_q$
            and sends $\lambda_1 = h^{r_1}, \lambda_2 = h^{r_2}$ to the verifier.
        \item The verifier replies with a challenge $c \xleftarrow{\$} \mathbb{Z}_q$.
        \item The prover responds with $\theta_1 = r_1 + c \cdot t_i$,
        $\theta_2 = r_2 + c \cdot v_i$.
        \item The verifier accepts if $h^{\theta_1} \stackrel{?}{=} \lambda_1 \cdot (l_i \cdot y_i^{-1})^c$
        and $h^{\theta_2} \stackrel{?}{=} \lambda_2 \cdot (p_i \cdot g^{-\balance(y_i)})^c$.
    \end{enumerate}
}{\label{fig:taxation_verification_proto} Address verification between $\exchange$ (prover) and a user $\user$ (verifier).}

\begin{lemma}\label{thm:user-proto}
    For public values $g, h$ and $y_i, l_i, p_i, \balance(y_i)$, the protocol
    $\taxationAddressProto$ is an honest-verifier zero-knowledge argument of
    knowledge of quantities $t_i, v_i$ satisfying $l_i = y_ih^{t_i}$ and $p_i =
    g^{\balance(y_i)}h^{v_i}$ respectively.
\end{lemma}

Finally, both protocols can be turned into non-interactive zero-knowledge
(NIZK) proofs of knowledge, in the random oracle model, using the
Fiat-Shamir transformation~\cite{C:FiaSha86}.


