\chapter{
    Macroeconomic Principles
}\label{chap:macroeconomics}

Chapter~\ref{chap:compliance} investigated distributed ledger systems from a
microeconomic perspective, focusing on the incentives and strategic choices of
individual parties. Nonetheless, if such systems are to support real-world
economies, a macroeconomic treatment is also imperative. In this chapter, we
provide some preliminary results on this line of research.

Section~\ref{sec:egalitarianism} explores the limitations in enforcing
macroeconomic policies in distributed ledgers. Importantly, we show that, in
decentralized anonymous systems, no macroeconomic policy which redistributes
wealth from the larger to the smaller parties can be applied. Instead, the best
one can hope for is a linear increase in each party's wealth, proportionally to
their capital. To quantify how different systems fare in this regard, we
introduce the notion of ``cryptocurrency egalitarianism'', a quantitative
metric that helps compare systems \wrt how much they favor wealthy investors.

Next, Section~\ref{sec:taxation} explores how taxation could be enforced. Given
the previous impossibility result, we assume the existence of a centralized
taxation authority. Our goal is to enable the authority to correctly identify
the users' assets, in order to enforce its taxation policy, in a
privacy-preserving and efficient manner. In that direction, we propose two
schemes based on programmable money, \ie currency which is transferable as long
as certain preconditions are met.

\section{
    Cryptocurrency Egalitarianism
}\label{sec:egalitarianism}

In almost all blockchain cryptocurrency systems, block generators are
incentivized to participate via block rewards, \ie for each block they
successfully produce and which is subsequently adopted by all other
participants. In many cryptocurrencies, the rewards serve a dual purpose:
incentivise the the miners/minters but also create and distribute the
underlying cryptocurrency to the system's maintainers. These rewards follow
various schedules that are designed based on the macroeconomic desiderata
envisioned by the architects of the cryptocurrency. For example, the rate of
coin production is halved every $210,000$ blocks in Bitcoin. Ethereum and
Litecoin follow similar schedules.  On the contrary, Monero has a smooth
emission schedule in which the rewards are gradually reduced at every new block
generated. The question of what this schedule should be can have significant
impact on the variance of stake ownership after an execution of a sufficient
number of protocol rounds~\cite{FC:FKORVW19}. Taking this into account, in this
chapter we consider the block generators as investors and focus on the
comparison of the expected returns of investors with different purchasing
power.

A central economic property that arises from this line of thought is
\emph{cryptocurrency egalitarianism} (also ``crypto-egalitarianism''). This
property states that rewards should be proportional to the invested capital.
Therefore, wealthy investors should not be disproportionately rewarded, but
everybody should have equal opportunity to both participate and earn rewards.
Until now, the term crypto-egalitarianism has been left undefined, although
several cryptocurrencies claim to be more egalitarian than
others~\cite{van2013cryptonote,mcmillan2013}. However, lacking a quantifiable
metric, the discussion around egalitarianism remains ill-posed.  The core
contribution of this chapter is to put forth the first concrete definition of
egalitarianism, in a way which is generic, practically measurable, and
applicable to any cryptocurrency. Additionally, we show that wealth
redistribution, from the rich to the poor, is impossible in decentralized,
anonymous (or pseudonymous) systems; thus, rewarding everybody proportionately
to their capital is the best achievable setting.

\paragraph{Related work}
The macro and microeconomics of blockchain design have been studied from
several perspectives, but remain an active area of research with a number of
open questions. \emph{Egalitarianism} in particular has been studied in PoW
systems from the perspective of \emph{memory-hard
functions}~\cite{EC:AlwBloPie17,USENIX:BirKho16}. These works operate under the
premise that memory hardness provides egalitarianism, in the sense that the
cost of one computational step is roughly the same irrespective of the
underlying computational platform (typically ASIC vs. generic). In this chapter
we generalize this question, by asking whether computational power grows
proportionally to capital invested, \ie whether larger wealth results
disproportionately more rewards. Additionally, a notable work by Fanti
\etal~\cite{FC:FKORVW19} introduces the complementary notion of
\emph{equitability}.  That work studies the evolution of a system after a
series of rounds, putting forth the property that stake ownership remains in
proportion \emph{before} and \emph{after} rewards have been awarded. By
studying the behavior of the returns' \emph{variance} under the randomness of
executions, they show that the distribution of capital follows a Pólya process.
This chapter can be seen as complementary to their results, by quantifying the
\emph{expectation} of rewards and then studying the variance under the
randomness of initial capital allocation. Therefore, a cryptocurrency can be
perfectly egalitarian and poorly equitable and vice versa; notably, it is
possible to obtain a cryptocurrency both egalitarian and equitable, by adopting
correctly parameterized PoS under a geometric reward function.

\subsection{PoW vs. PoS}

Before studying the egalitarianism of different cryptocurrency consensus
mechanisms, we consider the leader election process, to establish an
understanding of the differences in egalitarianism between the two models of
PoW and PoS.

\paragraph{Proof-of-Work}
The number of hash evaluations is one of the several critical parameters to
consider when purchasing mining hardware. Other important parameters include
the price of a mining unit, as well as its electricity consumption. Mining
hardware is divided in various tiers based on performance, namely CPU miners,
GPU miners, FPGA miners, and specialized ASIC miners~\cite{taylor2013bitcoin}.
Although the pricing of such devices may be similar, the hashing rate and, in
turn, the Return on Investment, is highly dependent on the hardware's tier.
For example, as of December 2018, the mining hardware ``Whatsminer M10''
produced by the company ``MicroBT'' cost $\$1{,}022.00$ per unit and produces
$\$0.104266$ per hour of operation in net gains, \ie average mined Bitcoins per
hour denominated in US dollars minus the electricity costs. On the other hand,
the mining hardware ``8 Nano Pro'', produced by the company ``ASICMiner'', cost
$\$6{,}000.00$ per unit, but produces $\$0.315327$ per hour of operation in net
gains, \ie almost three times the hourly net gains of its cheaper competitor.
Thus, if one can afford to purchase the more expensive hardware, each of their
subsequent dollar invested in electricity returns more mined coins.

It has long been folklore knowledge in the blockchain community that mining
becomes more egalitarian by using a memory-hard PoW function. This intuition is
correct, the core reason being the difficulty to construct specialized hardware
for memory-hard functions. For example, no ASICs currently exist for Monero
mining. Therefore, the only way to scale mining operations is by purchasing
more general purpose hardware. However, since the mining hardware in this case
varies little, both in terms of cost and performance, scaling returns become
proportional to investments.

In this chapter, we only analyze the scaling of the economics of mining with
respect to hardware. We also do not take into account basic costs such as
shipping and the availability of a basic machine to co-ordinate mining (such as
a personal computer not performing mining itself). A multitude of additional
factors play important roles for mining operations, such as space rental costs,
machine cooling and maintenance costs, or bulk electricity purchase. As is
common in economies of scale, these relative costs are reduced for large-scale
operations, although they are similar for all PoW cryptocurrencies and thus do
not affect relative comparisons between them. We also remark that we analyze
mining costs for small capital investments. If larger capital, \eg above a few
million US dollars, is available, corporations can develop their own
specialized hardware and gain a competitive advantage by treating it as a trade
secret~\cite{taylor2013bitcoin}. These details make the comparison in favour of
PoS \emph{more pronounced}, as PoS operations do not incur such types of costs
and do not lend themselves to specialized mining hardware research.

\paragraph{Proof-of-Stake}
PoS is often criticized for its lack of egalitarianism. The rationale is that,
in PoS, the more money one stakes, the more money one generates. Thus, ``the
rich get richer'', which is precisely the opposite of egalitarianism.
Additionally, in PoS systems, the money owners could constitute a \emph{closed,
rich club}, refusing to share the assets with any outsiders. In contrast, this
argument claims, PoW is naturally egalitarian; everyone is paid based not on
the money they own, but on the computational power they put to work. In this
case, since computational power is a \emph{natural} resource and cannot be
exclusively owned, a closed rich club cannot be formed. Although this argument
seems agreeable at first, the results of this chapter contradict it. In fact,
correctly parameterized stake-based systems are much more egalitarian than
work-based ones.\footnote{Variations of PoS, such as
delegated PoS, may not be perfectly egalitarian, since the delegates, \ie the
leaders of the stake pools, typically earn extra profits for managing the stake
pools~\cite{DBLP:conf/eurosp/BrunjesKKS20}.}

It is instructive to dispel the above argument intuitively, before we support
our position with data. First, the argument that money can be exclusively
owned, but computational power cannot, is rather misguided. Indeed, this may be
true in the case of a peculiar oligopoly, where a small faction of parties
mutually agrees to never sell to outsiders, despite external demand. However,
in an open market, both money and computational power can be freely purchased
and, in fact, any non-negligible amount of computational power must be
necessarily purchased that way.  In this work, we assume an open market for
both mining hardware and financial capital, which allows open participation.
Therefore, given that both money and computational power are purchasable, we
consider the funds one invests, either in technology or in financial capital,
in order to maximize the returns from a cryptocurrency's block generation
mechanisms. The amount of cryptocurrency generated by a given investment can be
concretely measured and compared, thus the question can now be analyzed
quantitatively and answered concretely.

\subsection{A Formal Model of Crypto-Egalitarianism}\label{sec:definition}

The core contribution of this chapter is a formal definition of an economic
measure of \emph{egalitarianism} in cryptocurrencies.

Before we present our definition, let us first state the \emph{desiderata} of
such a definition. First, we want to enable concrete measurements on
cryptocurrencies, in a manner that is quantitative and not vague. Thus far,
egalitarianism claims have been rather informal, failing to include exact
data~\cite{van2013cryptonote,mcmillan2013}. As such, different cryptocurrencies
claim egalitarianism over the others, without demonstrating the claims or
provide conclusive arguments. Second, an egalitarianism definition must measure
the protocol maintenance returns of smaller vs. larger investors.  We thus
desire a metric which extracts a smaller value to indicate a \emph{lack of
egalitarianism} (\eg when large wealth generates blocks disproportionately
faster than small wealth), or a larger value to indicate \emph{perfect
egalitarianism} (\ie when every invested dollar has exactly equal power in
terms of cryptocurrency generation).

The first step in establishing our crypto-egalitarianism definition is to
define the \emph{egalitarian curve} $f$. The horizontal axis of this curve
plots the financial capital, which is available for investment, denominated in
a fiat currency (USD).\footnote{Given that we explore a small investment
duration, it makes little difference whether these are nominal USD or real USD,
as long as they are the same when applying comparisons.} The vertical axis
plots the Return On Investment (ROI), which measures the amount of
cryptocurrency that is created during the investment period and remains unspent
at the end of it, given an optimal allocation of the initial capital. We
require that ROI is computed over \emph{freshly generated} cryptocurrency;
thus, it must be newly-mined or minted, and not purchased from existing
investors. Finally, the curve is plotted with a fixed investment duration in
mind; naturally, curves of different cryptocurrencies can be compared only if
they use the same duration.

\begin{definition}[Egalitarian curve]
    Given a cryptocurrency $c$ and the set of all possible investment
    strategies $\mbb{B}$, we define the \emph{egalitarian curve} $f_{c,d}:
    \mbb{R}^+ \longrightarrow \mbb{R}^+$ of $c$ for an investment period $d$
    as:
    \begin{align}
        f_{c, d}(v) = \frac{\underset{B \in \mbb{B}}{\max}{E[B(v)]} - v}{v}
    \end{align}
\end{definition}

The value $\underset{B \in \mbb{B}}{\max}{E[B(v)]}$ identifies the maximum
expectation of returns across all investment strategies $\mbb{B}$, \ie the
amount of returns which the \emph{optimal} strategy ensures for a given initial
capital $v$. The blockchain execution is modeled as a random variable, since
returns vary by execution; specifically, the randomness of the execution can
affect returns, as a participant may be ``lucky'', \ie produce more blocks than
expected~\cite{FC:FKORVW19}.

We remark that we \emph{do} allow strategies to reinvest capital. For instance,
returns earned from mining rewards can be reinvested in electricity costs for
future mining. Furthermore, for unit consistency, we assume the strategy $B(v)$
returns the freshly generated coins denominated in the same units as the
capital $v$. Second, we assume participants act independently and follow the
protocol according to its specifications.

Using our definition of the egalitarian curve, we now define
(Definition~\ref{def:egalitarianism}) egalitarianism as a concrete number.
Considering the initial capital $v$ as a random variable, which follows a
certain distribution $\mc{D}$, egalitarianism is the variance of the expected
ROI, when the capital is chosen from the given distribution.

\begin{definition}[Egalitarianism]\label{def:egalitarianism}
    Given a cryptocurrency $c$ and its egalitarian curve $f$, we define the
    \emph{egalitarianism} $e$ of $c$, for investment duration $d$ under initial
    capital distribution $\mc{D}$, as follows:
    \begin{align}
      e_{c, d, \mc{D}} = -\msf{Var}_{v \gets \mc{D}}[f_{c, d}(v)]
    \end{align}
\end{definition}

The intuition behind this definition is that, to have egalitarianism, the ROI
must remain the same across different capital investments. As such, any
deviation from the mean is non-egalitarian. Naturally, if a system's
egalitarianism is \emph{higher} than another, we say that the former is
\emph{more egalitarian} than the latter. Of course, to be accurate, such
comparisons must be made after fixing the parameters $c$ and $d$, as well as
the initial capital distribution $\mc{D}$. Fixing $\mathcal{D}$ to be the
uniform distribution between a minimum and a maximum capital, the returns are
the same for all initial capitals alike.

Based on the above, we can define the \emph{ideal egalitarian curve}. First,
as an interesting thought experiment, we consider the egalitarian curve which
is decreasing (and is, arguably, \emph{the} ideal curve). In this case, small
investors would receive proportionally more newly created cryptocurrencies for
every dollar they invest, \ie the system would redistribute wealth from the
rich to the poor. However, one can quickly see that, in decentralized
cryptocurrencies where the identities of the participants are unknown, this is
impossible. Indeed, the fact that decentralized cryptocurrencies allow
anonymous generation of new identities~\cite{douceur2002sybil} allows a wealthy
investor to split their capital into smaller ones. Thus, if the curve were ever
to have a negative slope, the sum of the smaller splits of the rich investment
would achieve a higher gain. By the definition of the curve, which mandates
that it depicts the ROI of an \emph{optimal} investment, this would be a
contradiction. Corollary~\ref{cor:sybil} makes this intuition more precise.

\begin{corollary}[Sybil strategies]\label{cor:sybil}
    Fix a cryptocurrency $c$ and an investment period interval $d$. Given
    capital $v$, for every natural number $i \in \mathbb{N}^\star$, it holds
    that $f_{c,d}(v) \leq f_{c, d}(i \cdot v)$.
\end{corollary}
\begin{proof}
    We prove the statement by contradiction. Assume that, for some capital $v$,
    there exists a natural number $i \in \mathbb{N}^\star$ such that
    $f_{c,d}(v) > f_{c,d}(i \cdot v)$. Also assume that, for capital $v$, the
    optimal strategy is $B'$, so $\underset{B \in
    \mathbb{B}}{\max}{\mathbb{E}[B(v)]} = \mathbb{E}[B'(v)]$. For capital $i
    \cdot v$, there exists a strategy $B''$, such that the capital is split
    into $i$ equally-sized parts, with the strategy $B'$ applied on each part.
    Given that the execution of each such sub-strategy is independent, the
    expected returns for $B''$ are:
    \begin{align}\label{eq:break-strategy}
        \mathbb{E}[B''(i \cdot v)] = i \cdot \mathbb{E}[B'(v)]  = i \cdot \underset{B \in \mathbb{B}}{\max}{\mathbb{E}[B(v)]}
    \end{align}
    Additionally, $B''$ is at best the optimal strategy, so:
    \begin{align}\label{eq:multi-strategy}
        \underset{B \in \mathbb{B}}{\max}{\mathbb{E}[B(i \cdot v)]} \geq \mathbb{E}[B''(i \cdot v)] \xRightarrow{\text{(\ref{eq:break-strategy})}}
        \underset{B \in \mathbb{B}}{\max}{\mathbb{E}[B(i \cdot v)]} \geq i \cdot \underset{B \in \mathbb{B}}{\max}{\mathbb{E}[B(v)]}
    \end{align}
    However, it should also hold that:
    \begin{alignat}{2}
        f_{c,d}(v) &> f_{c,d}(i \cdot v) \Rightarrow \notag\\
        \frac{\underset{B \in \mathbb{B}}{\max}{\mathbb{E}[B(v)]} - v}{v} &> \frac{\underset{B \in \mathbb{B}}{\max}{\mathbb{E}[B(i \cdot v)]} - i \cdot v}{i \cdot v} \xRightarrow{\text{(\ref{eq:multi-strategy})}} \notag\\
        \frac{\underset{B \in \mathbb{B}}{\max}{\mathbb{E}[B(v)]} - v}{v} &> \frac{i \cdot \underset{B \in \mathbb{B}}{\max}{\mathbb{E}[B(v)]} - i \cdot v}{i \cdot v} \Rightarrow \notag\\
        \frac{\underset{B \in \mathbb{B}}{\max}{\mathbb{E}[B(v)]} - v}{v} &> \frac{\underset{B \in \mathbb{B}}{\max}{\mathbb{E}[B(v)]} - v}{v} \notag
    \end{alignat}
    which is impossible.
\end{proof}

Corollary~\ref{cor:sybil} shows that, in purely decentralized systems, a
decreasing egalitarian curve is impossible. Therefore, the next-best ideal
curve is a constant one, where the ROI is stable regardless of capital
invested. Under this condition, the amount of freshly generated cryptocurrency
is exactly proportional to the money invested. Consequently, a cryptocurrency
with an ideal egalitarian curve is perfectly egalitarian
(Definition~\ref{def:perfect-egalitarianism}).

\begin{definition}[Perfect egalitarianism]\label{def:perfect-egalitarianism}
    A cryptocurrency $c$ is \emph{perfectly egalitarian}, for investment
    duration $d$ and initial capital distribution $\mc{D}$, if $e_{c, d, \mc{D}}
    = 0$.
\end{definition}

\subsection{Discussion}

In this chapter, in providing a concrete definition of crypto-egalitarianism,
we enable an evidence-based discussion to substitute folklore arguments.
The first application of this metric was provided
in~\cite{karakostas2019cryptocurrency}, which compared some of the largest
cryptocurrency systems to date.  Using our model, the egalitarianism of four
indicative PoW-based cryptocurrencies (Bitcoin,
Litecoin~\cite{lee2011litecoin}, Ethereum~\cite{buterin,wood2014ethereum}, and
Monero~\cite{van2013cryptonote}) was measured.  The assessed claims of these
projects were found in agreement with our data, thus presenting for the first
time economic comparisons which quantify them precisely. On the pure PoS side,
it was shown that egalitarian behavior is similar across all coins,
independently of externalities such as hardware characteristics. Therefore, it
suffices to perform a case study of an indicative PoS protocol (in this case,
Ouroboros~\cite{C:KRDO17}).  It was then shown that pure PoS coins can be
perfectly egalitarian, contrary to their PoW counterparts.
These results were very optimistic in terms of usability of our metric, as they
provide concrete figures which measure the egalitarianism of several popular
cryptocurrencies. The most unexpected result arised from the comparison between
the PoW and PoS mechanisms. Although blockchain folklore argued in favour of
PoW systems in terms of egalitarianism, these results show that, in fact, it is
PoS systems which are more egalitarian under our proposed model.

Another interesting property that arises from this work is the impossibility of
wealth redistribution in cryptocurrency systems. As shown in
Corollary~\ref{cor:sybil}, in a purely decentralized setting, where no
real-world identity checks exist (and, arguably, cannot exist), a wealthy
participant can always pose as multiple poor users. Therefore, any attempt to
redistribute wealth from the rich to the poor, based on entirely technological
tools and without taking into account real-world social structures, seems
doomed to fail. In conclusion, all decentralized cryptocurrencies are ``rich
get richer'' schemes; the pertinent question, which this chapter aimed at
resolving, is \emph{how fast} this takes place.


\section{
    Tax Applications of Programmable Money
}\label{sec:taxation}

With the advent of Bitcoin~\cite{nakamoto2008bitcoin} the economic aspects of
consensus protocols came to the forefront. While classical literature in
consensus primarily dealt with ``error models'', such as fail-stop or Byzantine
\cite{DBLP:journals/jacm/PeaseSL80}, the pressing question post-Bitcoin is
whether the incentives of the participants align with what the consensus
protocol asks them to do.

Motivated by this, existing literature pursued various research paths.
One line of work investigated
whether the Bitcoin protocol is an equilibrium under certain conditions
\cite{KrollDaveyFeltenWEIS2013,kiayias16EC}. Another, pinpointed protocol
deviations that can be more profitable for some players, assuming others follow
the protocol~\cite{FC:EyaSir14,FC:SapSomZoh16,FCW:JLGVM14,CCS:CKWN16}.
Some works proposed tweaks towards improving the underlying blockchain protocol
in various settings~\cite{FC:FKORVW19,koutsoupias19www}, game-theoretic studies
of pooling behavior~\cite{lewenberg15,CCS:CKWN16,ITCS:ArnWei19}, as well as
equilibria involving abstaining from the protocol~\cite{DBLP:conf/ec/FiatKKP19}
in high cost scenaria. Going beyond consensus, economic mechanisms have also
been considered in the context of multi-party
computation~\cite{CCS:KumMorBen15,FC:DavDowLar19,FC:DavDowLar18}, to
disincentivize ``cheating''.
Finally, a large body of research was dedicated to
optimizing particular attacks; respresentative works
\begin{inparaenum}[i)]
    \item identify optimal selfish mining strategies~\cite{FC:SapSomZoh16};
    \item propose a framework~\cite{CCS:GKWGRC16} for quantitatively
    evaluating blockchain parameters and identifies optimal strategies for
    selfish mining and double-spending, taking into account network delays;
    \item propose alternative strategies~\cite{EPRINT:NKMS15}, that are more
        profitable than selfish mining.
\end{inparaenum}

Though the above works provide some glimpses on how Bitcoin and related
protocols behave from a game-theoretic perspective, they still offer very
little guidance on how to design new consensus protocols. This is a problem of
high importance, given the current negative light shed on Bitcoin's perceived
energy inefficiency and carbon footprint~\cite{martin2021energy} that
asks for alternative protocols. Proof-of-Stake (PoS) ledgers is currently the
most prominent alternative to Bitcoin's Proof-of-Work (PoW) mechanism. While
PoW requires computational effort to produce valid messages, \ie blocks which
are acceptable by the protocol, PoS relies on each party's stake, \ie the
assets they own, and each block is created at (virtually) no cost.
Interestingly, while it is proven that PoS protocols are Byzantine resilient
\cite{C:KRDO17,EPRINT:CGMV18,EPRINT:GHMVZ17,buterin2017casper} and are even
equilibriums under certain conditions \cite{C:KRDO17}, their security is
heavily contested by proponents of PoW protocols via an economic argument. In
particular, the argument termed the \emph{nothing-at-stake}
attack~\cite{li2017securing,ethereumFaq,nothing-at-stake-1} asserts that
parties who maintain a PoS ledger will opt to produce conflicting blocks,
whenever possible, to maximize their expected rewards.

What merit do these criticisms have?  Participating in a blockchain protocol is
a voluntary action that involves a participant downloading the software,
committing some resources, and running the software. Furthermore, especially
given the open source nature of these protocols, nothing prevents the
participant from modifying the behaviour of the software in some way and engage
with the other parties following a modified strategy. There are a number of
adjustments that a participant can do which are undesirable, \eg
\begin{inparaenum}[i)]
    \item run the protocol intermittently instead of continuously;
    \item not extend the most recent ledger of transactions they are aware of;
    \item extend simultaneously more than one ledger of transactions.
\end{inparaenum}
One can consider the above as fundamental {\em infractions} to the protocol
rules and they may have serious security implications, both in terms of the
consistency and the liveness of the underlying ledger.

To address these issues, many blockchain systems introduce additional
mechanisms on top of Bitcoin incentives, frequently with only rudimentary game
theoretic analysis. These include:
\begin{inparaenum}[i)]
    \item rewards for ``uncle blocks'' in Ethereum;
    \item stake delegation~\cite{eosWhitepaper}  in Eos and Polkadot, where
    users assign their participation rights to delegates, as well as stake
    pools in Cardano \cite{SCN:KarKiaLar20};
    \item penalties~\cite{buterin2017casper,casper-incentives} for misbehavior
    in Ethereum 2.0, that enforce forfeiture of large deposits (referred to as
    \emph{``slashing''}) if a party misbehaves, in the sense of being offline
    or using their cryptographic keys improperly.
\end{inparaenum}
The lack of thorough analysis of these mechanisms is of course a serious
impediment to the wider adoption of these systems.

For instance, forfeiting funds may happen inadvertently, due to server
misconfiguration or software and hardware bugs~\cite{khatri2021slashed}.
A party that employs a redundant configuration with multiple replicas, to
increase its crash-fault tolerance, may produce conflicting blocks if, due to a
faulty configuration or failover mechanism, two replicas come alive
simultaneously.  Similarly, if a party employs no failover mechanism and
experiences network connectivity issues, it may fail to participate. Finally,
software or hardware bugs can always compromise an -- otherwise safe and secure
-- configuration.  This highlights the flip side of such penalty mechanisms:
participants may choose to not engage, (\eg to avoid the risk of forfeiting
funds, or because they do not own sufficient funds to make a deposit), or, if
they do engage, they may steer clear of fault-tolerant sysadmin practices, such
as employing a failover replica, which could pose quality of service concerns
and hurt the system in the long run.

The above considerations put forth the fundamental question that motivates our
work: {\em How effective are blockchain protocol designs in disincentivizing
particular infractions?} In more detail, the question we ask is whether selfish
behavior can lead to deviations, starting with a given blockchain protocol as
the initial point of reference of honest --- compliant --- behavior.

\paragraph{Related Work}
In cryptographic literature, pools are mostly treated from an engineering
perspective. In PoW systems, SmartPool~\cite{USENIX:LVTS17} is a notable design
of a distributed mining pool for Ethereum, which, similar to our work, utilizes
smart contracts for reward distribution.  On the PoS domain,
Ouroboros~\cite{C:KRDO17} offers a brief description of how delegation can be
used within the protocol. This idea is expanded in~\cite{SCN:KarKiaLar20},
which provides a formal definition of PoS wallets and includes stake pool
formation method via certificates. However, the pool's management is again
centralized around the operator; our work extends this line of work by enabling
the formation of a collective pool. Another work, orthogonal to ours, by
Br{\"{u}}njes \etal~\cite{DBLP:conf/eurosp/BrunjesKKS20} considers the
incentives of distributing rewards among stake pools and aims to incentivize
the creation of a (pre-defined) number of pools. However, it assumes that the
pool operator commits part of their stake to make the pool more appealing, thus
favoring larger pool operators. Our work eases such wealth concentration
tendencies by enabling a collective pool to be equally competitive to a
centralized one.

\section{Desiderata}\label{sec:collective-pool-desiderata}

Our design assumes a group of stakeholders who jointly create a stake pool
without a single operator. Since large stakeholders typically form pools on
their own, our protocol concerns smaller stakeholders, who could otherwise not
participate directly. Therefore, our design could \eg be appealing to a group
of friends or colleagues, who aim for a more steady reward ratio without
relying on a third party. Importantly, it should operate in a trustless environment as,
unfortunately, even in these scenarios, trust is not
a given. Notably, our targeted audience is parties who wish to
actively participate, \ie always be online to perform the required consensus
actions; parties who wish to remain offline may instead opt for delegation
schemes~\cite{eosWhitepaper,SCN:KarKiaLar20}.

In the absence of a central party, the responsibility of running the pool is
shared among all pool's members, requiring some level of coordination which may
be cumbersome. For instance, if the protocol requires unanimous actions, a
single member could halt the pool's operation. To ensure good performance, the
pool should allow a subset (of a carefully chosen size) to act on behalf of the
whole group. The choice of such subsets depends on each party's ``weight'',
which is in proportion to their stake.  In summary, we have the following
initial assumptions, which form the basis for outlining our work's desiderata:

\begin{itemize}[noitemsep]
    \item \emph{small number of parties}: a collective pool is operated by a
        small group of players;

    \item \emph{small stake disparity}: the profiles of the collective pool's
        members are similar, \ie they contribute a similar amount of stake to
        the pool;

    \item \emph{stake proportion as ``weight''}: each party is assigned a
        weight for participating in the pool's actions, relative to their part
        of the pool's total stake.
\end{itemize}

Next, we provide an exhaustive list of basic requirements of a collective stake
pool. We note that an \emph{admissible party set} is a set of parties with
enough stake, \ie above a threshold of the total pool's stake which is agreed
upon during the pool's initialization. To the extent that some desiderata are
conflicting, our design will aim to satisfy as many requirements as possible:
\begin{itemize}[noitemsep]
    \item \emph{Proportional Rewards}: the claim of each member on the
        entire pool's protocol rewards should be proportional to
        their individual contribution.

    \item \emph{Joint Control of Rewards}: the members of a pool should
        jointly control the access to its funds.

    \item \emph{Unilateral Reward Withdrawal}: at any point in time, a
        stakeholder should be able to claim their reward,
        accumulated up to that point, without necessarily interacting with
        other members of the pool.

    \item \emph{Permissioned Access}: new users can join the pool following
        agreement by an admissible set of pool members.

    \item \emph{Robustness against Aborting}: the pool should not fail to
        participate in consensus, unless an admissible set of members aborts or
        is corrupted.

    \item \emph{Public Verifiability}: stake pool formation and operation
        should be publicly verifiable (\st consensus could take into account
        the aggregate pool's stake).

    \item \emph{Stake Reallocation}: users should freely change their
        personal stake allocated to the pool, without interacting with other
        members of the pool.

    \item \emph{Parameter Updates}: an admissible set of parties should be
        able to update the stake pool's parameters.

    \item \emph{Force Removal}: an admissible set of parties should be able
        to remove a member from the pool.

    \item \emph{Pool Closing}: an admissible set of parties should be able to
        permanently close the stake pool.

    \item \emph{Prevention of Double Stake Allocation}: a party should not
        simultaneously commit the same stake to two different stake pools.
\end{itemize}

\input{content/taxation/taxchain}
\input{content/taxation/provisions_extension}

