\begin{laysummary}
    Distributed ledgers have been hailed as the ``next big thing'' for more
    than a decade. Bitcoin, which established the blockchain paradigm, paved
    the way for a technology that touches on a multitude of interdisciplinary
    domains. However, many subsequent blockchain systems have inherited various
    deficiencies of Bitcoin, most importantly in terms of energy consumption,
    the number of computations and storage needed to maintain the ledger, and
    the incentives offered to the ledger's maintainers. This thesis uses tools
    of cryptography, computer security, distributed systems, and economics to
    explore how users can participate in the maintenance of digital assets via
    distributed ledgers in a secure, efficient, and sustainable manner.
\end{laysummary}
