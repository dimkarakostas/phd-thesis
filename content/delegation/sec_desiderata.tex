\section{General Desiderata}\label{sec:desiderata-delegation}

Before presenting our framework, we first identify the properties that the
wallet in a PoS setting should offer. This investigation is an important step
in understanding the restrictions in designing such systems, as well as
evaluating the choices that a PoS protocol's designer should make. We organize
the desiderata in three basic categories, based on the related system
component.

\paragraph{Addresses and Attributes}\label{subsec:address-desiderata}
In a PoS system, each account manages a set of addresses. These addresses own a
non-negative amount of cryptocurrency assets and may contain various account
metadata. Any PoS system should offer at minimum two basic operations for each
user's account:
\begin{inparaenum}[i)]
    \item \emph{paying} and
    \item \emph{staking}.
\end{inparaenum}
Addresses, simply put, are strings which have cryptocurrency balances
associated with them. They may also contain metadata, in the form of arbitrary
attributes, which are useful for various system operations. We identify the
following desiderata for addresses in a PoS setting:
\begin{itemize}
    \item \emph{Address Non-malleability}: assuming access to an address and
        the ledger, an adversary should not be able to construct a different
        address that shares \emph{only some} of the address's attributes;
    \item \emph{Address Uniqueness}: any two addresses with different
        attributes should be distinct;
    \item \emph{Short Addresses}: addresses should be relatively short (in
        order to be usable and storage efficient);
    \item \emph{Multiple Types of Addresses}: construction of more than one
        type of addresses should be allowed, with each type supporting a
        different subset of basic operations (\eg to ban some addresses from
        participating in consensus);
    \item \emph{Multiple Device Support}: an account should be able to exist on
        multiple devices that share \emph{no joint internal state};
    \item \emph{Address Recovery}: an account should be able to identify its
        addresses, assuming access to the ledger and the payment keys which it
        controls;
    \item \emph{Privacy and unlinkability:} addresses should be
        indistinguishable from one another and not publicly-linkable to their
        account.
\end{itemize}

\paragraph{Basic Account Operations}
The two basic types of operations, \ie \emph{payment} and \emph{staking}, can
be performed independently by two separate pieces of information, denoted by
$\accPay$ and $\accSta$. The main advantage of this approach is that $\accPay$ is
reserved only for transferring funds, while staking operations require access
only to $\accSta$.  Another desirable result is the ability to recover all
addresses given a master key, \eg as implemented by HD
Wallets~\cite{CCS:DasFauLos19,FC:GutSte15,bip32}; this feature is particularly
important in case the equipment which hosts the wallet is lost. We summarize
the above, with some additions, as follows:
\begin{itemize}
    \item \emph{Account Master Key:} there should exist a ``master'' piece of
        information, \eg a master seed, that can be used to generate all of the
        account's management information, \ie its keys;
    \item \emph{Staking and Payment Separation:} compromising the staking
        operations should not affect the payment operations (and vice-versa);
    \item \emph{Payment Key Information Safety:} apart from a cryptographic
        hash, no other information about the payment key $\accPay$ should be
        public prior to issuing a payment;
    \item \emph{Key Exposure Mitigation:} ownership of the account's assets and
        staking ability should be recoverable, in case the staking information
        $\accSta$ is compromised.
\end{itemize}

\paragraph{Delegation Mechanism and Stake Pools}
Delegation depends on the ability of a user to give the rights over her stake
to another user. This action should be distinguishable from other actions, like
payment, in order to protect the users and facilitate automatic reward schemes.
Therefore, the delegation desiderata are as follows:
\begin{itemize}
    \item \emph{Cost Effectiveness:} stake delegation, re-delegation, or pool
        formation should be cost effective;
    \item \emph{Chain Delegation Restriction:} a limit to the number of
        re-delegations of the same stake unit should be enforceable;
    \item \emph{Delegation Verification:} participants in the system should be
        able to verify the status of active delegations.
\end{itemize}
