\section{Discussion}\label{sec:future}

Our framework offers a range of choices for embedding information in PoS
addresses, as well as enabling multiple address types.  The former allows the
addition of \emph{metadata} into the addresses, including keys for staking and
spending and \emph{device and account identification tags}. The latter gives
the ability to various players, \eg enterprises such as exchanges, to operate
using well-crafted, special-purpose addresses that are fit for special needs.
These features embrace a wide range of the desiderata outlined in
Section~\ref{sec:desiderata-delegation}. \emph{Address Non-Malleability} is addressed
during the address generation phase, described in detail in
Section~\ref{subsec:malleability_predicate}. Moreover, the two types of keys,
for payment and staking, cover the need for \emph{Staking and Spending
Separation}, whereas \emph{Key Exposure Mitigation} is achieved by providing
the flexibility to issue new delegation certificates using the ``staking
action'' interface, in case the staking key of the delegate is compromised.
\emph{Address Uniqueness} is addressed by the checks that the functionality
performs upon receiving a possible address from the adversary. Additionally,
the flexibility on defining attributes allows for \emph{Multiple Devices
Support} and \emph{Address Recovery}, by constructing special tags that are
embedded in the address. Furthermore, \emph{Delegation Verification} is
possible by obtaining the delegation certificates that pertain to its staking
key. Another advantage of this design is the \emph{Cost Effectiveness} of the
delegation mechanism, assigning and changing a delegate at the cost of only one
transaction. The delegation mechanism also allows a party to prove that it has
the right to append the ledger, although possibly restricting \emph{Chain
Delegation}. Finally, our framework enables a smooth \emph{bootstrapping}
delegation process, by allowing an initial delegation assignment phase which
depends on the implementation details of the ledger.

Notably, our desiderata revolve primarily around key management and address
generation. Therefore, we don't capture network reliability or availability
requirements, but rather assume an abstract ledger model, which satisfies
persistence and liveness in a synchronous setting. This allows us to focus on
address and signature generation, while treating the other components of the
system as black boxes, which presumably satisfy the required properties. This
is evident in Section~\ref{sec:delegation-transaction}, which assumes a generic
ledger model, abstracted as a set of variables and algorithms. Nonetheless, a
more rigorous analysis on the incorporation of our wallet core in a complete
wallet is an important next step. Such treatment could formally capture
security on all layers, \ie key management, consensus, network, \etc A possible
path towards this goal could propose a variant of $\FuncW$ with key-evolving
signatures, which would replace $\mc{F}_{\mathsf{KES}}$ in a protocol like
Ouroboros Praos~\cite{EC:DGKR18}, followed by a rigorous proof of security of
this consensus protocol variant.
