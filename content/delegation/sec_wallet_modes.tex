\subsection{Modes of Execution}\label{sec:wallet-modes}

Our final contribution is a set of modes of operation of a PoS wallet.

\paragraph{Regular}
This being most straightforward wallet deployment method, a regular wallet is
bootstrapped with a \emph{base} address $\addr_0$ and its stake is managed by a
key $\stakingkeypair$. After $\addr_0$ receives its first assets, the wallet
may perform staking actions by using $\stakingkeypair$. To perform staking on
its own, the wallet publishes a delegation certificate $\Sigma$ that delegates
to its own key $\stakingkeypair$. Subsequent addresses are pointer addresses to
$\Sigma$, so eventually all addresses are managed by the same staking key.
When the user wishes to delegate to a staking pool, identified by the key
$\stakingkeyverify_P$, the wallet simply publishes a certificate $\Sigma_d$
delegating from $\stakingkeyverify$ to $\stakingkeyverify_P$.

\paragraph{Offline with Cold Staking}
This wallet lives in an offline device, \eg on paper. It is rarely accessed for
payments, but regularly performs staking actions. It is bootstrapped similarly
to the regular wallet and, since its payment keys are stored offline, the
staking keys are managed as follows:
\begin{itemize}
    \item \emph{basic security}: the staking key $\stakingkeypair$, which
        manages all addresses, is online; in case $\stakingkeysign$ is
        compromised, the user accesses the payment keys and sends the funds to
        new addresses, controlled by a new staking key;
    \item \emph{enhanced security}: the wallet creates a certificate
        $\Sigma^\prime$, which delegates from $\stakingkeyverify$ to the
        ``hot'' key $\stakingkeyverify_h$; after $\Sigma^\prime$ is published,
        the user stores $\stakingkeypair$ offline and $(\stakingkeyverify_h,
        \stakingkeysign_h)$ online; in case $(\stakingkeyverify_h,
        \stakingkeysign_h)$ is compromised, $\stakingkeypair$ is used to
        delegate to a new ``hot'' key, without requiring access to the wallet's
        payment keys.
\end{itemize}

\paragraph{Enhanced Unlinkability of Addresses}
This wallet aims at a higher level of privacy, with each address managed by a
single staking key. Therefore, to change its delegation profile, the wallet
creates a certificate for each address. Additionally, it achieves different
security guarantees as follows:
\begin{itemize}
    \item \emph{online:} the wallet is online and creates only pointer
        addresses, which point to the stake pool's registration certificate; to
        re-delegate, it creates new pointer addresses and moves the funds from
        the old to the new addresses;
    \item \emph{offline:} the payment keys are stored offline, so the wallet
        creates base addresses, each managed by a unique staking key, and keeps
        the staking keys online; to re-delegate, it publishes a new certificate
        for each address.
\end{itemize}

\paragraph{The Stake Pool's Wallet}
A stake pool's wallet performs only staking actions, so its key
$(\stakingkeyverify_P, \stakingkeysign_P)$ is managed as follows:
\begin{itemize}
    \item \emph{basic security}: the key pair $(\stakingkeyverify_P, \stakingkeysign_P)$
        is stored online and is used directly for staking; in case of
        compromise, the wallet creates a new staking key while, for practical
        purposes, an alert mechanism should exist to notify the users to
        re-delegate their stake to the new key;
    \item \emph{enhanced security}: the wallet creates a lightweight
        certificate $\Sigma_l$, which delegates to a ``hot'' key
        $\stakingkeyverify_{Ph}$, and then stores $(\stakingkeyverify_P,
        \stakingkeysign_P)$ offline, while using $(\stakingkeyverify_{Ph}, \stakingkeysign_{Ph})$ and
        $\Sigma_l$ for staking; if $(\stakingkeyverify_{Ph}, \stakingkeysign_{Ph})$ is compromised, the
        wallet creates a new hot key $(\stakingkeyverify_{Ph}^\prime, \stakingkeysign_{Ph}^\prime)$ and a new
        lightweight certificate $\Sigma_l^\prime$, which delegates from
        $(\stakingkeyverify_{P}, \stakingkeysign_{P})$ to $\stakingkeyverify_{Ph}^\prime$ and includes a
        higher counter compared to $\Sigma_l$, using them for staking instead.
\end{itemize}
We note that, in particular, the stake pool's wallet is further explored in the
upcoming chapter on collective stake pools.
