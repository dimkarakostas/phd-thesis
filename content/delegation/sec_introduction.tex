Blockchain protocols based on the Proof-of-Stake (PoS) paradigm depend --- by
nature --- on the active participation of stakeholders.  Specifically, in PoS,
the set of potentially eligible parties is comprised of ``stakeholders'', \ie
parties who own some amount of the digital assets which are maintained by the
ledger.  Importantly, this set is open, \ie parties can arbitrarily join and
leave, while also remaining pseudonymous.  Therefore, the eligible party,
dubbed the ``minter'', is selected proportionally to its stake, \ie the amount
of assets that it owns; the details of this selection process have resulted in
a number of PoS flavors and protocols.

However, the inherent {\em duality} of PoS digital assets, which can both be
transferred between entities and used in the protocol's execution, diverges
from the Proof-of-Work (PoW) setting and might result in users abstaining from
engaging in the PoS mechanism.  The stakeholders are expected to consistently
follow the protocol's execution, by checking whether they are eligible to
participate, and, in those cases and within a specific time frame, engage in
transaction processing per the PoS protocol's rules~\cite{C:KRDO17,EPRINT:CGMV18,EPRINT:GHMVZ17,vasin2014blackcoin}. This
feature is in sharp contrast with PoW protocols, which offer a natural
decoupling between the consensus layer participants who generate blocks, \ie
the \emph{miners}, and the users of the system who issue transactions.
Naturally, the set of users subsumes the miners, since \eg the miners collect
fees and may also transact using them, and a substantial number of users do not
participate in the consensus protocol.

This dual nature of PoS assets raises two major considerations. First, some
secret, key-related information is frequently used on behalf of an asset, thus
potentially revealing critical information that weakens the asset's security or
increase the attack surface against a user's wallet. For instance, in the UTxO
model (implemented by Bitcoin), the public key which controls the assets is
published only upon spending them; however, PoS systems cannot adapt this model
directly, since the public key is revealed when participating in the protocol,
\ie (typically) before spending the funds. Therefore, using the same key for
all operations both increases the attack surface against it and introduces
quantum attacks, given that most implementations employ non-post-quantum secure
signature schemes.  Second, a computational and availability requirement is
imposed on the stakeholders; therefore, in an environment where the majority of
users are often offline and abstain from the protocol's execution, the security
guarantees of the ledger are weakened.

The above issues are well-known and have already been informally considered by
the blockchain community. For instance, some schemes propose a separation
between a staking and a payment key to address the first
consideration\footnote{One such discussion is available at:
\url{https://www.reddit.com/r/ethereum/comments/6idf2c}}.  The second
consideration, although seemingly unavoidable given the nature of PoS
protocols, can be countermeasured by enabling the delegation of the PoS
protocol participation rights, \ie block generation and transaction validation.
This mechanism would allow users to organize in ``stake pools'', \ie
consortiums managed by a single user, the pool ``leader'', who participates in
the PoS protocol on behalf of all members. Stake pools bring efficiency
advantages, since the set of stake pool leaders is (typically) smaller than the
entire stakeholders' set. Given that some PoS protocols~\cite{C:KRDO17} are
based on multi-party computation, introducing a \emph{committee} of pool
leaders substantially reduces the computational and communication complexity
overhead.

Interestingly, PoS implementations don't typically address these concerns
comprehensively. On the one hand, ``delegated PoS'', \eg in Steem~\cite{steem}
and EOS~\cite{eosWhitepaper}, attempts to address the second consideration by
enabling the voting of delegates. However, both systems use a single key for
payment and voting, thus failing to address the first consideration.  On the
other hand, systems like NEO~\cite{neo-consensus}, as of September 2020, enabled
only $7$ consensus nodes, $5$ of which are controlled by a single entity,
whereas the only way of participating in the consensus mechanism is via central
approval. Finally, Decred~\cite{decred} uses a ticketing system, \ie
stakeholders buy a ticket to participate in the protocol, which is akin to
using a separate key; however, it requires the locking of funds while staking,
\ie it does not allow concurrent payments and staking, a major blow to the
system's usability.

Evidently, even though practical solutions do exist, a comprehensive and formal
treatment of a PoS system's account management is less well-researched. In
fact, due to the very little systematization of the PoS wallets' security and
the lack of concise guidelines, developers often resolve to ad hoc solutions.
Given that PoS systems are increasingly gaining momentum, it is imperative that
this problem receives a thorough and formal treatment. The results of this
chapter fill this gap by acting as a guideline for system architects and
developers, aiming to better wallet implementations in terms of safety, as well
as enable robust designs with improved performance.  A further important
motivation behind our work is the current low level of decentralization in PoS
systems. As illustrated above, some projects are yet to allow stakeholders to
practice staking, opting instead for a closed set of block producing nodes.
Even in cases where stakeholders are arguably in control of their stake, they
may choose their stake representatives from a very narrow set of accounts; for
instance, the EOS block producing nodes are only $21$ at any given time. Our
work aims at alleviating such centralization tendencies by enabling every user
to either participate on their own or assign their stake to any delegate of
their choice.  Consequently, the formalization of the PoS wallet is a stepping
stone for future research, given that the wallet is the gateway through which
users interact with the system, and a core element of the consensus protocol
itself.  As a result, the composable nature of our framework allows future
research to employ it without composability concerns about its underlying
implementation.
