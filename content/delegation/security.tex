\section{Security Analysis}

The security of $\ProtoW$ is given \wrt $\FuncW^{M}$, the signature scheme's
Existential Unforgeability under Adaptive Chosen Message Attacks (\eufcma)
property, and a number of properties of our custom algorithms. Therefore,
before presenting the analysis for $\FuncW^{M}$ parameterized with the
predicate $M_{\mathsf{SM}}$, we introduce the properties of the address and
metadata generation algorithms.

\subsection{Properties of  the Generation Algorithms}\label{subsec:addrgen_properties}

Briefly, $\mathsf{GenAddr}$ is the address generation algorithm which, given an
attribute list, returns an address. This algorithm is implemented in
Section~\ref{sec:address}, although $\ProtoW$ can be parameterized with any
address generation scheme which offers the necessary properties.

Specifically, address generation should be collision resistant, \ie analogously
to hash functions, it should be infeasible for an adversary to produce the same
address for different attribute lists, cf.
Definition~\ref{def:address_collision}. Also address generation should be
attribute non-malleable, \ie it should be infeasible for an adversary to
generate valid addresses for a payment key without access to the entire
attribute list (including the private attributes), cf.
Definition~\ref{def:address_non_malleability}. In addition, we assume that
$\algohierarchicalkeygen(\cdot)$ is hierarchical, \ie the distribution of
produced keys is indistinguishable from that of $\algokeygen$, cf.
Definition~\ref{def:hierarchical_keygen_ind}, and $\algorecoverytaggen(\cdot)$
is collision resistant, thus ensuring that the recovery tags are both unique
(with overwhelming probability) and deterministically computable.

\paragraph{Address and Attribute Generation Properties}
In order to prove the security of our protocol, the address generation
algorithm $\mathsf{GenAddr}: \Delta_1 \times \ldots \times
\Delta_\addrgenlength \rightarrow \addresset$ should offer the following
properties.
\begin{definition}[Address collision resistance]\label{def:address_collision}
    Analogously to hash functions (cf.
    Definition~\ref{def:hash}), $\mathsf{GenAddr}$ is collision
    resistant when it is infeasible to produce two different attribute lists
    $l_i = (\attribute^{i}_{1}, \dots, \attribute^{i}_{g})$ for $i \in \{1, 2\}$,
    \ie they differ in at least one attribute like $\exists j \in [1,
    \addrgenlength]$: $\attribute^{1}_j \neq \attribute^{2}_j$, such that
    $\mathsf{GenAddr}(l_1) = \mathsf{GenAddr}(l_2)$, after running
    $\mathsf{GenAddr}(\cdot)$ a polynomial number of times.
\end{definition}

Similarly, we assume the existence of an algorithm $\mathsf{GenMeta}:\Delta_1
\times \ldots \times \Delta_\addrgenlength \rightarrow \{0,1\}^{p(\secparam)}$
for the generation of the metadata $m$ associated with the address. Before
describing extra properties for $\mathsf{GenAddr}$,  we introduce the following
property for the $\mathsf{GenMeta}$ function.

\paragraph{Metadata extraction resistance}
For a challenge address $\addr$ with attribute list $l$, it is intractable for
the  adversary that is given $\addr$ and the public attributes to generate the
metadata information $m$, such that $\mathsf{GenMeta}(l) = m$, even with
polynomial number of address generation and metadata queries, upon arbitrary
choices of public attributes
$(\attribute^\prime_{i}, \dots, \attribute^\prime_{\addrgenlength})$. We say that
$\mathsf{GenMeta}$ algorithm has metadata extraction resistance with respect to
private attributes $(\attribute_{1}, \dots, \attribute_{i-1})$ from an attribute
list $l=(\attribute_{1}, \dots, \attribute_{g})$, and address generation
algorithm $\mathsf{GenAddr}$, when for every attribute list $l \in \Delta_1
\times \ldots \times \Delta_\addrgenlength$, and a fixed index $i$, such that
the private list of attributes is $d = (\attribute_1, \dots, \attribute_{i-1})$,
it holds that:
\begin{align}
    \Pr\left[
    \begin{tabular}{c}
        $l = (\attribute_{1}, \dots, \attribute_\addrgenlength)$, \\
        $\addr \leftarrow \mathsf{GenAddr}(l)$, \\
        $m^\prime \leftarrow \adv^{\mathsf{GenAddr}^{d}(\cdot)}(\attribute_{i}, \dots, \attribute_\addrgenlength)$ \\
    \end{tabular}
    :
    \mathsf{GenMeta}(l)=m^\prime
    \right]
    \leq \negl(\secparam) \nonumber
\end{align}
where the probabilities are computed on the used randomness, the values of $l$
and every PPT adversary $\adv$.

\begin{definition}[Non-malleable attribute address generation]\label{def:address_non_malleability}
    Let $\attributedistribution$ be a distribution of attribute lists, such
    that $\textsc{dom}(\attributedistribution) = \addrgendomain$, and that
    $\attribute_{1},\dots,\attribute_{i}$, \ie the first attributes of  the
    attribute list $l \leftarrow \textsc{dom}(\attributedistribution)$, relate
    to a property over which we define non-malleability. \emph{Even with access
    to the metadata} of the address, \ie the semi-public attributes, it is
    infeasible for the adversary to generate valid addresses, \ie acceptable
    addresses with the same verification payment key, from $\addr$, without
    accessing the attribute list, including the private attributes.
    Concretely, given $\mathsf{Addrs}$, \ie the list of addresses queried by
    $\adv$ to the oracle $\mathsf{GenAddr}^{d}(\cdot)$, it holds that:
    \begin{align}
        \Pr\left[
        \begin{tabular}{c}
        $l = (\attribute_{1}, \dots, \attribute_\addrgenlength)$, \\
        $\addr \leftarrow \mathsf{GenAddr}(l)$, \\
        $(\addr^\prime, \attribute^\prime_{i}, \dots, \attribute^\prime_\addrgenlength) \leftarrow \adv^{\mathsf{GenAddr}^{d}(\cdot), \mathsf{GenMeta}^{d}(\cdot)}(\attribute_{i}, \dots, \attribute_\addrgenlength)$ \\
        \end{tabular}
         \right. : \nonumber \\
        \left.
        \begin{tabular}{c}
        $(\mathsf{GenAddr}^d(\attribute^\prime_{i}, \dots, \attribute^\prime_\addrgenlength) = \addr^\prime) \wedge$ \\
        $(\addr^\prime\neq\addr) \wedge$ \\
        $(\addr^\prime \not \in \mathsf{Addrs})$ \\
        \end{tabular}
        \right]
        \leq \negl(\secparam) \nonumber
    \end{align}
    for probabilities computed over the randomness  in $\mathsf{GenAddr}$
    algorithm, every PPT adversary $\adversary$ and the values of $l$.
\end{definition}

\paragraph{Hierarchical Key Generation Properties}\label{subsec:hierarchical_keygen_properties}

We now define the hierarchical property of the key generation function
$\algohierarchicalkeygen(\cdot)$, which is used by the core wallet protocol $\ProtoW$.
Intuitively, this allows us to use $\algohierarchicalkeygen$ in order to
securely and deterministically produce keys for indexes, instead of using the
$\algokeygen$ function of $\sigscheme$.

\begin{definition}[Hierarchical Key Generation]\label{def:hierarchical_keygen_ind}
    Assume a signature scheme $\sigscheme$. A key
    generation function $\algohierarchicalkeygen(\cdot)$ is hierarchical for
    $\sigscheme$ if, for all parameters $i$, the distribution of keys produced
    by $\algohierarchicalkeygen(i)$ is computationally
    indistinguishable from the distribution
    of keys produced by $\algokeygen$.
\end{definition}

\subsection{Security in the Sink Malleable Setting}

We now describe the core theorem of this chapter. Here, the ideal
functionality, parameterized with the sink malleability predicate
$M_{\mathsf{SM}}$, is realized by the protocol that employs the sink
malleable address generation function. It also uses tag and hierarchical key
construction functions which present the above necessary properties.

\begin{theorem}\label{thm:sec_sink_malleable}
    Let the generic protocol $\ProtoW$ be parameterized by a signature scheme
    $\sigscheme$
    and the $\algorecoverytaggen$, $\algohierarchicalkeygen$, and
    $\mathsf{GenAddr}$ functions. Then $\ProtoW$ securely realizes the ideal
    functionality $\FuncW^{M_{\mathsf{SM}}}$ if and only if $\sigscheme$ is
    $\eufcma$, $\mathsf{GenAddr}$ is collision resistant and attribute
    non-malleable (cf. Definitions~\ref{def:address_collision}
    and~\ref{def:address_non_malleability}), $\algorecoverytaggen$ is collision
    resistant (cf. Definition~\ref{def:hash}), and $\algohierarchicalkeygen$ is
    hierarchical for $\sigscheme$ (cf.
    Definition~\ref{def:hierarchical_keygen_ind}).
\end{theorem}

\begin{proof}
    The proof is constructed in the UC Framework, therefore it is a simulation
    based proof. As such, we will show that the environment $\env$ cannot
    efficiently distinguish between two executions, the ideal and the real.
    Here, the simulator $\Iadv$ interacts with the ideal functionality
    $\FuncW^{M_{\mathsf{SM}}}$ in the ideal execution, whereas $\adv$ interacts
    with $\ProtoW$ in the real execution.
    We divide the proof in the ``if'' and ``only
    if'' parts. First, the ``if'' part shows that if $\ProtoW$ does securely
    realize the ideal functionality $\FuncW^{M_{\mathsf{SM}}}$, when
    instantiated with a $\eufcma$ signature scheme $\Sigma$, a collision
    resistant and non-malleable address generation scheme $\mathsf{GenAddr}$,
    and suitable $\algorecoverytaggen, \algohierarchicalkeygen$ functions at
    least one of the conditions is violated. The ``only if'' part shows that,
    if either of the functions' properties does not hold, \eg if $\Sigma$ is
    not $\eufcma$ or $\mathsf{GenAddr}$ is either not collision resistant or
    non-malleable, then $\ProtoW$ does not securely realize the functionality
    $\FuncW^{M_{\mathsf{SM}}}$, \ie the environment is able to distinguish
    between the two executions.

    Let us now provide the construction for the simulator $\Iadv$, which will
    be useful in the ``if'' part of the proof.

    \emph{The simulator $\Iadv$.}
    The simulator $\Iadv$ runs internally a copy of the adversary $\adv$, and
    keeps a table $\mathsf{TABLE}$ of tuples $(\cdot,\cdot,\cdot,\cdot)$ of
    respectively addresses, attributes, and staking key pairs. Also it performs
    as follows:

    \begin{itemize}
        \item Any inputs received from the environment $\env$, forward them to
            the internal copy of $\adv$. Moreover, forward any output from
            $\adv$ to $\env$;
        \item \emph{Initialization:} Upon receiving $\msg{Initialise}{}$ from
            the functionality $\FuncW$, compute a dummy master key $\masterprivkey
            \xleftarrow{\$} \{0,1\}^\secparam$ and return
            $\msg{InitialiseOk}{}$;
        \item \emph{Address Generation: } Upon receiving the message
            $\msg{GenerateAddress}{\msf{aux}}$ from $\FuncW$, do similarly to
            protocol $\ProtoW$, that is:
            \begin{itemize}
                \item set $i \leftarrow \indexSet$,
                \item set the key pair $(\paymentkeyverify_c, \paymentkeysign_c) = \algohierarchicalkeygen(\langle \masterprivkey, \msf{payment}, i \rangle)$,
                \item set the tag $\msf{wrt} = \algorecoverytaggen(\langle \masterprivkey, i \rangle)$,
            \end{itemize}
            and do the following:
            \begin{itemize}
                \item if $\msf{aux} = (\msf{base})$ compute
                    $(\stakingkeyverify_c, \stakingkeysign_c) = \algohierarchicalkeygen(\langle \masterprivkey, \msf{staking}, i \rangle)$
                    and set $\beta = \stakingkeyverify_c$;
                \item if $\msf{aux} = (\msf{pointer}, \stakingkeyverify)$, set
                    $\beta = \stakingkeyverify$;
                \item if $\msf{aux} = (\msf{exile})$, set $\beta = \bot$.
            \end{itemize}
            Then compute $\addr = \mathsf{GenAddr}(\langle \msf{aux}, \beta,
            \paymentkeyverify_c, \msf{wrt} \rangle)$ and set its attribute list $\addresslist = \langle
            \stakingkeyverify_c, \msf{wrt}, \msf{aux}, \paymentkeyverify_c,
            \paymentkeysign_c, \stakingkeysign_c \rangle$. Then record the
            tuple $(\addr, \addresslist, \beta, \paymentkeysign_c)$ to
            $\mathsf{TABLE}$. Finally, hand to $\FuncW$ the message
            $\msg{Address}{\addr, \addresslist}$;

        \item \emph{Issue Transaction:} Upon receiving $\msg{Pay}{\assetset,
            \addr_s, \addr_r, m}$ find a record $\addresslist \in \mathsf{TABLE}$ that contains the sender's address $\addr_s$ as
            the first item. Then generate the signature $\sign$ for the
            transaction $\tx$, such that $\tx = (\assetset, \addr_s, \addr_r,
            m)$, using $\mathsf{Sign}$ and the payment key of $\addr_s$, and
            hand the message $\msg{Transaction}{\tx, \sign}$ to the
            functionality $\FuncW$. Note that with the attribute list
            $\addresslist$, $\Iadv$ can properly generate $\sign$. Moreover
            such record is expected to be in $\mathsf{TABLE}$, since the
            functionality allows the issuing of transactions by properly
            generated addresses by checking on the list $L_{P}$ before sending
            to $\Iadv$;

        \item \emph{Verify Transaction:} Upon receiving the message
            $\msg{VerifyPayment}{\tx, \sign}$ from $\FuncW$, find the recorded
            verification key $\paymentkeyverify_c$ for the sender's address
            $\addr_s$ in $\tx = (\assetset, \addr_s, \addr_r, m)$ by looking up
            $\addresslist$ for $\addr_{s}$ in $\mathsf{TABLE}$, and use
            $\mathsf{Verify}$ to retrieve the verification bit $\phi$. Then
            return $\msg{VerifiedPayment}{\tx, \sign, \phi}$ to $\FuncW$;

        \item \emph{Issue Staking:} Similarly to issuing a payment, upon
            receiving $\msg{Stake}{\tx_s}$, such that
            $\tx_s = (\paymentkeyverify, m)$, find the correspondent staking key
            $\paymentkeysign$ and use $\mathsf{Sign}$ to generate the signature
            $\sign$, then hand $\msg{Staked}{\tx_s, \sign}$ to $\FuncW$;

        \item \emph{Verify Staking:} As before, upon receiving the message
            $\msg{VerifyStaking}{\tx_s, \sign}$, find the staking key that
            pertains to $\tx_s$, use $\mathsf{Verify}$ to retrieve the
            verification bit $\phi$, and send the message
            $\msg{VerifiedPayment}{\tx_s, \sign, \phi}$ to $\FuncW$. Similarly to
            \emph{Issue Transaction} interface, note that $\Iadv$ knows
            $\paymentkeysign_c$ via $\mathsf{TABLE}$;

        \item \emph{Party Corruption:} Whenever the adversary $\adv$ corrupts a
            party $P$, $\Iadv$ corrupts it in the ideal process and hands to
            $\adv$ the corresponding entries in $\mathsf{TABLE}$.
    \end{itemize}

    \emph{The ``if'' part.}
    Assume, for the sake of the argument, that the environment $\env$ can
    distinguish between the ideal and the real execution with non-negligible
    probability for any simulator construction, including the earlier $\Iadv$
    construction. In that case, it suffices to show that, if $\ProtoW$ does not
    securely realize  $\FuncW$, and given the $\Iadv$ construction, then either
    of the following holds when a ``bad'' event $E$ occurs: the signature
    scheme is not $\eufcma$, $\mathsf{GenAddr}$ is not collision resistant or
    attribute non-malleable, the function $\algorecoverytaggen$ is not
    collision resistant, or the hierarchical property of
    $\algohierarchicalkeygen$ does not hold.

    \emph{Unforgeability:} We assume that \emph{collision} resistance and
    non-malleability properties of the algorithm $\mathsf{GenAddr}$ hold,
    likewise the collision resistance and hierarchical properties for
    $\algorecoverytaggen$ and $\algohierarchicalkeygen$ respectively, along
    with the signature scheme properties \emph{completeness} and
    \emph{non-repudiation}, but unforgeability does not hold (otherwise the
    theorem completes). Note that, by hypothesis, $\env$ distinguishes between
    the two worlds for any construction of $\Iadv$, including the earlier
    $\Iadv$ construction.

    Now we construct a forger $G$ for the unforgeability game per the $\eufcma$
    definition, which initially receives $(\paymentkeyverify,\paymentkeysign)$
    as a challenge (we focus on the payment keys and interfaces, but we note
    that the case for staking is analogous). $G$ simulates the earlier
    described $\Iadv$ in the interaction with $\env$. It then issues signature
    queries to its game, when requested by its simulation of $\Iadv$ and
    $\FuncW$ for signatures on $\paymentkeyverify$.

    In addition, when receiving verification messages of the form
    $\msg{VerifyPayment}{\tx, \sign}$ for other addresses (possibly from other
    verification keys), $G$ uses its internal simulation, specifically the
    \emph{Transaction Verification} interface, \ie its internally generated
    keys, to properly simulate the execution to $\env$. In particular it checks
    if $(\tx, \sigma)$  has been queried in the security game. If it is not
    listed as queried and $\algoverify(\tx, \sigma, \paymentkeyverify) = 1$,
    then it outputs $(\tx, \sign)$ and wins the game. Otherwise, it continues
    with the simulation.

    Given that the environment $\env$ distinguishes between the two executions
    by hypothesis and all other properties of the functions hold, we are
    guaranteed that if $\ProtoW$ does not securely realize $\FuncW$, then the
    unforgeability property does not hold.

    Note that the earlier unforgeability reasoning is valid for $\sigscheme$
    with key generation relying on $\algokeygen$, however $\ProtoW$ relies on
    $\algohierarchicalkeygen$. Therefore, consider the following argument.

    \emph{$\algohierarchicalkeygen$ is hierarchical for $\sigscheme$ and
    every index $i$ is used only once:} Assume the two following protocols:
        \begin{inparaenum}[i)]
            \item a protocol which is similar to $\ProtoW$, except the key
                generation function $\algokeygen$ of $\sigscheme$ is used
                instead of $\algohierarchicalkeygen$, and
            \item the protocol $\ProtoW$.
        \end{inparaenum}
        Now, it is evident that the first protocol securely realizes the ideal
        functionality (since all other properties needed as per the Theorem
        hold). Therefore, the execution of the first protocol is
        indistinguishable from the execution of the ideal functionality, as
        proved in the earlier reasoning.

        Next, consider a PPT algorithm $D$ who tries to distinguish between the
        executions of the two protocols above. Specifically, assume that there
        exists a ``special'' index $i$, for which the usage of
        $\algohierarchicalkeygen$ in the signature scheme is insecure, \ie a
        forgery can be computed by the adversary. For the sake of argument,
        consider the probability that the scheme breaks for this index $i$ by
        $p$. It is clear that, for this index $i$, $D$ is successful, by
        observing the violation of the properties of the signature scheme with
        $\algohierarchicalkeygen$. Note also that the number of produced keys,
        \ie the number of used indexes in both executions, is bounded by a
        polynomial $P(\secparam)$. Therefore, the overall probability that $D$
        is successful is equal to $\frac{p}{P(\secparam)}$.

        However, by definition, $\algohierarchicalkeygen$ is hierarchical for
        $\sigscheme$. Thus, the executions of the two protocols are
        indistinguishable and, as a result, the probability that $D$ is
        successful, \ie $\frac{p}{P(\secparam)}$, is negligible. Consequently,
        the probability $p$, \ie that the signature scheme which uses
        $\algohierarchicalkeygen$ breaks, is also negligible. Therefore, the
        execution of $\ProtoW$ is indistinguishable from the execution of the
        ideal functionality as well, thereby $\ProtoW$ securely realizes the
        ideal functionality.

    \emph{The ``only if'' part.}
    Here we show that, if a single property does not hold, then the environment
    $\env$ can distinguish between the real and ideal executions with
    non-negligible probability. In other words, there is no simulator
    construction that prevents $\env$ from distinguishing both executions.

    \emph{Success probability of $\env$ under weakened assumptions.}
    We now assume that some property of the functions used by the protocol is
    broken. We will then show that $\ProtoW$ does not securely realize
    $\FuncW$. Specifically, we can create an environment $\env$ and an
    adversary $\adv$ such that, for \emph{any} simulator $\Iadv$, $\env$
    distinguishes between the real execution of $\adv$ with $\ProtoW$ and the
    ideal execution of $\Iadv$ with $\FuncW$.

    Initially, the environment sends $\msg{Initialise}{}$ for some party $\party$.
    For each property required by the theorem, we show that the environment can
    distinguish between the two executions as follows:

    \begin{itemize}
        \item \emph{Completeness:} We assume that $\Sigma$ is not complete.
            $\env$ initializes the wallet for a second party $\party'$ and
            sends two messages $\msg{GenerateAddress}{\msf{aux}}$, for some
            arbitrary auxiliary information $\msf{aux}$, and obtains two
            addresses $\addr_s$ and $\addr_r$ for the parties $\party$ and $\party'$
            respectively. Next, it creates a transaction object $\tx =
            (\assetset, \addr_s, \addr_r, m)$, for arbitrary values of assets
            $\assetset$ and metadata $m$, and obtains a signature $\sign$ for
            the transaction $\tx$ by sending the message $\msg{Pay}{\tx}$.
            Finally, it calls the verification interface by
            sending the message $\msg{VerifyPayment}{\tx, \sigma}$. In the ideal
            execution the output is always $\msg{VerifiedPayment}{\tx, \sigma,
            1}$, whereas in the real execution the probability that the output
            is $\msg{VerifiedPayment}{\tx, \sigma, 0}$ is
            non-negligible. The environment could also succeed in
            distinguishing the executions by accessing the \emph{Staking} and
            \emph{Staking Verification} interfaces, issuing staking acts and
            checking the verification bit similarly as with payment
            transactions.

        \item \emph{Non-repudiation:} We assume that $\Sigma$ does not offer
            non-repudiation. The environment now acts like in the case of
            \emph{completeness}, obtaining a signed transaction $(\tx,
            \sigma)$.  However, it now calls the verification interface twice.
            In the ideal execution, the verification bit of the response will
            both times be equal to $1$, whereas in the real execution the
            probability that the verification bit is $0$ is non-negligible.
            Again the environment could access the staking issuing and
            verification interfaces similarly.

        \item \emph{Unforgeability:} We assume that $\Sigma$ is forgeable, so
            there exists a forger $G$ for $\Sigma$. The environment now runs an
            internal copy of $G$. When $G$ wishes to obtain a signature from
            its oracle for some transaction $\tx$, $\env$ accesses the
            \emph{Issue Transaction} interface by sending the message
            $\msg{Pay}{\tx}$ and obtains a signature, which it forwards to $G$.
            When $G$ outputs a signed transaction $(\tx, \sigma)$, $\env$
            proceeds as follows. If $\tx$ has been previously signed, \ie if
            $\sigma$ has been creating by accessing the \emph{Issue
            Transaction} before, then it halts.  Otherwise, it accesses the
            verification interface by sending the message
            $\msg{VerifyPayment}{\tx, \sigma}$. Now, in the ideal execution the
            verification bit in the response from the verification interface is
            always $0$, whereas in the real world it is $1$ with non-negligible
            probability.

        \item \emph{Collision resistance:} We assume that $\mathsf{GenAddr}$ is
            not collision resistant. The environment obtains two addresses by
            calling the address generation interface twice, \ie sending two
            messages $\msg{GenerateAddress}{\msf{aux}}$ for the same auxiliary
            information $\msf{aux}$. In the ideal execution the attribute lists in
            the address responses will always be different, whereas in the real
            execution the probability that two equal addresses for different
            attribute lists are returned is non-negligible.

       \item \emph{Attribute non-malleable:} We assume that $\mathsf{GenAddr}$
           is not attribute non-malleable. Then the environment $\env$, which
            may retrieve correctly generated addresses by accessing the
            \emph{Address Generation} interface,  can generate a forged address
            $\addr^\ast$. Assume, without loss of generality, that $\addr^\ast$
            has been the receiving address for some past transaction, therefore
            $\addr^\ast$ owns some assets. Assume also that $\env$ issues a
            transaction from $\addr^\ast$ to a legitimate address $\addr_r$,
            \ie created via the address generation interface of $\FuncW$.
            On submitting $\msg{VerifyPayment}{\assetset, \addr^\ast,
            \addr_r, m,\sigma}$, for some assets $\assetset$ and metadata $m$,
            the environment $\env$ will receive a message
            $\msg{VerifiedPayment}{\assetset, \addr^\ast, \addr_r, m, \sigma,
            0}$, since the check of the predicate $M_{\mathsf{SM}}$ within
            $\FuncW^{M_{\mathsf{SM}}}$ outputs $0$. On the other hand, in the
            real world interaction with $\ProtoW$, $\env$
            receives $\msg{VerifiedPayment}{\assetset, \addr^\ast,
            \addr_r, m, \sigma, 1}$. Therefore, $\env$ is able to distinguish
            between the executions.

        \item \emph{$\algorecoverytaggen$ collision resistance and every index
            $i$ is used only once:} Let us now assume that either
            $\algorecoverytaggen$ is not collision resistant or that an index
            $i$ is used more than once. Then $\env$ can generate several
            address requests $\msg{GenerateAdddress}{\msf{aux}}$ and observe the
            generated tags $\msf{wrt}$ within each address $\addr$. If
            $\algorecoverytaggen$ is not collision resistant, then $\env$ will
            observe a bias in the distribution of the output of
            $\algorecoverytaggen$ in the real world, \ie it will observe the
            same recovery tag for two different addresses.

        \item \emph{$\algohierarchicalkeygen$ hierarchical property and every
            index $i$ is used only once:} Assume now that
            $\algohierarchicalkeygen$ is not hierarchical for $\sigscheme$.
            Then, following the reasoning above, the execution of the protocol
            which uses $\algokeygen$ is indistinguishable from the ideal
            execution. However, by assumption the execution of $\ProtoW$ is not
            indistinguishable from the execution of that protocol anymore.
            Therefore, it is not indistinguishable from the execution of the
            ideal functionality as well.

    \end{itemize}

    Note that, in all cases, there is no mention of the simulator $\Iadv$,
    therefore the reasoning applies for any construction of $\Iadv$.

    In conclusion, we have shown that if either of the properties is broken,
    then the environment can distinguish between the two executions, thus a
    protocol that uses a scheme that does not provide one of the properties
    does not securely realize the ideal functionality $\FuncW$.
\end{proof}

\subsection{Security in the Fully Malleable Setting}

We also realize the functionality parameterized with the fully malleable
predicate $M_{\mathsf{FM}}$; the predicate description is given in
Algorithm~\ref{algo:fm_predicate}.

\begin{theorem}\label{thm:sec_malleable}
    Let the generic protocol $\ProtoW$ be parameterized by a signature scheme
    $\sigscheme$
    and the $\algorecoverytaggen$, $\algohierarchicalkeygen$, and
    $\mathsf{GenAddr}$ functions. Then $\ProtoW$ securely realizes the ideal
    functionality $\FuncW^{M_{\mathsf{FM}}}$ if and only if $\sigscheme$ is
    $\eufcma$, $\algorecoverytaggen$ is a collision resistant (cf.
    Definition~\ref{def:hash}), $\algohierarchicalkeygen$ is
    hierarchical for $\sigscheme$ as (cf.
    Definition~\ref{def:hierarchical_keygen_ind}), and $\mathsf{GenAddr}$ is
    collision resistant (cf. Definition~\ref{def:address_collision}).
\end{theorem}
\begin{proof}
    The proof follows similarly to the proof of
    Theorem~\ref{thm:sec_sink_malleable}, while not considering the
    non-malleability property of the address generation scheme.
\end{proof}

\subsection{Attacking the Malleable Protocol in the Non-Malleable Setting}

In this section, before presenting the attack,  we describe an additional
property of $\mathsf{GenAddr}$.  Next we show that with the extra property, in
a more restricted setting which is captured by the non-malleable predicate
$M_{\mathsf{NM}}$ of Algorithm~\ref{algo:nm_predicate}, the protocol cannot be
proven secure.

\paragraph{Metadata extraction for a posteriori malleability}
Given $\addr$, it is infeasible for the adversary to generate valid addresses,
\ie for the same verification payment key,  without access to at least the
metadata (for completeness, or the attribute list of the original address).
More concretely:

\begin{align}
    \Pr\left[
    \begin{tabular}{c}
    $l = (\attribute_{1}, \dots, \attribute_\addrgenlength)$, \\
    $\addr \leftarrow \mathsf{GenAddr}(l)$, \\
    $(\addr^\prime, \attribute^\prime_{i}, \dots, \attribute^\prime_\addrgenlength) \leftarrow \adversary^{\mathsf{GenAddr}^{d}(\cdot)}(\attribute_{i}, \dots, \attribute_\addrgenlength)$ \\
    \end{tabular}
     :
    \begin{tabular}{c}
    $(\mathsf{GenAddr}^{d}(\attribute^\prime_{i},\dots,\attribute^\prime_\addrgenlength)=\addr^\prime) \wedge $ \\
    $(\addr^\prime\neq\addr)\wedge$ \\
    $(\addr^\prime\notin \mathsf{Addrs})$ \\
    \end{tabular}
    \right]& \nonumber \\
    \leq \negl(\secparam) \nonumber
\end{align}
where the probabilities are computed on the randomness used in
$\mathsf{GenAddr}$ algorithm, every PPT adversary $\adversary$ and the values
of $l$. Furthermore, $\mathsf{Addrs}$ is the list of all the addresses received
from the $\mathsf{GenAddr}(\cdot)$ oracle.

\begin{remark*}
The \emph{a posteriori malleability} property implicitly
requires that $\mathsf{GenMeta}$ is \emph{metadata extraction resistant},
otherwise the metadata are easily accessible by the adversary.
\end{remark*}

We now describe the attack on the protocol $\ProtoW$ instantiated with an a
posteriori malleable $\mathsf{GenAddr}$ algorithm.

\begin{theorem} \label{thm:nm-sec}
    Let the protocol $\ProtoW$ be instantiated with a collision resistant and a
    posteriori malleable $\mathsf{GenAddr}$, a metadata extraction resistant
    $\mathsf{GenMeta}$ function as defined in the a posteriori malleability
    setting of Section~\ref{subsec:addrgen_properties}, an $\eufcma$ signature scheme
    $\sigscheme$, a collision resistant $\algorecoverytaggen$, and a hierarchical
    $\algohierarchicalkeygen$ for $\sigscheme$ as per
    Definition~\ref{def:hierarchical_keygen_ind}. Then, $\ProtoW$ does not
    securely realize the ideal functionality $\FuncW^{M_{\mathsf{NM}}}$.
\end{theorem}
\begin{proof}
    We construct an environment $\env$ which distinguishes efficiently between the
    ideal execution of $\FuncW^{M_{\mathsf{NM}}}$ and $\Iadv$, and the real
    one given by $\ProtoW$ and $\adv$. The attack exploits the malleable
    feature of the address construction regarding the exposure of the metadata
    while spending the funds.

    The environment issues an address via party $\party_1$, say $\addr_1$, now assume
    without loss of generality that $\addr_1$ receives some funds  during the
    execution. Now $\env$ issues an address  generation request for party
    $\party_2$, therefore it would receive $\addr_2$. The next step is to generate a
    transaction from $\addr_1$ to $\addr_2$, which $\env$ can do by issuing
    message $\msg{Pay}{\assetset, \addr_1, \addr_2}$ from $\party_1$.
    $\env$ receives $\msg{Transaction}{\tx, \mathsf{Sign}(\paymentkeysign, \tx)}$  such that $\tx = (\assetset, \addr_1,
    \addr_2,m)$ for metadata $m$.

    At this point, the property metadata extraction resistance of the
    $\mathsf{GenMeta}$ nor $\mathsf{GenAddr}$ help any longer, since $m$ is
    exposed to $\env$. It is fair to assume that $\env$ uses $m$ to generate
    locally, \ie without using the \emph{Address Generation} interface, the
    address $\addr_{local}$.

    Next, for the sake of argument, assume the existence of a simulator $\Iadv$
    which perfectly simulates the ideal execution, therefore we can also assume
    that eventually $\addr_{local}$ has assets $\assetset^\prime$. At this
    point, $\env$ submits the message $\msg{Pay}{\assetset^\prime,
    \addr_{local},\addr}$ for any party $\party$ and some (correctly generated)
    address $\addr$, then the environment $\env$ receives
    $\msg{Transaction}{\tx ,\sign}$  such that $\tx = (\assetset^\prime,
    \addr_{local}, \addr,m^\prime)$ for metadata $m^\prime$.

    Finally, in order to distinguish the executions, $\env$ submits
    $\msg{VerifyPayment}{\tx, \sign}$ to any party, for which it  receives
    $\msg{VerifiedPayment}{\tx, \sign,b}$. Note that $b = 1$ if it is a real
    execution by $\ProtoW$  and $\adv$, because $b = \mathsf{Verify}(\tx, \sign,
    \paymentkeyverify)$. On the other hand, due to the fully non-malleable
    predicate $M_{\mathsf{NM}}$, which would output $0$, $b = 0$.

    $\env$ efficiently distinguishes between the executions,
    thereby $\ProtoW$, in the a posteriori malleability setting, does not
    securely realizes the ideal functionality $\FuncW^{M_{\mathsf{NM}}}$.
\end{proof}
