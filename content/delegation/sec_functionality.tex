\section{The Core-Wallet Functionality}\label{sec:basicdef}

The major contribution of this chapter is the definition of the ideal
functionality of the core wallet. The goal of this definition is to distill, in
a concise way, a formal model of the properties of a PoS wallet.

Our ideal functionality $\FuncW$ is inspired by
Canetti~\cite{EPRINT:Canetti03}. $\FuncW$ (Figures~\ref{fig:FWalletCore-1}
and~\ref{fig:FWalletCore-2}) interacts with the ideal adversary $\Iadv$ and a
set of parties denoted by $\partyset$ and is parameterized by the predicate
$M(\cdot, \cdot, \cdot)\rightarrow\{0,1\}$. It also keeps the, initially empty
lists, $S$ of staking actions and $\mc{T}$ of transactions. We assume, without
loss of generality, that, given a list of attributes $\addressgenlist =
(\attribute_1, \ldots, \attribute_\addrgenlength)$, $\attribute_1$ is the
staking key's information and $\attribute_2$ is a recovery tag (which will be
further investigated in the upcoming Section~\ref{sec:address}). We remind that
access to an address implies access to its public attributes $d =
(\attribute_1, \ldots, \attribute_{p-1})$ and, given the ledger, access to its
semi-public attributes $(\attribute_p, \ldots, \attribute_{i-1})$ as well.

The functionality distinguishes the addresses in three types: base,
pointer, and exile. As we will show in Section~\ref{sec:address}, each
type has a specific utility; briefly, base addresses help bootstrap a wallet,
pointer addresses are shorter and aim at better performance, and exile
addresses are withdrawn from the PoS protocol's execution.

\myhalfbox{Functionality $\FuncW^M$}{white!40}{white!10}{
   \noindent \emph{Initialization:}
        Upon receiving $\msg{Init}{}$ from $\party \in \partyset$, forward it
        to $\Iadv$ and wait for $\msg{InitOk}{}$. Then initialize the empty
        lists $L_{\party}$ of addresses and attribute lists and $K_{\party}$ of
        staking keys, and send $\msg{InitOk}{}$ to $\party$.

    \noindent \emph{Address Generation:}
        Upon receiving $\msg{GenerateAddress}{\msf{aux}}$ from $\party \in \partyset$,
        forward it to the $\Iadv$. Upon receiving $\msg{Address}{\addr,
        \addresslist}$ from $\Iadv$, parse $\addresslist$ as $\addresstuple$
        and $\forall \party' \in \partyset$ check if $\forall (\addr^\prime,
        \addresstupleprime) \in L_{\party'}$ it holds that $\addr \neq
        \addr^\prime$, $\attribute'_2 \neq \attribute_2$, and $\forall j \in
        [i, \ldots, \addrgenlength]: \attribute'_j \neq \attribute_j$, \ie the
        address, recovery tag, and private attributes are unique. If so, then:
        \begin{enumerate}
            \item if $\msf{aux} = (\msf{base})$, check that $\forall
                (\addr^\prime, \addresstupleprime) \in L_{\party}: \attribute'_1 \neq
                \attribute_1$, \ie the new staking key is unique,
            \item else if $\msf{aux} = (\msf{pointer}, \stakingkeyverify)$,
                check that $\attribute_1  = \stakingkeyverify$,
            \item else if $\msf{aux} = (\msf{exile})$, check that
                $\attribute_1  = \bot$.
        \end{enumerate}
        If the checks hold or $\party$ is corrupted, then insert $(\alpha,
        \addresslist)$ to $L_{\party}$ and return $\msg{Address}{\addr}$ to $\party$. If
        $\msf{aux} = (\msf{base})$ also insert $\attribute_1$ to $K_{\party}$ and
        return $\msg{StakingKey}{\attribute_1}$ to $\party$.

   \noindent \emph{Wallet Recovery:}
        Upon receiving $\msg{RecoverWallet}{i}$ from $\party \in \partyset$, for
        the first $i$ elements in $L_{\party}$ return $\msg{Tag}{\attribute_2}$.

   \noindent \emph{Address Recovery:}
        Upon receiving a message $\msg{RecoverAddr}{\addr, i}$ from a party $\party
        \in \partyset$, if $(\alpha, l)$ is one of the first $i$ elements of
        $L_{\party}$ or $M(L_{\party}, \msf{recover}, \addr) = 1$, return
        $\msg{RecoveredAddr}{\addr}$.

   \noindent \emph{Issue Transaction:}
        By receiving from $\party \in \partyset$ the message $\msg{Pay}{\assetset,
        \addr_s, \addr_r, m}$, if $\exists \addresslist:
        (\addr_s, \addresslist) \in L_{\party}$ forward it to $\Iadv$. Upon
        receiving $\msg{Transaction}{\tx, \sign}$ from $\Iadv$, such that
        $\tx = (\assetset, \addr_s, \addr_r, m)$, check if
        $\forall (\tx', \sign',b') \in \mc{T}: \sign' \neq \sign$, if $(\tx,
        \sign, 0) \not \in \mc{T}$, and if $M(L_{\party}, \msf{issue},
        \addr_r) = 1$. If all checks hold, then insert $(\tx, \sign, 1)$ to
        $\mc{T}$ and return $\msg{Transaction}{\tx, \sign}$.
}{\label{fig:FWalletCore-1} The PoS core-wallet ideal functionality. (Part 1)}

\myhalfbox{Functionality $\FuncW^M$}{white!40}{white!10}{
    \noindent \emph{Verify Transaction:}
        Upon receiving from $\party \in \partyset$ the message $\msg{VerifyPay}{\tx,
        \sign}$, with $\tx = (\assetset, \addr_s, \addr_r, m)$
        for a metadata string $m$, forward it to $\Iadv$ and wait for a reply
        message $\msg{VerifiedPay}{\tx, \sign, \phi}$. Then:
        \begin{itemize}
            \item if $M(L_{\party}, \msf{verify}, \addr_s) = 0$, set $f = 0$
            \item else if $(\tx, \sign, 1) \in \mc{T}$, set $f = 1$
            \item else, if $\party$  is not corrupted and $(\tx, \sign, 1) \not \in
                \mc{T}$, set $f = 0$ and insert $(\tx, \sign, 0)$ to
                $\mc{T}$
            \item else, if $(\assetset, \addr_s, \addr_r, m, \sign, b) \in
                \mc{T}$, set $f = b$
            \item else, set $f = \phi$.
        \end{itemize}
        Finally, send $\msg{VerifiedPay}{\tx, \sigma, f}$ to $\party$.

    \noindent \emph{Issue Staking:}
        Upon receiving $\msg{Stake}{\tx_{s}}$ from $\party$, such that $\tx_{s} =
        (\stakingkeyverify, m)$ for a metadata string $m$, forward the message
        to $\Iadv$. Upon receiving $\msg{Staked}{\tx_{s}, \sign}$ from $\Iadv$, if
        $\forall (\tx_{s}', \sign', b') \in S: \sign' \neq \sign$, $(\tx_{s}, \sign, 0)
        \not \in S$, and $\stakingkeyverify \in K_{\party}$, then add $(\tx_{s}, \sign,
        1)$ to $S$ and return $\msg{Staked}{\tx_{s}, \sign}$ to $\party$.

    \noindent \emph{Verify Staking:}
        Upon receiving from $\party \in \partyset$ the message
        $\msg{VerifyStake}{\tx_{s}, \sign}$, with $\tx_{s} = (\stakingkeyverify, m)$,
        forward it to $\Iadv$ and wait for $\msg{VerifiedStake}{\tx_{s}, \sign,
        \phi}$. Then find $\party_s$, such that $\stakingkeyverify \in
        K_{\party_s}$, and:
        \begin{itemize}
            \item if $(\tx_{s}, \sign, 1) \in S$, set $f = 1$
            \item else if $\party_s$ is not corrupted and $(\tx_{s}, \sign, 1)
                \not \in S$, set $f = 0$ and insert $(\tx_{s}, \sign, 0)$ to $S$
            \item else if exists an entry $(\tx_{s}, \sign, f') \in S$, set $f = f'$
            \item else set $f = \phi$ and insert $(\tx_{s}, \sign, \phi)$ to $S$.
        \end{itemize}
        Finally, return $\msg{VerifiedStake}{\tx_{s}, \sign, f}$ to $\party$.
}{\label{fig:FWalletCore-2} The PoS core-wallet ideal functionality. (Part 2)}

\begin{remark*}
Although $\FuncW$ offers a suitable security definition, for our
requirements, as it relies on the \emph{standard} security properties of
digital signatures and the address generation properties to be described in
Section~\ref{subsec:addrgen_properties}, it does not offer any type of
\emph{forward security} in the sense of Bellare and Miner~\cite{C:BelMin99}.
However, protocols which require stronger security properties for their
building blocks \emph{do} exist. For instance, Ouroboros Praos~\cite{EC:DGKR18}
relies on a forward secure digital signature scheme, among other primitives, to
provide security guarantees against fully-adaptive corruption in a
semi-synchronous setting. Additionally, protocols like
Cryptonote~\cite{van2013cryptonote} allow an arbitrary number of parties to
operate the address generation interface, instead of restricting it to the
wallet's owner.
\end{remark*}
