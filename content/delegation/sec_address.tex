\section{PoS Addresses: Construction and Recovery}\label{sec:address}

The final step in fully realizing the core wallet is to implement the functions
used by the protocol $\ProtoW$, more importantly the address generation
function. In the following paragraphs, we outline the three types of addresses
in our framework, \ie \emph{base}, \emph{pointer}, and \emph{exile}. We
concretely describe an address's attributes and how an index is used to
generate a ``child'' address. Then, we present multiple address schemes, each
offering different levels of protection against malleability attacks, resulting
in different address lengths, but all easily implementable with standard
cryptographic primitives.

\subsection{Address Types and their Attributes}\label{subsec:address-attributes}

As discussed above, at least two attributes are required per address, \ie the
staking $\stakingkeypair$ and the payment $\paymentkeypair$ key pairs. As shown
in Section~\ref{sec:basicdef}, the signing keys $\stakingkeysign$ and
$\paymentkeysign$ are \emph{private}, while the verification keys
$\paymentkeyverify$ and $\stakingkeyverify$ are \emph{semi-public} and
\emph{public} attributes respectively. We remind that $\paymentkeypair$ is used
in proving ownership of the assets and issuing payments, whereas
$\stakingkeypair$ is used to perform staking actions on behalf of the assets.

The first step in computing a ``child'' address and its attributes is the
choice of an index $i$ from the set $\indexSet$. As with hardware wallets
(Chapter~\ref{chap:hardware-wallets}), an index is an identifier that is used
to generate a ``child'' key. We define a list of domains $[ \indexSet_1,
\indexSet_2, \ldots, ]$, where each $\indexSet_i$ has a finite, relatively
small, cardinality. During address generation, the wallet initially picks
indexes from $\indexSet_1$. After all indexes in $\indexSet_1$ have been
used, it uses $\indexSet_2$ and so on. It is required that at least one
address for an index in $\indexSet_j$ is published on the ledger, \ie is on
the sending or receiving end of a transaction, before indexes from
$\indexSet_{j+1}$ are used. During recovery, the wallet sets $j = 1$ and
generates all indexes in $\indexSet_j$. It then constructs the recovery tag
for each index and compares it with each address in the blockchain. If at least
one index has been used in a published address, then the wallet sets $j = j +
1$ and repeats for $\indexSet_{j+1}$. When, for some $j$ no index corresponds
to any published address, recovery is complete.

We now discuss the complexity of the recovery procedure. Given that the
cardinality of $\indexSet$ is small, the probability that the same index is
chosen twice, even for a random choice, is not negligible. Therefore, two
devices that maintain the same wallet core would need to share the state of the
indexes that have been used by each. The number of indexes and addresses that
the wallet can generate is not restricted, since the set of domains is
infinite. In terms of complexity, this naive scheme is linear to the number of
addresses in the ledger. We can improve this with an index of published
addresses, based on their recovery tag. Using such index, the recovery
complexity becomes linear to the cardinality of
$\bigcup\limits_{{\scriptscriptstyle j}} \indexSet_j$, \ie the number of
addresses that the wallet owns and has published. Since the recovery tags are
public, such index can be constructed and circulated on the network by
everybody.

A hierarchical key, which is derived from the wallet's master key
$\masterprivkey$ and is linked to a child address, is created by
$\algohierarchicalkeygen$. This function takes the master key, a label
$\texttt{lbl} \in \{ \msf{payment}, \msf{staking} \}$ and an index $i$ and
passes them to a pseudorandom function, which outputs a pseudorandom value
passes it to a pseudorandom number generator that outputs $\poly(\secparam)$
bits $\rho$, for some suitable polynomial $\poly$. These bits are then passed
as random coins to the key generation function $\algokeygen(1^\secparam;
\rho)$. Therefore, to generate a child payment key, the protocol runs
$\algohierarchicalkeygen(\langle \masterprivkey, \msf{payment}, i \rangle)$, while
similarly $\texttt{lbl} = \msf{staking}$ is used to issue a new staking
key.

The wallet produces three types of addresses, differentiated by the staking
object $\beta$. To produce a \emph{base} address, the wallet computes a staking
key $\stakingkeyverify$ and sets $\beta$ as the hash of it. For a
\emph{pointer} address, the address's staking key is set indirectly.
Specifically, the staking object $\beta$ is a delegation pointer $ptr$. The
pointer is a string that identifies a published certificate, \ie the
representation of a staking action on the ledger, which is described in detail
in Section~\ref{sec:delegation-transaction}. If $ptr$ points to a valid
delegation certificate $\tau_{del}$, then the address's staking key is the
delegate's key in $\tau_{del}$, whereas, if $ptr$ points to a registration
certificate, then the delegate's key is the key of the stake pool defined in
that certificate.  Finally, for an \emph{exile} address, the staking object is
a fixed value $\epsilon = \bot$. Since $\epsilon$ does not identify a staking
key, the owner of an exile address cannot perform staking actions or delegate
the address's staking rights, so all assets owned by such addresses are
effectively removed from the PoS protocol.

Each address also contains the recovery tag $\msf{wrt}$, \ie a public parameter which
allows the identification of addresses. The tag is created by the function
$\algorecoverytaggen$ and links the address to the attribute list, without
revealing the semi-public attributes.  Recovery is a process that relies only
on the master key of the wallet, so the wallet should be able to recover an
address by \emph{only knowing its payment key}. In the simplest setting, during
recovery the wallet computes the keys for its indexes and hashes them to
compute the recovery tags. Therefore, $\algorecoverytaggen$ is the hash
function $\hash$ and, by definition, is collision resistant as required.

\paragraph{Searchable recovery}\label{subsubsec:searchable_tags}
This design builds on the premise of searchable encryption
\cite{C:BelBolONe07}. Specifically, access to the wallet's master key enables
searching all addresses in the ledger and recognize the ones that belong to the
wallet.  Assuming an instance of a semantically secure symmetric encryption
scheme $\langle \mathsf{Enc, Dec} \rangle$, the hierarchical tag generation
function $\mathsf{SearchableTagGen}$ computes the output of $\mathsf{Enc}$,
using $\masterprivkey$ as the encryption key and the index $i$ as the plaintext:
\begin{align}
    \mathsf{sht} = \mathsf{Enc}(\masterprivkey, i)
\end{align}
During recovery, the wallet parses all addresses in the blockchain and decrypts
the tag $\mathsf{sht}'$ in each as $i^\prime = \mathsf{Dec}(\masterprivkey, \mathsf{sht}')$. If
the output is a well-formed index $i' \in \mc{I}$, the address
belongs in the wallet and its hierarchical index is $i'$. Since the
output of the encryption function $\mathsf{Enc}$ is by definition random in the
domain of $\mathsf{Enc}$, the recovery tag generation is a PRF as needed.

\paragraph{Malleability verification tags}\label{subsubsec:verifiable_malleability_tags}
These tags are used to verify whether a part of the address
has been altered during transit. Specifically, the motivation is for the
address owner to identify whether a malleability attack has occurred, even in
case the transaction is accepted by the system.  The tag is computed as
follows:
\begin{align}
    \mathsf{mht} = \hash(\masterprivkey || i || \beta)
\end{align}
where $\beta$ is the staking attribute, for which the wallet needs to identify
possible malleability attacks. Clearly the tag generation is again a
pseudorandom function. Also, without the master key $\masterprivkey$, an attacker cannot
forge such tag. However, only the wallet owner can verify that such tag is
properly constructed for a given address, \ie it is not sufficient to construct
non-malleable address schemes.

\subsection{Malleable Addresses}\label{subsubsec:malleable_addrgen}

As discussed in Section~\ref{subsec:addrgen_properties}, The two basic
properties of an address generation function are \emph{collision resistance}
and \emph{non-malleability}. On the one hand, collision resistance is a
feature that has been extensively investigated in the literature. On the other
hand, address non-malleability, in the manner presented in this chapter, is a
novel notion.  Therefore, let us first present a malleable scheme and the
problems that stem from it.

Suppose two users $\party_{A}$ and $\party_{B}$. $\party_{B}$ wishes to receive
some assets from $\party_{A}$, so she generates an address $\addr$ and gives it
to $\party_{A}$; the payment key of $\addr$ is $\paymentkeyverify$ and the
staking object is $\beta$. If the address generation function is malleable,
$\party_{A}$ can create an address $\addr'$, where the payment key is again
$\paymentkeyverify$ but the staking object is $\beta'$. Notice that
$\party_{A}$ only knows the address $\addr$, \ie neither the payment key
$\paymentkeyverify$ nor any other information, \eg other addresses that
$\party_{B}$ owns. Now, $\party_{B}$ can spend from $\addr^\prime$, so
$\party_{A}$ can claim that the payment is valid and complete, even though
$\party_{A}$ has effectively chosen the key that controls that stake. To make
matters worse, the malleability attack may go unnoticed by $\party_{B}$, unless
she keeps a state of her generated addresses and compares with the forged one.

A malleable address is constructed by concatenating the hash of the
verification payment key $\paymentkeyverify$ and the staking object $\beta$, so
$\addr = \hash(\paymentkeyverify) || \beta$. The value
$\hash(\paymentkeyverify)$ acts as both the association of the address with the
payment key and the wallet recovery tag $\msf{wrt}$. The staking object $\beta$ takes
the following forms, depending on the type of address:
\begin{inparaenum}[i)]
    \item \emph{base:} $\beta = \hash(\stakingkeyverify)$;
    \item \emph{pointer:} $\beta = \mathsf{getPointer}(\stakingkeyverify)$;
    \item \emph{exile:} $\beta = \epsilon$.
\end{inparaenum}
The malleable address generation function is defined in
Algorithm~\ref{alg:malleable_addrgen}, while
Lemma~\ref{lem:malleable_addrgen_collision} shows that $\mathsf{MGenAddr}$ is
collision resistant.

\begin{algorithm}
    \begin{algorithmic}
        \Function {$\mathsf{MGenAddr}$}{$\addressgenlist$}
            \State $(\msf{aux}, st, \paymentkeyverify) = \mathsf{parse}(\addressgenlist)$
            \Switch {$\msf{aux}$}
                \Case {$\msf{base}$}
                    \State $\beta = \hash(\stakingkeyverify)$
                \EndCase
                \Case {$\msf{pointer}$}
                    \State $\beta = \mathsf{getPointer}(\stakingkeyverify)$
                \EndCase
                \Case {$\msf{exile}$}
                    \State $\beta = \epsilon$
                \EndCase
            \EndSwitch
            \State $\addr = \hash(\paymentkeyverify) || \beta$
            \State \Return $\addr$
        \EndFunction
    \end{algorithmic}
    \caption{
        The malleable address generation function, parameterized by
        $\hash(\cdot)$. The input is a tuple $\addressgenlist$, consisting of
        the auxiliary information $\msf{aux}$ and the attributes.
    }
    \label{alg:malleable_addrgen}
\end{algorithm}

\begin{lemma}\label{lem:malleable_addrgen_collision}
    $\mathsf{MGenAddr}$ is collision resistant if $\hash$ is collision
    resistant.
\end{lemma}
\begin{proof}
    The proof is by contradiction. Suppose an adversary produces two attribute
    lists $l_1 = (\msf{aux}_1, \attribute_1, \paymentkeyverify_1)$, $l_2 =
    (\msf{aux}_2, \attribute_2, \paymentkeyverify_2)$, such that $l_1 \neq l_2$, which
    correspond to the addresses $\addr_1 = \mathsf{MGenAddr}(l_1) =
    \hash(\paymentkeyverify_1) || \beta_1$ and $\addr_2 =
    \mathsf{MGenAddr}(l_2) = \hash(\paymentkeyverify_2) || \beta_2$.

    If the adversary finds a collision of addresses, then $\addr_1 = \addr_2$
    and it also holds that $\hash(\paymentkeyverify_1) =
    \hash(\paymentkeyverify_2)$ and $\beta_1 = \beta_2$. However, by assumption
    it holds that $l_1 \neq l_2$, so it also holds that $\paymentkeyverify_1
    \neq \paymentkeyverify_2$ and the adversary has found a collision for
    $\hash$.
\end{proof}

$\mathsf{MGenAddr}$ is malleable since, for every address $\addr =
\hash(\paymentkeyverify) || \beta$ that $\party_{B}$ gives to $\party_{A}$,
$\party_{A}$ can create a new address as $\addr^\prime =
\hash(\paymentkeyverify) || \beta^\prime$ for some staking object $\beta^\prime
\neq \beta$. Also, $\party_{B}$ cannot identify a forgery without keeping track
of the addresses that she has honestly generated. However, this is not always
feasible, \eg wallet recovery, when the wallet owner knows the payment keys but
does not necessarily keep track of the staking object for each address in the
wallet.

\paragraph{Verifiably malleable addresses}
This scheme is similar to the above, with the addition that the wallet can
identify a forgery.  The address $\addr$ is now constructed by concatenating
the string generated by $\mathsf{MGenAddr}$, as before, and the malleability
verification tag, which is constructed as in
Section~\ref{subsubsec:verifiable_malleability_tags}:
\begin{align}
    \mathsf{mht} = \hash(\masterprivkey || i || \beta) \\
    \addr = \hash(\paymentkeyverify) || \beta || \mathsf{mht}
\end{align}

This scheme is also susceptible to the malleability attack described above,
since, given $\addr = \hash(\paymentkeyverify) || \beta || \mathsf{mht}$, $\adversary$
can create a forgery like $\addr' = \hash(\paymentkeyverify) || \beta'
|| \mathsf{mht}$. However, now the owner of the wallet can compare the staking object
$\beta'$ with the tag $\mathsf{mht}$ and identify the attack, while also using
$\hash(\paymentkeyverify)$ as the recovery tag.

\subsection{A Posteriori Malleable Addresses}\label{subsec:non_malleable_addrgen}

In this section we describe various schemes of a posteriori malleable address
generation. First, Algorithm~\ref{alg:apost-addrgen} defines an a posteriori
verifiable malleable address generation function $\mathsf{PNMGenAddr}$.

\begin{algorithm}
    \begin{algorithmic}
        \Function {$\mathsf{PNMGenAddr}$}{$\addressgenlist$}
            \State $(\msf{aux}, \stakingkeyverify, \paymentkeyverify, \msf{ht}) = \mathsf{parse}(\addressgenlist)$
            \Switch {$\msf{aux}$}
                \Case {$\msf{base}$}
                    \State $\beta = \hash(\stakingkeyverify)$
                \EndCase
                \Case {$\msf{pointer}$}
                    \State $\beta = \mathsf{getPointer}(\stakingkeyverify)$
                \EndCase
                \Case {$\msf{exile}$}
                    \State $\beta = \epsilon$
                \EndCase
            \EndSwitch
            \State $\msf{root} = \hash(\paymentkeyverify || \msf{ht} || \beta)$
            \State $\addr = \msf{root} || \msf{ht} || \beta$
            \State \Return $\addr$
        \EndFunction
    \end{algorithmic}
    \caption{
        A posteriori malleable address generation function, parameterized
        by $\hash(\cdot)$. The input is a tuple $\addressgenlist$, consisting
        of the auxiliary information $\msf{aux}$ and the attributes.
    }
    \label{alg:apost-addrgen}
\end{algorithm}

In this case, $\adversary$ cannot act as above. Specifically, if $\adversary$
changes $\beta$ to some $\beta^\prime$, without changing $\msf{root}$, then, upon
spending from the address, $\msf{root}$ will be invalid. Since $\adversary$ does not
know $\paymentkeyverify$, it cannot construct a valid $\msf{root}$ for both
$\paymentkeyverify$ and $\beta'$, unless explicitly asking $\party_{B}$ for
such address. Alternatively, if she only changes $\msf{root}$ and keeps $\beta$, it
can be easily found that $\msf{root} \neq \hash(\paymentkeyverify || \msf{ht} || \beta)$,
so the payment is rejected and the assets are effectively burnt in the
malformed address. We note that tag $\msf{ht}$ is necessary, since $\msf{root}$ includes
the staking object, so a wallet that knows only the payment key could not
recreate it during recovery.

However, this scheme allows for a posteriori malleability, as defined in
Section \ref{subsec:malleability_predicate}. Specifically, $\adversary$ can
perform the attack if she knows a past key of the wallet, for which she wishes
to generate the forged address. In this case, both the wallet recognizes
the forged address as its own and the owner is able to spend from it.

Finally, as with the malleable addresses, this scheme can be transformed into a
verifiable one via a verification tag. The tag would be $\msf{ht} = (\msf{ht}_1, \mathsf{mht})$,
where $\msf{ht}_1$ is the recovery tag, for either a predefined index $\msf{iht}$ or a
searchable one $\mathsf{sht}$, and $\mathsf{mht}$ is the malleability verification tag.

\subsection{Sink Malleable Addresses}\label{subsec:nm-genaddr}

Our final address scheme is a sink malleable construction. The core idea is to
certify the staking object with the payment key; thus, to produce a forgery,
the adversary needs to forge a payment key's signature. A sink malleable
address is as follows:
\begin{align}
    \addr = \hash(\paymentkeyverify) || \beta || \algosign(\paymentkeysign, \beta)
\end{align}
The value $\hash(\paymentkeyverify)$ associates the address with the payment
key, while also serving as the recovery tag $\msf{wrt}$. The staking object $\beta$
for the three address types is:
\begin{inparaenum}[i)]
    \item \emph{base:} $\beta = \hash(\stakingkeyverify)$;
    \item \emph{pointer:} $\beta = \mathsf{getPointer}(\stakingkeyverify)$;
    \item \emph{exile:} $\beta = \epsilon$.
\end{inparaenum}

In practice, upon issuing a transaction and revealing the payment key
$\paymentkeyverify$, everybody can check the signature to identify forgeries.
Algorithm~\ref{alg:sink_addrgen} formally defines the sink malleable
construction, while Lemmas~\ref{lem:sink_addrgen_collision}
and~\ref{lem:sink_addrgen_attribute_mall} prove that our scheme is both
collision resistant and non-malleable.

\begin{algorithm}
    \begin{algorithmic}
        \Function {$\mathsf{SinkGenAddr}$}{$\addressgenlist$}
            \State $(\msf{aux}, st, \paymentkeyverify, \paymentkeysign) = \mathsf{parse}(\addressgenlist)$
            \Switch {$\msf{aux}$}
                \Case {$\msf{base}$}
                    \State $\beta = \hash(st)$
                \EndCase
                \Case {$\msf{pointer}$}
                    \State $\beta = \mathsf{getPointer}(st)$
                \EndCase
                \Case {$\msf{exile}$}
                    \State $\beta = st$
                \EndCase
            \EndSwitch
            \State $\addr = \hash(\paymentkeyverify) || \beta || \algosign(\paymentkeysign, \beta)$
            \State \Return $\addr$
        \EndFunction
    \end{algorithmic}
    \caption{
        The sink malleable address generation function, parameterized by a hash
        $\hash(\cdot)$ and a signature scheme $\sigscheme$. The input is a
        tuple $\addressgenlist$, consisting of the auxiliary information $\msf{aux}$
        and the attributes.
    }
    \label{alg:sink_addrgen}
\end{algorithm}

\begin{lemma}\label{lem:sink_addrgen_collision}
    $\mathsf{SinkGenAddr}$ is collision resistant if $\hash$ is collision
    resistant.
\end{lemma}

\begin{proof}
    Let $l_1 = (\msf{aux}_1, st_1, \paymentkeyverify_1,
    \paymentkeysign_1)$, $l_2 = (\msf{aux}_2, st_2, \paymentkeyverify_2,
    \paymentkeysign_2)$ be two attribute lists, $l_1 \neq l_2$, corresponding to
    addresses $\addr_1 = \mathsf{MGenAddr}(l_1) =
    \hash(\paymentkeyverify_1) || \beta_1 ||
    \algosign(\paymentkeysign_1, \beta_1)$ and $\addr_2 =
    \mathsf{MGenAddr}(l_2) = \hash(\paymentkeyverify_2) || \beta_2 ||
    \algosign(\paymentkeysign_2, \beta_2)$.

    If $\addr_1 = \addr_2$ then it also holds that $\hash(\paymentkeyverify_1)
    = \hash(\paymentkeyverify_2)$, $\beta_1 = \beta_2$, and
    $\algosign(\paymentkeysign_1, \beta_1) = \algosign(\paymentkeysign_2,
    \beta_2)$. However, by assumption we have $l_1 \neq l_2$, so it also holds
    that $\paymentkeyverify_1 \neq \paymentkeyverify_2$ and thus the adversary
    has found a collision for $\hash$ on the two values of the payment
    verification keys and signatures on the staking object.
\end{proof}

\begin{lemma}\label{lem:sink_addrgen_attribute_mall}
    $\mathsf{SinkGenAddr}$, when parameterized with a signature scheme
    $\Sigma$, is attribute non-malleable if $\Sigma$ is $\eufcma$.
\end{lemma}

\begin{proof}
    Assume the key pair $(\paymentkeyverify,\paymentkeysign)$ and the address
    $\addr = \hash(\paymentkeyverify) || \beta || \algosign(\paymentkeysign,
    \beta)$, for staking object $\beta$.
    Also assume the existence of an adversary $\adversary$ who breaks the
    attribute non-malleability property of $\mathsf{SinkGenAddr}$. We will
    construct a forger $F$ for the signature scheme, which simulates the
    security game for $\adversary$. The forger works as follows.
    $F$ receives a key $\paymentkeyverify$ and has access to the signing
    oracle. It sets the attribute list $l = (\paymentkeyverify, \msf{aux}, \beta,
    \msf{wrt})$ and initializes $\adversary$ with public attributes $(\msf{aux}, \beta,
    \msf{wrt})$.  Note that $F$ answers generation address queries by using its own
    signature oracle. That is, upon receiving $(\msf{aux}_i, \beta_i, \msf{wrt}_i)$, it
    issues a signature query on $\beta_i$ and generates a new address
    $\addr_i$.  Moreover, $F$ may receive a metadata query on issued addresses
    $\addr_{i}$ and answer by revealing $\paymentkeyverify$.
    By hypothesis $\adversary$ outputs a list $(\addr^\ast, \msf{aux}^\ast,
    \beta^\ast, \msf{wrt}^\ast)$, as per the attribute non-malleability game, for
    which it holds that $\mathsf{SinkGenAddr}(\paymentkeysign,
    \paymentkeyverify, \msf{aux}^\ast, \beta^\ast, \msf{wrt}^\ast) \rightarrow \addr^\ast$,
    where $\addr^\ast$ was not queried during the game and $\addr^\ast \neq
    \addr$, where $\addr$ is the original address provided during the
    challenge: $\mathsf{SinkGenAddr}(\paymentkeysign, \paymentkeyverify, \msf{aux},
    \beta, \msf{wrt}) \rightarrow \addr$.
    Since $\adversary$ is successful, the signature holds for both $\addr$ and
    $\addr^\ast$, so $F$ uses $\addr^\ast = (\hash(\paymentkeyverify ||
    \sign^\ast), \beta^\ast)$ to output $(\addr^\ast, \sign^\ast)$ as its pair
    of forged message and signature.
\end{proof}

\begin{remark*}
In an attempt to shorten the address, one may entertain the idea
of including the signature in the hashed data. However, in such case, the hash
could not act as the recovery tag anymore, since both the staking object
$\beta$ and the signature cannot be predicted, given only the address's payment
key. Therefore, either the wallet's recovery becomes necessarily linear to the
number of addresses in the ledger, which is a significant performance overhead
compared to the recovery mechanism of Section~\ref{subsec:address-attributes},
or an additional component is introduced, \ie a dedicated recovery tag, which
effectively counters the address shortening effort. Alternatively, the address
could contain the signature's hash, rather than the signature itself, as the
signature is only checked after a payment is issued, \ie when
$\paymentkeyverify$ is revealed. However, although such design could reduce the
address's length, it would also increase the ledger's overall size, since both
the signature and its hash would be published.
\end{remark*}
