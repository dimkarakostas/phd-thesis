\section{Execution Model}\label{sec:execution-model}

In our setting, the ledger's maintainers are the stake pools. An adversary can
break the ledger's properties by corrupting a set of stake pools which are
allocated a majority of the total stake. Consequently, since we assume that the
majority of stake is owned by honest stakeholders, the adversary needs to
corrupt at least one pool to which honest players delegate. Especially in the
collective setting, the adversary may control a subset of the pool's members,
although not a majority of them. Therefore, our work considers adversaries who
attempt to violate the security of a collectively-owned stake pool while
controlling a minority of its members. Security violations include producing
invalid blocks or transactions, as well as abstaining from join the consensus
protocol.

In terms of message delivery, we assume a synchronous network.  As such, the
adversary may delay messages up to an upper bound. This assumption will prove
particularly useful in the implementation of the collective pool protocol,
during which participants employ consensus and broadcast algorithms. We note
that, although this assumption is realizable in a setting where parties are
well-connected, it does not cover possible real-world threats, such as eclipse
attacks. Nonetheless, exploring variations of the collective pool functionality
and protocol designs over semi-synchronous or asynchronous networks presents an
interesting problem, which would offer a more robust implementation.

\subsection{Weighted Threshold Digital Signatures}\label{sec:thold-sign}

In a weighted threshold digital signature
scheme~\cite{morillo1999weighted,Shamir79}, each party $\party$ is associated
with a (integer) weight $\wfunc[\party] \geq 0$, where $\wfunc$ is a mapping of
players to weights. A signature can be produced by any set of keys, the
aggregate weight of which is above the defined threshold. The weighted
threshold signature scheme (Definition~\ref{def:thresh-sig}) is constructed by
combining a digital signature scheme (Definition~\ref{def:digsign}) with a
weighted threshold secret sharing scheme (Definition~\ref{def:thresh-ss}).
Additionally, standard threshold signatures is a special case of the weighted
variant, with $\wfunc[\party] = 1$ for every party $\party$.

\begin{definition}[Weighted Threshold Secret Sharing]\label{def:thresh-ss}
    A $(\thold, \totalParties, \wfunc)$-threshold secret sharing of a secret
    $x$ consists of $n$ shares $x_1, \dots, x_{\totalParties}$, each associated
    with a weight $\weight_1, \dots, \weight_\totalParties$, such that an
    efficient algorithm exists, that takes as input a set of shares
    $\tholdset$, with $\sum_{i \in \tholdset} \weight_i > \thold$, and
    outputs the secret value $x$. Any set of shares $\tholdset$
    with $\sum_{i \in \tholdset} \weight_i \leq \thold$ cannot obtain any
    information about the secret $x$.
\end{definition}

\begin{definition}[Weighted Threshold Signature]\label{def:thresh-sig}
    Given a signature scheme $\sigscheme$, a $(\thold, \totalParties,
    \wfunc)$-threshold signature scheme $\threshsigscheme = \langle \threshkeygen, \threshsign, \threshver \rangle$, given $\totalParties$ parties
    $\p_1, \dots, \p_{\totalParties} \in \parties$, is defined as:
    \begin{itemize}
        \item $\threshkeygen(1^\secparam, \wfunc) \rightarrow (\keyverify,
            \keysign_1, \dots, \keysign_n)$:
            given the security parameter $\secparam$, outputs a public key
            $\keyverify$ and a list of private keys $\keyset = [\keysign_{1}, \dots,
            \keysign_{\totalParties}]$ which form a $(\thold, \totalParties,
            \wfunc)$-threshold secret sharing of $\keysign$; the pair
            $(\keyverify, \keysign)$ has the same distribution as the keys
            output by $\algokeygen$ of Definition~\ref{def:digsign};

        \item $\threshsign(\mesg, \tholdset) \rightarrow \signature$:
            given a message $\mesg$ and a set of private keys
            $\tholdset$, $\tholdset \subseteq \keyset$, outputs a
            signature $\signature$;

        \item $\threshver(\mesg, \keyverify, \signature) \rightarrow \{0,1\}$: a
            deterministic algorithm that, given a message $\mesg$, a public key
            $\keyverify$, and a signature $\signature$ outputs $1$ if a
            signature is valid \wrt message $m$ and verification key
            $\keyverify$ (resp. $0$ if the signature is invalid).
    \end{itemize}

    A $(\thold, \totalParties, \wfunc)$-threshold signature scheme
    $\threshsigscheme$ is $\eufcma$ if it presents the properties of
    Definition~\ref{def:digsign} and the following:

    {\bf Threshold Completeness:} For any message $\mesg$, it holds:
    \begin{align}
        \Pr[(\keyverify, \keyset) \leftarrow \threshkeygen(1^\secparam, \wfunc), \signature \leftarrow \threshsign(\mesg, \tholdset),  \nonumber \\
        \sum_{k \in \tholdset} \weight_k > \thold: 0 \leftarrow \algover(\mesg, \signature, \keyverify)]
        \leq \negl(\secparam) \nonumber
    \end{align}
    and
    \begin{align}
        \Pr[(\keyverify, \keyset) \leftarrow \threshkeygen(1^\secparam, \wfunc), \signature \leftarrow \threshsign(\mesg, \tholdset),  \nonumber \\
        \sum_{k \in \tholdset} \weight_k \leq \thold: 1 \leftarrow \algover(\mesg, \signature, \keyverify)]
        \leq \negl(\secparam) \nonumber
    \end{align}
    where all the probabilities are computed over the random coins of the
    key generation and sign algorithms.
\end{definition}

\subsection{Transactions, Blocks, and the Global Ledger}\label{sec:global-ledger}
Our protocol utilizes two features of the Kachina~\cite{kachina} framework: i)
the formalization of the ledger and ii) smart contracts.

\paragraph{The Simple Ledger Functionality}
Our collective pool protocol interacts with a ledger functionality in a hybrid
execution.  One option is the formalization of Bitcoin's ledger from~\cite{C:BMTZ17}.
However, this functionality is local, while we would prefer a global
functionality, following the Global UC Framework~\cite{TCC:CDPW07}, hence we
will use the $\Gledger$ functionality from Kachina~\cite{kachina}. The
functionality is available in Figure~\ref{fig:gledger}, where $\prec$ defines
the prefix operation, \ie $\contractstate\prec\contractstate^\prime$ means the
state $\contractstate$ is included in $\contractstate^\prime$, and, for
readability and consistency purposes, we rename \emph{transaction} ($\tau$) to
\emph{block} ($b$).

\myhalfbox{Global Ledger Functionality $\Gledger$}{white!40}{white!10}{
    The functionality keeps a state $\contractstate$ and a mapping $M$ of parties to
    states, both initially empty.

    \begin{itemize}
        \item When receiving a message $(\mathsf{SUBMIT}, b)$ from a party
            $p$, query $\adversary$ with $(\mathsf{BLOCK}, b)$.

        \item When receiving a message $\mathsf{READ}$ from a party $p$,
            return $M(p)$; if $p$ is $\adversary$, it returns $\contractstate$.

        \item When receiving a message $(\mathsf{EXTEND}, \contractstate^\prime)$ from
            $\adversary$, set $\contractstate \leftarrow \contractstate || \contractstate'$.

        \item When receiving a message $(\mathsf{ADVANCE}, p, \contractstate^\prime)$ from
            $\adversary$, if $M(p) \prec \contractstate^\prime \prec \contractstate$ then set $M(p)
            \leftarrow \contractstate^\prime$.
    \end{itemize}

}{\label{fig:gledger} The Simple Global Ledger Ideal Functionality.}

The $\Gledger$ Functionality is generic enough to abstract transactions and
blocks, focusing on the ledger's properties. However, in our setting we need to
define these objects, in order to better formulate a real-world blockchain.

In this chapter, a transaction is $\tx = \langle \addr_s,
\addr_r, v, f \rangle$, where
\begin{inparaenum}[i)]
    \item $\addr_s, \addr_r \in \{0, 1\}^*$ are the sender's and receiver's
        addresses respectively,
    \item $v \in \mathbb{R}$ is the value transferred from $\addr_s$ to
        $\addr_r$, and
    \item $f \in \mathbb{R}$ is the fees of the transaction.
\end{inparaenum}
A block consists of an ordered list of transactions. To organize transaction in
blocks, we assume a function $\blockify$ which, given a set of transactions and
a chain, returns a block which can extend the chain, \ie satisfies the validity
requirements of the system.

\paragraph{Reward Management via Smart Contracts}
We also employ the formal model of smart contracts from~\cite{kachina}. This
model considers smart contracts from a privacy-preserving perspective.
However, it also provides a UC definition of standard smart contracts,
consisting of the universal machine $\mc{U}$, which acts as the oracle over the
ledger's state, and the \emph{core} contract $\Gamma$, as illustrated by
Figure~\ref{fig:rewardFunc} adapted to our stake pool design. In our setting,
the latter relates directly to the management of the rewards for the members of
the pool, and therefore it is presented as an auxiliary (reward) functionality
$\rewardFunc$ in Section~\ref{sec:management}.

\paragraph{Delegation and Stake Pools}\label{sec:delegation}
We utilize the UC Model for delegated PoS systems, as presented in
Chapter~\ref{chap:delegation}. This framework partially fulfills our earlier
desiderata. In particular, \emph{Prevention of Double Stake Allocation} and
\emph{Public Verifiabilitty} are addressed by the certificate-based
registration and revocation mechanisms.  However, the remaining items do not
seem immediately solved without further assumptions. For instance, if the
members have the same proportion of shares, a standard threshold signature
scheme could address more of our desiderata, \eg \emph{Offline and Online
Participation}, \emph{Pool Proportional Rewards}, \emph{Joint Control of
Rewards}, and \emph{Robustness against Aborting}.  Following, reconfiguration
of the pool is accomplished by regenerating the registration certificate and
the pool's threshold key. Other desiderata can be approached in a similar
fashion. Thus, our idea is to generalize the access structure of an efficient
threshold signature scheme to add ``weight'' capabilities, such that the
weights capture the pledged stake distribution among the pool's members.
