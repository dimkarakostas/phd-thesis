\section{Desiderata}\label{sec:collective-pool-desiderata}

Our design assumes a group of stakeholders who jointly create a stake pool
without a single operator. Since large stakeholders typically form pools on
their own, our protocol concerns smaller stakeholders, who could otherwise not
participate directly. Therefore, our design could \eg be appealing to a group
of friends or colleagues, who aim for a more steady reward ratio without
relying on a third party. Importantly, it should operate in a trustless environment as,
unfortunately, even in these scenarios, trust is not
a given. Notably, our targeted audience is parties who wish to
actively participate, \ie always be online to perform the required consensus
actions; parties who wish to remain offline may instead opt for delegation
schemes~\cite{eosWhitepaper,SCN:KarKiaLar20}.

In the absence of a central party, the responsibility of running the pool is
shared among all pool's members, requiring some level of coordination which may
be cumbersome. For instance, if the protocol requires unanimous actions, a
single member could halt the pool's operation. To ensure good performance, the
pool should allow a subset (of a carefully chosen size) to act on behalf of the
whole group. The choice of such subsets depends on each party's ``weight'',
which is in proportion to their stake.  In summary, we have the following
initial assumptions, which form the basis for outlining our work's desiderata:

\begin{itemize}[noitemsep]
    \item \emph{small number of parties}: a collective pool is operated by a
        small group of players;

    \item \emph{small stake disparity}: the profiles of the collective pool's
        members are similar, \ie they contribute a similar amount of stake to
        the pool;

    \item \emph{stake proportion as ``weight''}: each party is assigned a
        weight for participating in the pool's actions, relative to their part
        of the pool's total stake.
\end{itemize}

Next, we provide an exhaustive list of basic requirements of a collective stake
pool. We note that an \emph{admissible party set} is a set of parties with
enough stake, \ie above a threshold of the total pool's stake which is agreed
upon during the pool's initialization. To the extent that some desiderata are
conflicting, our design will aim to satisfy as many requirements as possible:
\begin{itemize}[noitemsep]
    \item \emph{Proportional Rewards}: the claim of each member on the
        entire pool's protocol rewards should be proportional to
        their individual contribution.

    \item \emph{Joint Control of Rewards}: the members of a pool should
        jointly control the access to its funds.

    \item \emph{Unilateral Reward Withdrawal}: at any point in time, a
        stakeholder should be able to claim their reward,
        accumulated up to that point, without necessarily interacting with
        other members of the pool.

    \item \emph{Permissioned Access}: new users can join the pool following
        agreement by an admissible set of pool members.

    \item \emph{Robustness against Aborting}: the pool should not fail to
        participate in consensus, unless an admissible set of members aborts or
        is corrupted.

    \item \emph{Public Verifiability}: stake pool formation and operation
        should be publicly verifiable (\st consensus could take into account
        the aggregate pool's stake).

    \item \emph{Stake Reallocation}: users should freely change their
        personal stake allocated to the pool, without interacting with other
        members of the pool.

    \item \emph{Parameter Updates}: an admissible set of parties should be
        able to update the stake pool's parameters.

    \item \emph{Force Removal}: an admissible set of parties should be able
        to remove a member from the pool.

    \item \emph{Pool Closing}: an admissible set of parties should be able to
        permanently close the stake pool.

    \item \emph{Prevention of Double Stake Allocation}: a party should not
        simultaneously commit the same stake to two different stake pools.
\end{itemize}
