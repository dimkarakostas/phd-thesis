Both PoW and PoS ledgers are economies of scale, favoring parties with large
amounts of participating power. One reason is poorly-designed incentives,
resulting in disproportionate power
accumulation~\cite{karakostas2019cryptocurrency,FC:FKORVW19}. Another is
temporal discounting, \ie the tendency to disfavor rare or delayed
rewards~\cite{Reed2011}. Specifically, in Bitcoin, a party is rewarded for
every block it produces, so parties with insignificant amounts of power are
rarely rewarded. In contrast, accumulating the power of multiple small parties
in ``pools'' yields a steadier reward.

As a result, these systems often see the formation of collaborative entities of
participants. In PoW systems, this takes the form of mining
pools.\footnote{$86$\% of Bitcoin's hashing power and $83$\% of Ethereum's
hashing power are controlled by $5$ entities each.
(\url{https://miningpools.com}; May 2021)} Similarly, PoS systems often opt for
stake pools, \ie collaborative entities comprising of multiple stakeholders,
which allow a party to earn rewards more regularly, compared to participating
on an individual basis.  Particularly in PoS, delegation to stake pools is
often preferred over ``pure'' PoS, where parties act independently, as the
ledger's performance and security is often better under fewer participants. For
instance, PoS systems require participants to be constantly online, since
abstaining is a security hazard; this requirement is more easily guaranteed
within a small set of dedicated delegates.

However, a major drawback of existing stake pool designs, including our scheme
of Chapter~\ref{chap:delegation}, is that they are typically managed by a
single party, \ie the pool operator. This party participates in consensus,
claims the rewards offered by the system, and then distributes them among the
pool's members (after subtracting a fee). However, the operator is a single
point of failure. In this chapter, we extend the results of
Chapter~\ref{chap:delegation} by exploring a design which allows players to
jointly form a \emph{collective pool}, \ie a conclave. This design assumes no
single operator, minimizing excess fees, and trust and security concerns,
altogether. Collective stake pools also promote a more fair and decentralized
environment. In existing incentive
schemes~\cite{DBLP:conf/eurosp/BrunjesKKS20}, operators who can pledge large
amounts of stake to the pool are preferred. Consequently, the system favors a
few major pool operators and, in the long run, its wealth is concentrated
around them, resulting in a ``rich get richer'' situation. Although this
problem is inherent in all decentralized financial
systems~\cite{karakostas2019cryptocurrency}, a well-designed collective pool
may offset the stakeholder imbalance and slightly decelerate this tendency.
Especially in PoS systems, a well-designed pool mechanism can prevent attacks
observed on PoW~\cite{FCW:JLGVM14,FC:WalCli14,FCW:LasJohGro15}.
