\section{Incentives Analysis}\label{sec:incentives}

Although, as shown in Theorem~\ref{thm:collective-pool}, $\Ppool$ is secure, it
is unclear whether rational users will opt for using it. In this section, we
discuss the incentive compatibility of $\Ppool$. We identify its shortcomings
and propose a minor change, such that rational members cannot gain more rewards
by deviating from it.

First, we consider the cost of each operation performed in $\Ppool$. Signing
operations does not incur any cost, thus pool registration and revocation are
cost-free. Block production depends on the internal workings of $\blockify$.
For instance, solving the Knapsack problem can be expensive, while a greedy
algorithm that prioritizes high-fee transactions is typically not. Therefore,
without loss of generality, we also assume that block production is cost-free.
However, both mempool update and transaction verification incur costs
$\cost_{mu}$ and $\cost_{tv}$ respectively. A mempool consists of millions of
transactions and verifying them requires an accurate view of the
ledger. Thus, both objects may require significant amounts of computations
and storage.\footnote{As of January 2021, the Bitcoin chain is
roughly $320$GB and increases linearly over time.
(\url{https://www.blockchain.com/charts/blocks-size})}

We focus on the \emph{profit} of each member, \ie the rewards subtracted by
the cost of executing $\Ppool$. The core observation is that a member $\party$
receives $\reward_{\party} = \frac{\weight_{\party}}{\sum_{\party' \in \wfunc}
\weight_{\party'}}$ of the total pool's rewards \emph{regardless of its
performance}. For instance, if $\party$ acts only on the pool's
creation, it still receives its proportional share of rewards for the blocks
produced by the rest of the pool. Therefore, as long as a member believes that
the other members act honestly, it is incentivized to abstain and minimize its
cost, thus maximizing its profit. Naturally, if all parties follow this
strategy, the pool produces no blocks and receives no rewards.

A possible solution to the Free Rider problem above is to penalize a party
for misbehaving.  However, identifying misbehavior is not straightforward. For
instance, a party $\party$ who inputs $0$ to $\Pconsensus$ for a transaction
$\tx$ may do so either because the transaction is invalid or because it
didn't perform validation and input $0$ by default. Our approach is to
penalize a party when diverging from the rest of the pool. In the previous
example, if the output of $\Pconsensus$ is $1$, then $\party$ incurs a fixed
penalty $P_{tv}$. Similarly, if a party fails to sign a new block
then it incurs a (fixed) penalty $P_{mu}$. The penalty amount,
which is withheld from $\party$, is then distributed equally among the other
pool members.  To incentivize $\party$ to follow $\Ppool$ the penalties should
be high enough; specifically, it should hold $P_{tv} >
\cost_{tv}$ and $P_{mu} > \cost_{mu}$.  If all parties follow
$\Ppool$, diverting incurs a cost $P_{mu} - \cost_{mu} > 0$
(resp. $P_{tv} - \cost_{tv} > 0$), thus the new protocol is an
equilibrium.

Finally, penalties can be automatically enforced via an interface to the smart
contract $\rewardFunc$ which, given a proof of misbehavior, reduces the
misbehaving party's rewards accordingly. For transaction verification, a proof
of misbehavior is the transcript of $\Pconsensus$, which describes the
consensus sub-protocol's execution. For block issuing, we can use a threshold
signature scheme with identifiable aborts~\cite{EPRINT:GenGol20,CCS:CGGMP20},
which allows to identify the parties that do not participate in the signing of
a block.
