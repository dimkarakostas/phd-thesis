\subsection{Part 1: Stake Pool Management}\label{sec:management}

The functionality's first part (Figure~\ref{fig:Fpool-1}) includes all
operations that are not consensus-oriented. First, establishing a stake pool
consists of two parts, defined as corresponding interfaces in the ideal
functionality. The pool's members gather and jointly decide to create a staking
pool; they contact each other, \eg via off-chain direct channels, agree on the
pool's parameters, and generate its key. Importantly, the participants are
aware of the total number of participants in the pool, as well as their
weights. Then, the members of the pool perform a setup protocol and register
the new pool via a registration certificate, which is signed by the pool's key
and published on the ledger. Following, the pool receives rewards for
participating in the consensus protocol. The rewards are managed by a smart
contract and, at any point, each each party can withdraw their part, which is
proportional to the internal stake distribution. Finally, to close the pool,
the members sign and publish a revocation certificate.

In more detail, the functionality $\Fpool$ interacts with $\totalParties$
parties $\party_1, \dots, \party_\totalParties$ and is parameterized by:
\begin{itemize}
    \item the validation predicate $\algovalidate(\cdot, \cdot)$ which, given a
        transaction $\tx$ and a chain $\chain$, defines whether $\tx$ can be
        appended to $\chain$ (as part of a block);
    \item the algorithm $\blockify$ which, given a set of transactions,
        serializes them (deterministically) in a block;
    \item the probability $\prob^{\subselectionParam, \adversarialParties,
        \totalParties}$ that the elected committee, responsible for a
        transaction's verification, is corrupted, dependent on the subselection
        parameter $\subselectionParam$ and the number of corrupted parties
        $\adversarialParties$ out of $\totalParties$ total parties.
\end{itemize}

It also keeps the following, initially empty, variables:
\begin{inparaenum}[i)]
    \item the signature threshold $\thold$;
    \item the public key $\threshpubkey_{pool}$;
    \item the reward address $\addr_{reward}$;
    \item the set of valid and unpublished transactions $\mathsf{mempool}$;
    \item a mapping of parties to weights $\members$;
    \item a table of signatures $\mathsf{sigs}$.
\end{inparaenum}

\paragraph{Gathering and Registration}
The first step in creating a pool is the gathering of parties, in order to
collectively create the pool's public key $\threshpubkey_{pool}$. Following,
the parties create and publish on the ledger the registration certificate
$\cert_{reg}$, which contains the following:
\begin{itemize}[noitemsep]
    \item $\wfunc$: a mapping identifying each member's weight;
    \item $\addr_{reward}$: the address which accumulates the pool's rewards;
    \item $\threshpubkey_{pool}$: the pool's threshold public key;
    \item $\threshsig_{pool}$: the signature of $\langle \wfunc, \addr_{reward}
        \rangle$ created by $\threshpubkey_{pool}$.
\end{itemize}

\paragraph{Reward Withdrawal}
During the life cycle of the pool, a member may want to withdraw the rewards
received up to that point. As per the desiderata of
Section~\ref{sec:collective-pool-desiderata}, any party should be able to do so, without the
explicit permission of the other pool's members. Additionally, the rewards that
each party receives should be proportional to its stake, \ie its weight within
the collective pool. Reward withdrawal is implemented as the smart contract
functionality $\rewardFunc$. The contract is initialized with the weight
distribution of the pool's members and each member's public key. We assume that
the contract is associated with an address and can receive funds,
similar to real-world smart contract systems like
Ethereum~\cite{wood2014ethereum}. The state transition functionality
$\rewardFunc$ is defined in Figure~\ref{fig:rewardFunc} (following the ledger
formalization of Section~\ref{sec:global-ledger}).

\myhalfbox{Reward Smart Contract Functionality $\rewardFunc$}{white!40}{white!10}{
    $\rewardFunc$ maintains a mapping $\wfunc$, of parties to weights,
    and a variable $b$.

    \noindent{\bf Initialization:}
        Upon receiving $\msg{init}{\wfunc^\prime}$, forward it to $\iadv$. Upon receiving
        a response $\msg{init-ok}{\addr_{sc}}$, set $\wfunc \leftarrow \wfunc^\prime$ and
        return $\msg{init-ok}{\addr_{sc}}$.

    \noindent{\bf Balance Update:}
        On receiving $\msg{transaction}{\tx}$ from $\mc{U}$, such that $\tx =
        \langle \addr_s, \addr_r, v, f \rangle $, if $\addr_s = \addr_{sc}$ set
        $b := b - v$, else if $\addr_r = \addr_{sc}$ set $b := b + v$.

    \noindent{\bf Withdrawal:}
        Upon receiving $\msg{withdraw}{\addr, f}$ from the party $\p$, set $r =
        \frac{\weight_p}{\sum_{p^\prime \in \wfunc} \weight_{p^\prime}} \cdot b$ and return
        $\msg{transaction}{\langle \addr_{sc}, \addr, r, f \rangle}$.

}{\label{fig:rewardFunc} The pool's Reward Smart Contract Functionality.}

\paragraph{Closing}
Eventually, the members halt the operation of the pool. In order to do so, they
revoke the pool's registration by jointly producing a revocation certificate
$\cert_{rev}$. The certificate is relatively simple, containing a timestamp $x$
announcing the end of the pool and signed by the pool's public key
$\threshpubkey_{pool}$.

The first part of our functionality definition is given by
Figure~\ref{fig:Fpool-1}, whereas the management routines, \ie the first part
of the description, of our protocol construction is given by
Figure~\ref{fig:Ppool-1}.

\myhalfbox{Collective Pool Functionality $\Fpool^{\thold,\wfunc}$ (first part)}{white!40}{white!10}{
    \noindent{\bf Gathering:}
        Upon receiving $\msg{gather}{}$ from $\p$, forward it to $\iadv$. After
        every party $\p_i, i \in [1, \totalParties]$ has submitted
        $\mathsf{gather}$, upon receiving from $\iadv$
        $\msg{gather-ok}{\threshpubkey_{pool}}$, store $\thold$ and
        $\threshpubkey_{pool}$, add all party-weight pairs $(\p_{i}, \wfunc_i)$
        to $\members$, and reply with
        $\msg{gather-ok}{\threshpubkey_{pool}}$ to all parties.

    \noindent{\bf Pool Registration:}
        Upon receiving $\msg{register}{\members}$ from $\p$, forward it
        to $\iadv$. After all parties $\p_i, i \in [1, \totalParties]$ have
        submitted $\mathsf{register}$, upon receiving from $\iadv$
        $\msg{register-ok}{\addr_{reward}, \threshsig_{pool}}$, set $\cert_{reg}
        = \langle (\members, \addr_{reward}, \threshpubkey_{pool},
        \threshsig_{pool}) \rangle$. Then check if $\forall (m, \sigma, b') \in
        \mathsf{sigs}: \sigma \neq \threshsig_{pool}, (\cert_{reg},
        \threshsig_{pool}, 0) \not \in \mathsf{sigs}$; if the checks hold,
        insert $(\cert_{reg}, \threshsig_{pool}, 1)$ to $\mathsf{sigs}$.
        Finally, store $\addr_{reward}$ and reply with
        $\msg{register-ok}{\cert_{reg}}$.

    \noindent{\bf Reward Withdrawal:}
        Upon receiving the message $\msg{withdraw}{\addr, f}$ from $\p_{i}$,
        forward it to $\iadv$. Then, compute $\reward =
        \frac{\weight_{\p_i}}{\sum_{j = 1}^{\totalParties} \weight_{\p_j}}
        \cdot \reward_{pool}$, where $\reward_{pool}$ is the funds of address
        $\addr_{sc}$ as defined in $\Gledger$. Finally, return
        $\msg{transaction}{\langle \addr_{sc}, \addr, \reward, f \rangle}$.

    \noindent{\bf Closing:}
        Upon receiving $\msg{close}{x}$ from $\p$, forward it to $\iadv$. After
        a set of parties $\tholdset$ has submitted $\mathsf{close}$ for the
        same $x$, if $\sum_{\p \in \tholdset} \weight_{\p} > \thold$, upon
        receiving $\msg{close-ok}{\threshsig_{pool}}$ from $\iadv$, check if
        $\forall (m, \sigma, b') \in \mathsf{sigs}: \sigma \neq
        \threshsig_{pool}, (x, \threshsig_{pool}, 0) \not \in \mathsf{sigs}$;
        if the checks hold, insert $(x, \threshsig_{pool}, 1)$ to
        $\mathsf{sigs}$. Finally, return to all parties
        $\msg{close-ok}{\cert_{rev}}$, with $\cert_{rev} = \langle x,
        \threshsig_{pool} \rangle$.

}{\label{fig:Fpool-1} The first part of the Collective Pool Functionality, parameterized with threshold $\thold$ and weight mapping $\wfunc$, refers to the creation and management of the pool (the second part is given by Figure~\ref{fig:Fpool-2}).}

\myhalfbox{Collective Pool Protocol $\Ppool^{\thold,\wfunc}$ (first part)}{white!40}{white!10}{
    \noindent{\bf Gathering:}
        Upon receiving $\msg{gather}{}$, send $\msg{KeyGen}{}$ to $\Fweight$,
        with $sid$ containing the weight mapping $\wfunc$ and the threshold
        $\thold$. Upon receiving the reply
        $\msg{KeyGen}{\threshpubkey_{pool}}$, return
        $\msg{gather-ok}{\threshpubkey_{pool}}$.

    \noindent{\bf Pool Registration:}
        Upon receiving $\msg{register}{\members}$, send
        $\msg{init}{\members}$ to $\rewardFunc$ and wait for the reply
        $\msg{init-ok}{\addr_{reward}}$. Then, set $m = (\members,
        \addr_{reward})$ and send $\msg{Sign}{m}$ to $\Fweight$. Upon receiving a
        reply $\msg{Sign}{m, \threshsig_{pool}}$, return
        $\msg{register-ok}{\cert_{reg}}$, where $\cert_{reg} = \langle
        (\members, \addr_{reward}, \threshpubkey_{pool},
        \threshsig_{pool}) \rangle$.

    \noindent{\bf Reward Withdrawal:}
        Upon receiving $\msg{withdraw}{\addr, f}$, forward it to $\rewardFunc$.
        Upon receiving a response $\msg{transaction}{\langle \addr_{sc}, \addr,
        r, f \rangle}$ return it.

    \noindent{\bf Closing:}
        Upon receiving $\msg{close}{x}$, send $\msg{Sign}{x}$ to $\Fweight$.
        Upon receiving a reply $\msg{Sign}{x, \threshsig_{pool}}$, return
        $\msg{close-ok}{\cert_{rev}}$ with $\cert_{rev} = \langle x,
        \threshsig_{pool} \rangle$.

}{\label{fig:Ppool-1} The first part of the Collective Pool Protocol, which describes the set of management operations (the second part is given by Figure~\ref{fig:Ppool-2}).}
