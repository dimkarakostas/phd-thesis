\section{The Collective Stake Pool}\label{sec:stake-pool}

Our analysis is based on the UC Framework, following Canetti's formulation of
the ``real world''~\cite{EPRINT:Canetti00}. Specifically, we define the
collective pool ideal functionality $\Fpool$, which distills the required
(operational and security) properties; for readability, $\Fpool$ is divided in
two parts, \emph{management} and \emph{consensus participation}. The ideal
functionality is realized --- in the ``real world'' --- by the distributed
protocol $\Ppool$, which employs various established cryptographic primitives,
and, therefore, $\Ppool$ can described with auxiliary functionalities.
Before proceeding with the functionality's definition, we first describe the
hybrid execution of $\Ppool$ and its building blocks.

\subsection{Hybrid Protocol Execution}\label{sec:execution}

The protocol $\Ppool$ is performed by $n$ parties, where each party $\p_i$
holds two pairs of keys: $(\keyverify_{\p_{i}}, \keysign_{\p_{i}})$ for issuing
transactions, and $(\keyverify_{s_{i}}, \keysign_{s_{i}})$ for staking
operations, \eg issuing delegation certificates (cf.~\cite{SCN:KarKiaLar20}). The
public key $\keyverify_{i}$ is also used to generate an address $\addr_{i}$.
Each pool member $\p_i$ pledges the funds of an address $\addr_{i}$ (which it
owns) to the pool. These funds are the player's stake in the pool and form
the player's weight in the weight distribution mapping $\wfunc$.

We assume the members' stake, \ie their weight $\weight_{i}$ in the pool, is
public. Therefore, the weight distribution mapping $\wfunc$ is also public.
Furthermore, each member of the pool has its own signature key, and can issue
standard signatures through a standard signature scheme.  A
weighted version for a threshold signature scheme follows by having each party
holding as many shares, of the original threshold scheme, as its weight. This
approach has the extra advantage that security guarantees of the original
scheme are carried straightforwardly into the weighted version.

Additionally, our construction relies on the consensus sub-protocol $\Pconsensus$
to validate a transaction by the elected committee.
Specifically, the collective stake pool protocol is
parameterized by:
\begin{inparaenum}[i)]
    \item the validation predicate $\algovalidate$,
    \item the permutation algorithm $\permutationAlgo$, and
    \item a consensus sub-protocol $\Pconsensus$.
\end{inparaenum}

Our (modular) protocol is described in a hybrid world with auxiliary
functionalities for established primitives. The functionality
$\Fbroad$~\cite{EC:HirZik10} provides a reliable broadcast channel to all parties;
$\Fcore$ (cf. Chapter~\ref{chap:delegation}) enables delegation to the pool;
$\Fweight$ (cf. Section~\ref{sec:uc-weighted-tss}) enables weighted threshold
signature operations; the Smart Contract Functionality $\rewardFunc$ realizes
the reward distribution mechanism; $\Gledger$ is a global Ledger
Functionality~\cite{kachina}. Finally, we use $\hybpool$ to denote the $\{
    \Gledger, \Fbroad, \Fcore, \Fweight, \rewardFunc \}$-hybrid execution of
$\Ppool$ in the (global) UC Framework.
