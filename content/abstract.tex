\begin{floating-abstract}
    Distributed ledgers rose to prominence with the advent of Bitcoin, the
    first provably secure protocol to solve consensus in an
    open-participation setting. Following, active research and engineering
    efforts have proposed a multitude of applications and alternative designs,
    the most prominent being Proof-of-Stake (PoS). This thesis expands the
    scope of secure and efficient asset management over a distributed ledger
    around three axes:
    \begin{inparaenum}[i)]
        \item cryptography;
        \item distributed systems;
        \item game theory and economics.
    \end{inparaenum}
    % i) cryptography; ii) distributed ledgers; iii) game theory and economics.

    First, we analyze the security of various wallets. We start with a
    formal model of hardware wallets, followed by an analytical
    framework of PoS wallets, each outlining the unique properties of
    Proof-of-Work (PoW) and PoS respectively. The latter also provides a
    rigorous design to form collaborative participating entities, called
    stake pools. We then propose Conclave, a stake pool design which enables a
    group of parties to participate in a PoS system in a collaborative manner,
    without a central operator.

    Second, we focus on efficiency. Decentralized systems are aimed at
    thousands of users across the globe, so a rigorous design for minimizing
    memory and storage consumption is a prerequisite for scalability.
    To that end, we frame ledger maintenance as an optimization problem and
    design a multi-tier framework for designing wallets which ensure that
    updates increase the ledger's global state only to a minimal extent, while
    preserving the security guarantees outlined in the security analysis.

    Third, we explore incentive-compatibility and analyze blockchain systems
    from a micro and a macroeconomic perspective. We enrich our cryptographic
    and systems' results by analyzing the incentives of collective pools and
    designing a state efficient Bitcoin fee function. We then analyze the Nash
    dynamics of distributed ledgers, introducing a formal model that evaluates
    whether rational, utility-maximizing participants are disincentivized from
    exhibiting undesirable infractions, and highlighting the differences
    between PoW and PoS-based ledgers, both in a standalone setting and under
    external parameters, like market price fluctuations. We conclude by introducing a
    macroeconomic principle, cryptocurrency egalitarianism, and then describing
    two mechanisms for enabling taxation in blockchain-based currency systems.
\end{floating-abstract}
