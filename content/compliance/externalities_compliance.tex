\subsection{Compliance under Externalities}

We now revisit our results, taking externalities into account. The core idea is
to find the relation between the assets' price and external, infraction-based
rewards, such that the former counters the latter, hence parties are
incentivized to remain compliant.

In systems that are compliant in the standalone setting, \ie without
considering externalities, it suffices to show that the exchange rate is
reduced to an extent which counterbalances the external rewards. We consider
two strategy profiles:
\begin{inparaenum}[i)]
    \item $\profile_\proto$ is the profile under which all parties follow the
        protocol $\proto$;
    \item $\profile_{\strategy_{\party}}$ is the profile under which all parties except
        $\party$ follow $\proto$, whereas $\party$ deviates by following a non-compliant strategy $\strategy_{\party}$; for this
        deviation, $\party$ receives external rewards
        $\utilityBoostVal_{\party, \profile_{\strategy_{\party}}}$.
\end{inparaenum}
Assume a system that is $(\epsilon, \infractionPredicate)$-compliant; under
externalities, the system remains compliant, for the same $\epsilon$, as long
as it holds:
$\forall \party \; \forall \strategy_{\party}: (\rewardVal_{\party, \profile_{\strategy_{\party}}} \cdot \exchangeRateVal_{\profile_{\strategy_{\party}}} + \utilityBoostVal_{\party, \profile_{\strategy_{\party}}}) - (\rewardVal_{\party, \profile_\proto} \cdot \exchangeRateVal_{\profile_\proto}) < \epsilon$.
In some cases, \eg under fair rewards or in the synchronous setting of
single-leader PoS (cf. Subsection~\ref{subsec:single-leader-pos}), the expected
rewards are equal under both profiles. In those cases, it holds:
$\utilityBoostVal_{\party, \profile_{\strategy_{\party}}} \leq \rewardVal_{\party, \profile_\proto} \cdot (\exchangeRateVal_{\profile_\proto} - \exchangeRateVal_{\profile_{\strategy_{\party}}}) + \epsilon$.
Therefore, if the exchange rate is sufficiently reduced, when $\party$ performs
an infraction, $\party$ is incentivized to remain compliant. More formally,
Theorem~\ref{thm:external-ouroboros} analyzes Ouroboros under a synchronous
network and externalities; similar statements can be made for the positive
results of Sections~\ref{sec:universal} and~\ref{subsec:bitcoin}.

\begin{theorem}\label{thm:external-ouroboros}
    Assume
    \begin{inparaenum}[i)]
        \item a synchronous network (cf. Section~\ref{sec:network-model}),
        \item the conflicting predicate $\infractionPredicate_{conf}$, and
        \item that $\forall \party \in \partySet: \miningpower_{\party} <$~$\frac{1}{2}$.
    \end{inparaenum}
    Also let:
    \begin{inparaenum}[i)]
        \item $\strategySet_{-\infractionPredicate_{conf}}$: the set of all non $\infractionPredicate_{conf}$-compliant strategies;
        \item $\exchangeRateVal_{\profile_\proto}$: the (expected) exchange rate under $\execution_{\profile_\proto}$;
        \item $\exchangeRateVal_{\profile_{\strategy_{\party}}}$: the (expected) exchange rate when only $\party$ employs some non $\infractionPredicate_{conf}$-compliant strategy $\strategy_{\party}$;
        \item $\utilityBoostVal_{\party, \profile_{\strategy_{\party}}}$: the external utility that $\strategy_{\party}$ yields for $\party$.
    \end{inparaenum}
    Ouroboros with block-proportional rewards (cf.
    Definition~\ref{def:proportional-rewards}, for fixed block reward $\reward$) under the aforementioned externalities is not $(\epsilon,
    \infractionPredicate_{conf})$-compliant (cf. Definition~\ref{def:compliant}) \wrt
    utility \emph{Reward} (cf. Definition~\ref{def:utility})
    and, if $\reward >
    \cost$, it is also not $(\epsilon, \infractionPredicate_{conf})$-compliant \wrt
    utility \emph{Profit}, in both cases
    for some
    $\epsilon < \mathsf{max}\{  \underset{\party \in \partySet}{\mathsf{max}}\{ \underset{\strategy_{\party} \in \strategySet_{-\infractionPredicate_{conf}}}{\mathsf{max}}\{ \rewardVal_{\party, \profile_{\proto}} \cdot (\exchangeRateVal_{\profile_{\strategy_{\party}}} - \exchangeRateVal_{\profile_\proto}) + \utilityBoostVal_{\party, \profile_{\strategy_{\party}}} \} \}, 0 \}$.
\end{theorem}
\begin{proof}
    Following the same reasoning as
    Theorem~\ref{thm:compliant-ouroboros-synchronous}, if a party $\party$
    deviates by only producing conflicting messages, but does not abstain, its
    expected rewards are the same as following the protocol; specifically, due
    to network synchronicity, after every round when $\party$ is leader, every
    other party adopts one of the blocks produced by $\party$ (although
    possibly not everybody adopts the same block), and, since all these blocks
    are part of the (equally-long) longest chain (at that point), eventually
    one of these blocks will be output in the chain of the observer.
    Consequently, it holds that $\rewardVal_{\party,
    \profile_{\strategy_{\party}}} = \rewardVal_{\party, \profile_{\proto}}$.

    Second, the maximum additional utility that a party $\party$ may receive by
    deviating from the honest protocol via producing conflicting blocks is:
    $\underset{\strategy_{\party} \in \strategySet_{-\infractionPredicate_{conf}}}{\mathsf{max}}\{ \rewardVal_{\party, \profile_{\proto}} \cdot (\exchangeRateVal_{\profile_{\strategy_{\party}}} - \exchangeRateVal_{\profile_\proto}) + \utilityBoostVal_{\party, \profile_{\strategy_{\party}}} \}$.
    Therefore, if for at least one party this value is non-negligible,
    $\epsilon$ is not small enough and so the protocol is not compliant.
\end{proof}

In the above negative results, non-compliance arises in
\begin{inparaenum}[(a)]
    \item systems that employ fair rewards and
    \item PoS systems where a party is incentivized to produce multiple
        conflicting messages, \ie under a lossy network or multiple leaders per
        slot.
\end{inparaenum}

Regarding (a), Section~\ref{sec:universal} shows that fair rewards ensure
compliance under utility \emph{Reward}, but non-compliance regarding profit.
Specifically, assuming a minimal participation cost $\cost^\bot_{\party}$, we
showed that, if $\party$ abstains, they incur zero cost without any reward
reduction. To explore compliance of fair rewards under externalities, we
consider two strategy profiles $\profile_\proto, \profile_{\strategy_{\party}}$,
as before. Notably, $\strategy_{\party}$ is the abstaining strategy which, as shown in Section~\ref{sec:universal}, maximizes utility in the standalone setting.
For the two profiles, the profit for $\party$ becomes
$\rewardVal_{\party, \profile_\proto} \cdot \exchangeRateVal_{\profile_\proto} - \costVal_{\party, \profile_\proto}$ and
$\rewardVal_{\party, \profile_{\strategy_{\party}}} \cdot \exchangeRateVal_{\profile_{\strategy_{\party}}} + \utilityBoostVal_{\party, \profile_{\strategy_{\party}}}$ respectively.
Again, in both cases the party's rewards are equal.
Therefore, since it
holds that $\cost^\bot_{\party} \leq \costVal_{\party, \profile_\proto}$,
$\party$ is incentivized to be $(\epsilon, \infractionPredicate_{conf})$-compliant (for some $\epsilon$) if:
\begin{align}
    \rewardVal_{\party, \profile_{\strategy_{\party}}} \cdot \exchangeRateVal_{\profile_{\strategy_{\party}}} + \utilityBoostVal_{\party, \profile_{\strategy_{\party}}} \leq \rewardVal_{\party, \profile} \cdot \exchangeRateVal_{\profile_\proto} - \costVal_{\party, \profile_\proto} + \epsilon \Rightarrow \nonumber \\
    \costVal_{\party, \profile_\proto} + \utilityBoostVal_{\party, \profile_{\strategy_{\party}}} \leq \rewardVal_{\party, \profile_\proto} \cdot (\exchangeRateVal_{\profile_\proto} - \exchangeRateVal_{\profile_{\strategy_{\party}}}) + \epsilon \Rightarrow \nonumber \\
    \cost^\bot_{\party} + \utilityBoostVal_{\party, \profile_{\strategy_{\party}}} \leq \rewardVal_{\party, \profile_\proto} \cdot (\exchangeRateVal_{\profile_\proto} - \exchangeRateVal_{\profile_{\strategy_{\party}}}) + \epsilon
\end{align}
If the abstaining strategy yields no external rewards, as is typically the
case, $\utilityBoostVal_{\party, \profile_{\strategy_{\party}}} = 0$, so the
exchange rate needs to only counterbalance the minimal participation cost.

Regarding (b), we consider single-leader PoS under a lossy network, since the
analysis is similar for multi-leader PoS. We again consider two strategy
profiles $\profile_\proto, \profile_{\strategy_{\party}}$ as above. Now, under
$\profile_{\strategy_{\party}}$, $\party$ produces $k$ blocks during each slot for which
it is leader, to increase the probability that at least one of them is output
in the observer's final chain. Also, for simplicity, we set
$\utilityBoostVal_{\party, \profile_{\strategy_{\party}}} = 0$. These PoS
systems become $(\epsilon', \infractionPredicate_{conf})$-compliant (for
some $\epsilon'$) if:
\begin{align}
    \rewardVal_{\party, \profile_{\strategy_{\party}}} \cdot \exchangeRateVal_{\profile_{\strategy_{\party}}} &\leq \rewardVal_{\party, \profile_\proto} \cdot \exchangeRateVal_{\profile_\proto} + \epsilon \Rightarrow \nonumber \\
    (1 - \networkLossProb^k \cdot (1 - \networkLossProb)^2) \cdot \reward_{max} \cdot \exchangeRateVal_{\profile_{\strategy_{\party}}} &\leq (1 - \networkLossProb \cdot (1 - \networkLossProb)^2) \cdot \reward_{max} \cdot \exchangeRateVal_{\profile_\proto} + \epsilon \Rightarrow \nonumber \\
    \exchangeRateVal_{\profile_{\strategy_{\party}}} &\leq \frac{1 - \networkLossProb \cdot (1 - \networkLossProb)^2}{1 - \networkLossProb^k \cdot (1 - \networkLossProb)^2} \cdot \exchangeRateVal_{\profile_\proto} + \epsilon'
\end{align}
where $\reward_{max} = \reward \cdot \sum_{i \in [1, \epoch]} \epochLength \cdot \miningpower_{\party, i}$ and $\networkLossProb, \reward, \epochLength$ are as in Subsection~\ref{subsec:multi-leader-pos}.
