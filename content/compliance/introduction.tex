With the advent of Bitcoin~\cite{nakamoto2008bitcoin} the economic aspects of
consensus protocols came to the forefront. While classical literature in
consensus primarily dealt with ``error models'', such as fail-stop or Byzantine
\cite{DBLP:journals/jacm/PeaseSL80}, the pressing question post-Bitcoin is
whether the incentives of the participants align with what the consensus
protocol asks them to do.

Motivated by this, existing literature pursued various research paths.
One line of work investigated
whether the Bitcoin protocol is an equilibrium under certain conditions
\cite{KrollDaveyFeltenWEIS2013,kiayias16EC}. Another, pinpointed protocol
deviations that can be more profitable for some players, assuming others follow
the protocol~\cite{FC:EyaSir14,FC:SapSomZoh16,FCW:JLGVM14,CCS:CKWN16}.
Some works proposed tweaks towards improving the underlying blockchain protocol
in various settings~\cite{FC:FKORVW19,koutsoupias19www}, game-theoretic studies
of pooling behavior~\cite{lewenberg15,CCS:CKWN16,ITCS:ArnWei19}, as well as
equilibria involving abstaining from the protocol~\cite{DBLP:conf/ec/FiatKKP19}
in high cost scenaria. Going beyond consensus, economic mechanisms have also
been considered in the context of multi-party
computation~\cite{CCS:KumMorBen15,FC:DavDowLar19,FC:DavDowLar18}, to
disincentivize ``cheating''.
Finally, a large body of research was dedicated to
optimizing particular attacks; respresentative works
\begin{inparaenum}[i)]
    \item identify optimal selfish mining strategies~\cite{FC:SapSomZoh16};
    \item propose a framework~\cite{CCS:GKWGRC16} for quantitatively
    evaluating blockchain parameters and identifies optimal strategies for
    selfish mining and double-spending, taking into account network delays;
    \item propose alternative strategies~\cite{EPRINT:NKMS15}, that are more
        profitable than selfish mining.
\end{inparaenum}

Though the above works provide some glimpses on how Bitcoin and related
protocols behave from a game-theoretic perspective, they still offer very
little guidance on how to design new consensus protocols. This is a problem of
high importance, given the current negative light shed on Bitcoin's perceived
energy inefficiency and carbon footprint~\cite{martin2021energy} that
asks for alternative protocols. Proof-of-Stake (PoS) ledgers is currently the
most prominent alternative to Bitcoin's Proof-of-Work (PoW) mechanism. While
PoW requires computational effort to produce valid messages, \ie blocks which
are acceptable by the protocol, PoS relies on each party's stake, \ie the
assets they own, and each block is created at (virtually) no cost.
Interestingly, while it is proven that PoS protocols are Byzantine resilient
\cite{C:KRDO17,EPRINT:CGMV18,EPRINT:GHMVZ17,buterin2017casper} and are even
equilibriums under certain conditions \cite{C:KRDO17}, their security is
heavily contested by proponents of PoW protocols via an economic argument. In
particular, the argument termed the \emph{nothing-at-stake}
attack~\cite{li2017securing,ethereumFaq,nothing-at-stake-1} asserts that
parties who maintain a PoS ledger will opt to produce conflicting blocks,
whenever possible, to maximize their expected rewards.

What merit do these criticisms have?  Participating in a blockchain protocol is
a voluntary action that involves a participant downloading the software,
committing some resources, and running the software. Furthermore, especially
given the open source nature of these protocols, nothing prevents the
participant from modifying the behaviour of the software in some way and engage
with the other parties following a modified strategy. There are a number of
adjustments that a participant can do which are undesirable, \eg
\begin{inparaenum}[i)]
    \item run the protocol intermittently instead of continuously;
    \item not extend the most recent ledger of transactions they are aware of;
    \item extend simultaneously more than one ledger of transactions.
\end{inparaenum}
One can consider the above as fundamental {\em infractions} to the protocol
rules and they may have serious security implications, both in terms of the
consistency and the liveness of the underlying ledger.

To address these issues, many blockchain systems introduce additional
mechanisms on top of Bitcoin incentives, frequently with only rudimentary game
theoretic analysis. These include:
\begin{inparaenum}[i)]
    \item rewards for ``uncle blocks'' in Ethereum;
    \item stake delegation~\cite{eosWhitepaper}  in Eos and Polkadot, where
    users assign their participation rights to delegates, as well as stake
    pools in Cardano \cite{SCN:KarKiaLar20};
    \item penalties~\cite{buterin2017casper,casper-incentives} for misbehavior
    in Ethereum 2.0, that enforce forfeiture of large deposits (referred to as
    \emph{``slashing''}) if a party misbehaves, in the sense of being offline
    or using their cryptographic keys improperly.
\end{inparaenum}
The lack of thorough analysis of these mechanisms is of course a serious
impediment to the wider adoption of these systems.

For instance, forfeiting funds may happen inadvertently, due to server
misconfiguration or software and hardware bugs~\cite{khatri2021slashed}.
A party that employs a redundant configuration with multiple replicas, to
increase its crash-fault tolerance, may produce conflicting blocks if, due to a
faulty configuration or failover mechanism, two replicas come alive
simultaneously.  Similarly, if a party employs no failover mechanism and
experiences network connectivity issues, it may fail to participate. Finally,
software or hardware bugs can always compromise an -- otherwise safe and secure
-- configuration.  This highlights the flip side of such penalty mechanisms:
participants may choose to not engage, (\eg to avoid the risk of forfeiting
funds, or because they do not own sufficient funds to make a deposit), or, if
they do engage, they may steer clear of fault-tolerant sysadmin practices, such
as employing a failover replica, which could pose quality of service concerns
and hurt the system in the long run.

The above considerations put forth the fundamental question that motivates our
work: {\em How effective are blockchain protocol designs in disincentivizing
particular infractions?} In more detail, the question we ask is whether selfish
behavior can lead to deviations, starting with a given blockchain protocol as
the initial point of reference of honest --- compliant --- behavior.
