\subsection{Penalties}

Table~\ref{tab:attacks} shows that, historically, the attacks are profitable,
so the market's response is insufficient to keep a party compliant.
Interestingly, in many occasions the external utility was so high that, even if
the exchange rate became $0$, it would exceed the amount of lost rewards.
Therefore, an additional form of utility reduction is necessary. In many PoS
systems, like
Casper~\cite{buterin2017casper,casper-incentives},
Gasper~\cite{buterin2020combining}, and Tezos~\cite{tezos-pos},
this decrease is implemented as penalties for misbehavior.

In detail, each party $\party$ is required to lock an amount of assets in a
deposit $\deposit_{\party}$. If $\party$ violates a well-defined condition, its
deposit is forfeited, along with its accumulated rewards. Typically,
misbehavior is identified by a non-interactive proof of the misbehavior, \eg a
signed malicious message.

Considering profiles $\profile_\proto, \profile_{\strategy_{\party}}$ as
before, $\party$'s rewards take shape as follows. Under $\profile_\proto$,
$\party$ receives $\rewardVal_{\party, \profile_\proto}$, which are exchanged
under rate $\exchangeRateVal_{\profile_\proto}$; also, it retains its deposit
$\deposit_{\party}$, which is exchanged under the same rate. Under
$\profile_{\strategy_{\party}}$, $\party$ forfeits both its rewards and deposit, but
receives external utility $\utilityBoostVal_{\party, \profile_{\strategy_{\party}}}$.
Therefore, under penalties a party is incentivized to be compliant if
the forfeited deposit and rewards are larger than the external utility
(cf. Theorem~\ref{thm:penalties-ouroboros}).

\begin{theorem}\label{thm:penalties-ouroboros}
    Assume
    \begin{inparaenum}[i)]
        \item a synchronous network (cf. Section~\ref{sec:network-model}),
        \item the conflicting predicate $\infractionPredicate_{conf}$, and
        \item that $\forall \party \in \partySet: \miningpower_{\party} <$~$\frac{1}{2}$.
    \end{inparaenum}
    Also let:
    \begin{inparaenum}[i)]
        \item $\strategySet_{-\infractionPredicate_{conf}}$: the set of all non-compliant strategies;
        \item $\exchangeRateVal_{\profile_\proto}$: the (expected) exchange rate under $\profile_\proto$;
        \item $\exchangeRateVal_{\profile_{\strategy_{\party}}}$: the (expected) exchange rate when only $\party$ employs some non-compliant conflicting strategy $\strategy_{\party}$;
        \item $\utilityBoostVal_{\party, \profile_{\strategy_{\party}}}$: the external utility that $\strategy_{\party}$ yields for $\party$.
    \end{inparaenum}
    Finally, assume the block-proportional rewards (cf.
    Definition~\ref{def:proportional-rewards} for fixed block reward $\reward$)
    for which it also holds:
    $$
    \forall \executionTrace \; \forall \party \in \partySet: \reward_{\party, \executionTrace} =
    \left\{
    \begin{array}{ll}
        \proportionalRewardFunc(\chain_{\observer,\executionTrace}, \party) \cdot \totalReward_{\observer,\executionTrace} + \deposit_{\party},&\party \mbox{ produces no conflicting blocks in } \executionTrace\\
        0,&\mbox{otherwise}
    \end{array}
    \right.$$
    for a protocol-specific deposit value $\deposit_{\party}$ with
    $\rewardVal_{\party, \profile} = E[\reward_{\party,
    \execution_{\profile}}]$, \ie the expected rewards of $\party$ under
    profile $\profile$.

    Ouroboros with the above rewards and under the aforementioned externalities
    is not $(\epsilon, \infractionPredicate_{conf})$-compliant (cf.
    Definition~\ref{def:compliant}) \wrt utility \emph{Reward} (cf.
    Definition~\ref{def:utility}) and, if $\reward > \cost$, it is also not
    $(\epsilon, \infractionPredicate_{conf})$-compliant \wrt utility \emph{Profit},
    in both cases for a bound
    $\epsilon < \underset{\party \in \partySet}{\mathsf{max}}\{ \underset{\strategy_{\party} \in \strategySet_{-\infractionPredicate_{conf}}}{\mathsf{max}}\{ \utilityBoostVal_{\party, \profile_{\strategy_{\party}}} \} \} - \rewardVal_{\party, \profile_{\proto}} \cdot \exchangeRateVal_{\profile_\proto} - \mathsf{negl}(\secparam)$.
\end{theorem}
\begin{proof}
    When a party $\party$ employs a non-compliant strategy $\strategy_{\party}$,
    it receives an external utility $\utilityBoostVal_{\party,
    \profile_{\strategy_{\party}}}$. Also, in that case, it produces
    conflicting blocks (due to the non-compliance property of
    $\strategy_{\party}$). Therefore, by definition of the above
    block-proportional rewards, $\reward_{\party, \profile_{\strategy_{\party}}}
    = 0$; in other words, when $\party$ employs $\strategy_{\party}$ and produces
    conflicting blocks, $\party$ forfeits the protocol's rewards (which include
    the original rewards plus the deposit $\deposit_{\party}$). Therefore, when
    $\party$ employs $\strategy_{\party}$ and all other parties employ $\proto$,
    $\party$'s utility is
    $\utility_{\party}(\profile_{\strategy_{\party}}) = \utilityBoostVal_{\party, \profile_{\strategy_{\party}}}$
    (cf. Definition~\ref{def:utility-external}).

    We also remind that (from the proof of
    Theorem~\ref{thm:compliant-ouroboros-synchronous}), $\party$ can bias the
    leader schedule with some negligible probability $negl(\secparam)$, if it
    controls a minority of power.

    Therefore, for any party $\party$ and strategy $\strategy_{\party}$,
    $\profile_{\strategy_{\party}}$ is directly $\epsilon$-reachable from
    $\profile_{\proto}$ if:
    \begin{align}
        \rewardVal_{\party, \profile_{\proto}} \cdot \exchangeRateVal_{\profile_\proto} + \mathsf{negl}(\secparam) + \epsilon < \utilityBoostVal_{\party, \profile_{\strategy_{\party}}} \Leftrightarrow \nonumber \\
        \epsilon < \utilityBoostVal_{\party, \profile_{\strategy_{\party}}} - \rewardVal_{\party, \profile_{\proto}} \cdot \exchangeRateVal_{\profile_\proto} - \mathsf{negl}(\secparam) \nonumber
    \end{align}

    Across all parties and all non-compliant strategies, the maximum such $\epsilon$ is:
    \begin{align}
        \epsilon < \underset{\party \in \partySet}{\mathsf{max}}\{ \underset{\strategy_{\party} \in \strategySet_{-\infractionPredicate_{conf}}}{\mathsf{max}}\{ \utilityBoostVal_{\party, \profile_{\strategy_{\party}}} \} \} - \rewardVal_{\party, \profile_{\proto}} \cdot \exchangeRateVal_{\profile_\proto} - \mathsf{negl}(\secparam) \nonumber
    \end{align}

    We note that, as shown in
    Theorem~\ref{thm:compliant-ouroboros-synchronous}, Ouroboros with block
    proportional rewards under a synchronous network is an equilibrium, \ie the
    honest protocol yields the maximum rewards for each party compared to all
    other strategies. In the present setting, the honest protocol again yields
    the maximum utility, compared to all other \emph{compliant} strategies. To
    prove this it suffices to observe that, if a party does not produce
    conflicting blocks, its rewards are a linear function of the rewards of the
    setting of Theorem~\ref{thm:compliant-ouroboros-synchronous}; therefore,
    between compliant strategies, the honest protocol yields the maximum
    rewards (as shown in Theorem~\ref{thm:compliant-ouroboros-synchronous}).

    Therefore, the given bound of $\epsilon$ is bounded from below and, for
    $\epsilon$ less than this bound, there exists a party $\party$ that is
    incentivized to employ a (non-compliant) strategy $\strategy_{\party}$ and
    produce conflicting blocks, rendering $\proto$ not $(\epsilon,
    \infractionPredicate_{conf})$-compliant.
\end{proof}

An issue of both externality-based Theorems (\ref{thm:external-ouroboros}
and~\ref{thm:penalties-ouroboros}) is that $\epsilon$'s bound depends on
$\utilityBoostVal_{\party, \profile_{\strategy_{\party}}}$. This bound may be
high (\ie non-negligible), but it does not depend on protocol parameters, but
instead depends on external (to the protocol) variables (\eg the payoff of a
double-spending attack). Therefore, it is outside of the control of the
protocol's designer, who can only ensure that a protocol is secure by only
\emph{assuming} an upper bound on this external payoff.

Intuitively though, $\epsilon$'s bound shows that
attacks are prevented by two parameters.
First, the larger the deposit, the more attacks it protects against. For
example, a deposit of \$$10$M would protect against all attacks of
Table~\ref{tab:attacks}. However, large deposits also shut off small parties,
who do not own adequate assets. Therefore, a tradeoff exists in preventing
double spending attacks and enabling widespread participation.
Second, the longer the duration of an attack, the more blocks an adversary
needs to produce, hence the larger the rewards that it forfeits. Typically,
the attack duration depends on the number of confirmations that the offended
party, \eg an exchange, enforces. Therefore, enforcing different confirmation
limits, based on a transaction's value, would both enable fast settlement for
small transfers and protect large transactions.

Taking this observation into account, we briefly review users' behavior, when
running Ouroboros (cf.  Section~\ref{subsec:single-leader-pos}) under deposits
and penalties. In an Ouroboros execution, it is possible to identify the
percentage of parties that actively participate during each epoch, by observing
the block density and the number of empty slots (\ie when no block is
diffused). Therefore, it is possible to estimate the level of infraction, \ie
double-signing, that a party needs to perform to mount a double-spending
attack, and then enforce a transaction finalization rule to disincentivize
such attacks.

Let $\party$ be a user of an Ouroboros ledger. $\party$ enforces
a rule of $k$ confirmations, \ie finalizes a transaction after it is
``buried'' under $k$ blocks. Let $\tau$ be a transaction published in a block
on slot $\slot$, with value $v_\tau$. After $l$ slots, $\party$ sees that
$\tau$ has been buried under $b$ blocks, with $b = x \cdot l$ for some $x > 0.5$. Therefore, $(1 -
x) \cdot 100$\% of slots are -- seemingly -- empty. $\party$ will (on expectation)
confirm $\tau$ after $\frac{1}{x} \cdot k$ slots, \ie when $k$
blocks are produced; of these, $\frac{1 - x}{x} \cdot k$ are empty.

Let $\adversary$ be a party that wants to double-spend $\tau$.
$\adversary$ should produce a private chain with at least $k$ blocks. Of these,
at most $\frac{1 - x}{x} \cdot k$ correspond to the respective empty slots,
while $k - \frac{1 - x}{x} \cdot k = \frac{2 \cdot x - 1}{x} \cdot k$ conflict
with existing blocks, \ie are evidence of infraction.

Let $d$ be a deposit amount, which corresponds to a single slot. Thus, for a
period of $t$ slots, the total deposited assets $D = t \cdot d$ are distributed evenly
across all slots.
$\adversary$ can be penalized only for infraction blocks, \ie for
slots which showcase conflicting blocks. In a range of $\frac{1}{x} \cdot k$ slots, infraction slots are $\frac{2 \cdot x - 1}{x} \cdot k$.

Therefore, $\adversary$ forfeits at most $\frac{2 \cdot x - 1}{x} \cdot k \cdot
d$ in deposit and $\frac{2 \cdot x - 1}{x} \cdot k \cdot \reward$ in rewards
that correspond to infraction blocks.
Therefore, if $v_\tau > \frac{2 \cdot x - 1}{x} \cdot k \cdot (d + \reward)$,
$\adversary$ can profitably double-spend $\tau$.
Consequently, depending on the amount $d$ of deposit per slot, the block reward
$\reward$, and the rate $(1 - x)$ of empty slots, for a transaction $\tau$ with
value $v_\tau$, $\party$ should set the confirmation window's size to:
\begin{align}
    k_\tau > \frac{v_\tau}{\frac{2 \cdot x - 1}{x} \cdot (d + \reward)}
\end{align}

Finally, the system should allow each participant to withdraw their deposit
after some time. However, it should also enforce \emph{some} time limit, to
ensure that adequate deposit exists to (possibly) enforce a penalty.
Intuitively, a party $\party$ should be able to withdraw a deposit amount that
corresponds to a slot $\slot$, only if no transaction exists, such that $\slot$
is part of the window of size $k$ (with $k$ computed as above). In other words,
$\party$'s deposit should be enough to cover all slots, which $\party$ has led
and which are part of the confirmations' window of at least one non-finalized
transaction.

\begin{remark*}
Penalties are applicable, and indeed common, in PoS systems,
like those mentioned above, but the same does not hold for PoW systems. In PoW
protocols like Bitcoin, the block producers are decoupled from the users of the
system, thus it is often impossible to identify which party produced each
block. Naturally, the same is true for fully anonymous protocols, both
PoW~\cite{SP:MGGR13,SP:BCGGMT14} and PoS~\cite{EC:GanOrlTsc19,SP:KKKZ19}.
\end{remark*}
