\section{Blockchain Protocols}\label{sec:blockchains}

In this work, we focus on blockchain-based distributed ledger protocols.  In
the general case, a ledger defines a global state, which is distributed across
multiple parties and is maintained via a consensus protocol. The distributed
ledger protocol defines the validity rules which allow a party to extract the
final ledger from its view. A
blockchain is a distributed database, where each message $\mesg$ is a block
$\block$ of transactions and each transaction updates the system's global
state. Therefore, at any point of the execution, a party $\party$ holds some view of
the global state, which comprises of the blocks that $\party$ has
adopted. We note that, if at least one valid block is diffused (\wrt the
validity rules of the protocol), then every honest party can
extract a final ledger from its execution view.

\subsection{The Setting}\label{subsec:setting}

Every blockchain protocol $\proto$ defines a \emph{message validity} predicate
$\messageValidityPredicate$.  A party $\party$ accepts a block $\block$,
received during a slot $\slot$ of the execution, if it holds that
$\messageValidityPredicate(\executionTrace^{\party}_\slot, \block) = 1$. For
example, in Proof-of-WorK (PoW) systems like
Bitcoin~\cite{nakamoto2008bitcoin}, a block is valid if its hash is below a
certain threshold, while in Proof-of-Stake (PoS) protocols like
Ouroboros~\cite{C:KRDO17}, a block is valid if it has been created by a
specific party, according to a well-known leader schedule. In all cases, a
block $\block$ is valid only if the party that creates it submits a query for
$\block$ to the oracle $\oracle_\proto$ beforehand.

Each block $\block$ is associated with the following metadata:
\begin{inparaenum}[i)]
    \item an index $\mathit{index}(\block)$;
    \item the party $\mathit{creator}(\block)$ that created $\block$;
    \item a set $\mathit{ancestors}(\block) \subseteq
    \executionTrace^{\mathit{creator}(\block)}$, \ie blocks in the view of
    $\mathit{creator}(\block)$ (at the time of $\block$'s creation)
    referenced by $\block$.
\end{inparaenum}

\noindent Message references are implemented as hash pointers, given a hash function
$\hash$ employed by the protocol. Specifically, each block $\block$ contains the
hash of all blocks in the referenced blocks $\mathit{ancestors}(\block)$.
Blockchain systems
are typically bootstrapped via a global common reference string, \ie a
``genesis'' block $\genesis$.
Therefore, the blocks form a hash tree, stemming from $\genesis$.
$\mathit{index}(\block)$ is the height of $\block$ in the hash tree. If
$\block$ references multiple messages, \ie belongs to multiple tree branches,
$\mathit{index}(\block)$ is the height of the longest one.

The protocol also defines the \emph{message equivalency operator}, $\equiv$. Specifically, two messages are equivalent if
their hashes match, \ie $\mesg_1 \equiv \mesg_2 \Leftrightarrow \hash(\mesg_1) =
\hash(\mesg_2)$. At a high level, two equivalent messages are
interchangeable by the protocol.

\paragraph{Infraction Predicate}\label{sec:blockchain-infraction-predicate}
In our analysis of blockchain systems, we consider two types of deviant
behavior (Definition~\ref{def:blockchain-infraction}):
\begin{inparaenum}[i)]
    \item creating conflicting valid messages,
    \item abstaining.
\end{inparaenum}
We choose these predicates because they may lead to non-compliance in
interesting use cases. For the former, conflicting valid messages, \ie chain
forks, are the primary reason for failure to achieve consensus, so exploring
when a party is incentivized to initiate a fork is particularly interesting.
For the latter, increased and continuous participation typically helps the
safety of deployed systems, as motivating more users to participate in a
compliant manner increases the level of power that an adversary needs to reach
in order to break a system's security.

\begin{definition}[Blockchain Infraction Predicate]\label{def:blockchain-infraction}
    Given a party $\party$ and an execution trace $\executionTrace_{\env, \sigma, \slot}$,
    we define the following infraction predicates:
    \begin{enumerate}
        \item
            \emph{conflicting predicate}: $\infractionPredicate_{conf}(\executionTrace_{\env, \sigma, \slot}, \party) = 1$ if
            $\exists \block, \block'
            \in \executionTrace_{\env, \sigma, \slot}:\messageValidityPredicate(\executionTrace_{\env, \sigma, \slot}^{\party}, \block)=\messageValidityPredicate(\executionTrace_{\env, \sigma, \slot}^{\party}, \block')=1\land
            \mathit{index}(\block) = \mathit{index}(\block') \land \block \not \equiv
            \block' \land \mathit{creator}(\block) = \mathit{creator}(\block') = \party$;
        \item
            \emph{abstaining predicate}: $\infractionPredicate_{abs}(\executionTrace_{\env, \sigma, \slot}, \party) = 1$ if
            $\party$ makes \emph{no} queries to the oracle $\oracle_\proto$
            during slot $\slot$;
        \item
            \emph{blockchain predicate}: $\infractionPredicate_{bc}(\executionTrace_{\env, \sigma, \slot}, \party) = 1$ if it holds
            $(\infractionPredicate_{conf}(\executionTrace_{\env, \sigma, \slot}, \party) = 1) \lor (\infractionPredicate_{abs}(\executionTrace_{\env, \sigma, \slot}, \party) = 1)$.
    \end{enumerate}
\end{definition}

\begin{remark*}
Preventing conflicting messages is not the same
as resilience against sybil attacks~\cite{douceur2002sybil}. Sybil resilience
mechanisms restrict an attacker from creating multiple identities, in order to
participate in a distributed system. In comparison, our infraction predicate
ensures that a user does not increase their utility by violating the infraction
conditions for each identity they control. Therefore, a system may be compliant
but not sybil resilient, \eg if a party can participate via multiple identities
but it cannot increase its utility by producing conflicting messages, and vice
versa.
\end{remark*}

Finally, at the end of the execution, the observer $\observer$ outputs a chain
$\chain_{\observer, \executionTrace}$. Typically, this is the longest valid chain, \ie the longest
branch of the tree that stems from genesis $\genesis$.\footnote{For simplicity we assume that the longest chain (in terms of
blocks) also contains the most hashing power, which is the metric used in PoW
systems like Bitcoin.} In case multiple longest chains exist, a choice is made
either at random or following a chronological ordering of messages. The number
of messages in $\chain_{\observer, \executionTrace}$ that are created by a party $\party$ is
denoted by $\msgOutputSet_{\party, \executionTrace}$.
