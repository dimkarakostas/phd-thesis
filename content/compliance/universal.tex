\section{Fair Rewards}\label{sec:universal}

As described in Section~\ref{sec:model}, each party $\party$ controls a percentage
$\miningpower_{\party}$ of the system's participating power. Although
this parameter is set at the beginning of the execution, it is not always
public. For instance, a party could obscure its amount of hashing power by
refraining from performing some queries. In other cases, each
party's power is published on the distributed ledger and, for all executions,
can be extracted from the observer's chain. A prime example of such systems is
non-anonymous PoS ledgers, where each party's power is denoted by
its assets, which are logged in real time on the ledger.

These systems, where power distribution is public, can employ a special
type of rewards, \emph{fair rewards}.\footnote{The term
\emph{fairness} is an allusion to Fruitchains~\cite{PODC:PasShi17}.}
Specifically, the system
defines a fixed, total number of rewards $\totalReward > 0$. At the end
of an execution, if at least one block is created, each party $\party$ receives
a percentage $\universalRewardFunc(\miningpower_{\party})$ of
$\totalReward$, where $\universalRewardFunc(\cdot): [0, 1] \rightarrow [0, 1]$;
in the real world, $\universalRewardFunc$ is usually the identity
function. If no blocks are created during the execution, then every party gets
$0$ rewards.

Intuitively, fair rewards compensate users for investing in the
system. Unless no block is created (which typically happens with very small
probability when the parties honestly follow the protocol), the level of
rewards depends \emph{solely} on a party's power, and not in the messages
diffused during the execution. Definition~\ref{def:universal-rewards} formally
describes the family of fair reward random variables.

\begin{definition}[Fair Rewards]\label{def:universal-rewards}
    For a total number of rewards $\totalReward \in \mathbb{R}_{> 0}$, a
    \emph{fair reward} random variable $\reward_{\party, \execution}$ satisfies
    the following:
    \begin{enumerate}
        \item $\forall \executionTrace \; \forall \party \in \partySet: \reward_{\party, \executionTrace} =
 \left\{\begin{array}{ll}
  \universalRewardFunc(\miningpower_{\party}) \cdot \totalReward,&\mbox{if at least one valid block is produced during }\executionTrace\\
  0,&\mbox{otherwise}
\end{array}
\right.$
        \item $\sum_{\party \in \partySet} \universalRewardFunc(\miningpower_{\party}) = 1$
    \end{enumerate}
    where $\universalRewardFunc: [0, 1] \rightarrow [0, 1]$.
\end{definition}

As shown in Theorem~\ref{thm:universal-reward}, blockchains with fair
rewards are $(\epsilon,\infractionPredicate)$-compliant for the utility Reward (cf. Definition~\ref{def:utility}), where $\epsilon$ is typically a small value and $\infractionPredicate$ is an arbitrary associated infraction predicate. Intuitively, since a
party is rewarded the same amount regardless of their protocol-related actions,
nobody can increase their rewards by deviating from the honest
strategy (as long as at least one block is produced).

\begin{theorem}\label{thm:universal-reward}
    Let:
    \begin{inparaenum}[i)]
        \item $\proto$ be a blockchain protocol run by the parties $\party_1, \ldots, \party_\totalParties$;
        \item $\infractionPredicate$ be any infraction predicate associated with $\proto$;
        \item $\bar{\utility}=\langle \utility_1, \ldots, \utility_\totalParties \rangle$ be a utility vector, where $\utility_i$ is the utility \emph{Reward} of party $\party_i$;
        \item $\totalReward$ be the total rewards distributed by the protocol;
        \item $\universalRewardFunc: [0, 1] \rightarrow [0, 1]$ be a fair reward function;
        \item $\alpha$ be the probability that no blocks are produced when all parties follow the honest strategy.
    \end{inparaenum}
    Then, $\proto$ is $(\epsilon, \infractionPredicate)$-compliant \wrt $\bar{\utility}$, for $\epsilon := \alpha \cdot \underset{j \in [\totalParties]}{\mathsf{max}}\{\universalRewardFunc(\miningpower_{\party_j}) \cdot \totalReward \}$.
\end{theorem}
\begin{proof}
    By Definition~\ref{def:universal-rewards} and the definition of $\alpha$, for the ``all honest'' strategy profile $\profile_\proto := \langle \proto, \ldots, \proto \rangle$, we have that $\Pr[\reward_{\party_i,\execution_{\profile_\proto}} = \universalRewardFunc(\miningpower_{\party_i}) \cdot \totalReward] = 1 - \alpha$ and $\Pr[\reward_{\party_i, \execution_{\profile_\proto}} = 0] = \alpha$, for every $i \in [\totalParties]$. Therefore, for every $i \in [\totalParties]$,
%
    $\utility_i(\profile_\proto) = E\big[\reward_{\party_i,\execution_{\profile_{\party}}}\big] = (1 - \alpha) \cdot \universalRewardFunc(\miningpower_{\party_i}) \cdot \totalReward$.

    Assume that for some $i \in [\totalParties]$, $\party_i$ unilaterally deviates
    from $\proto$ by employing a different strategy $\strategy_i$. In this case, we
    consider the strategy profile $\profile = \langle \strategy_1, \ldots, \strategy_n \rangle$
    where $\strategy_j = \proto$ for $j \in [\totalParties] \backslash \{i\}$. Since $\utility_i$ is the utility
    \emph{Reward} under fair rewards with $\totalReward, \universalRewardFunc(\cdot)$, we have that for all random coins of the execution $\execution_{\profile}$, the value of the reward random variable $\reward_{\party_i, \execution_{\profile}}$ is no more than $\universalRewardFunc(\miningpower_{\party_i}) \cdot \totalReward$.
    Consequently, $\utility_i(\profile) \leq \universalRewardFunc(\miningpower_{\party_i}) \cdot \totalReward$, and so we have that
    %
    \[\utility_i(\profile) \leq \utility_i(\profile_\proto) + \alpha \cdot \universalRewardFunc(\miningpower_{\party_i}) \cdot \totalReward \leq \utility_i(\profile_\proto) + \alpha \cdot \underset{j \in [\totalParties]}{\mathsf{max}}\{\universalRewardFunc(\miningpower_{\party_j}) \cdot \totalReward \}\;.\]
    %
    If $\epsilon := \alpha \cdot \underset{j \in [\totalParties]}{\mathsf{max}}\{\universalRewardFunc(\miningpower_{\party_j}) \cdot \totalReward \}$ and since $i$ and $\strategy_i$ are arbitrary, we conclude that $\proto$ is a $\epsilon$-Nash equilibrium
    \wrt $\bar{\utility}$. Thus, by Proposition~\ref{prop:eq_comp}, $\proto$ is
    $(\epsilon, \infractionPredicate)$-compliant \wrt $\bar{\utility}$.
\end{proof}

Theorem~\ref{thm:universal-reward} is consistent with the incentives' analysis
of~\cite{C:KRDO17} under fair rewards. However, when
introducing operational costs to analyze profit, a problem arises: a user can
simply abstain and be rewarded nonetheless. Such behavior results in a
``free-rider problem''~\cite{baumol2004welfare}, where a user reaps some
benefits while not under-paying them or not paying at all.
Theorem~\ref{thm:universal-profit} formalizes this argument and
shows that a blockchain protocol, associated with the abstaining infraction predicate $\infractionPredicate_{abs}$ (cf. Definition~\ref{def:blockchain-infraction}), under fair rewards is \emph{not}
$(\epsilon, \infractionPredicate_{abs})$-compliant \wrt utility \emph{Profit}, for reasonable values of $\epsilon$.

\begin{theorem}\label{thm:universal-profit}
    Let:
    \begin{inparaenum}[i)]
        \item $\proto$ be a blockchain protocol run by the parties $\party_1, \ldots, \party_\totalParties$;
        \item $\infractionPredicate_{abs}$ be the abstaining infraction predicate;
        \item $\bar{\utility}=\langle \utility_1, \ldots, \utility_\totalParties \rangle$ be a utility vector, where $\utility_i$ is the utility \emph{Profit} of party $\party_i$;
        \item $\totalReward$ be the total rewards distributed by the protocol;
        \item $\universalRewardFunc: [0, 1] \rightarrow [0, 1]$ be a fair reward function;
        \item $\alpha$ be the probability that no blocks are produced when all parties follow the honest strategy.
    \end{inparaenum}

    For $i \in [\totalParties]$, also let the following:
    \begin{inparaenum}[i)]
        \item $\cost^\bot_{\party_i}$ be the minimum cost of $\party_i$ across all possible execution traces where $\party_i$ employs $\proto$;
        \item $\beta_i$ be the probability that no blocks are produced when $\party_i$ abstains throughout the entire execution and all the other parties follow $\proto$.
    \end{inparaenum}

     Then, for every $\epsilon\geq0$ such that
     $\epsilon < \underset{i \in [\totalParties]}{\mathsf{max}}\{\cost^\bot_{\party_i} - (\beta_i - \alpha) \cdot \universalRewardFunc(\miningpower_{\party_i}) \cdot \totalReward \}$,
     the protocol $\proto$ is \emph{not} $(\epsilon, \infractionPredicate_{abs})$-compliant \wrt $\bar{\utility}$.
\end{theorem}
\begin{proof}
    By Definition~\ref{def:universal-rewards} and the definition of $\alpha$,
    for the ``all honest'' strategy profile $\profile_{\proto} := \langle
    \proto, \ldots, \proto \rangle$, we have that
    $\Pr[\reward_{\party_i, \execution_{\profile_{\proto}}} = \universalRewardFunc(\miningpower_{\party_i}) \cdot \totalReward] = 1 - \alpha$
    and $\Pr[\reward_{\party_i, \execution_{\profile_{\proto}}} = 0] = \alpha$,
    for every $i \in [\totalParties]$.  Since $\proto$ is an
    $\infractionPredicate_{abs}$-compliant strategy, if $\party_i$ follows
    $\proto$ then it does not abstain, \ie it makes queries to
    $\oracle_\proto$. Therefore, the cost of $\party_i$ is at least
    $\cost^\bot_{\party_i}$ and for $\profile_{\proto}$, the utility
    \emph{Profit} $\utility_i(\profile_{\proto})$ is no more than
    \[\utility_i(\profile_{\proto}) = E\big[\reward_{\party_i, \execution_{\profile_{\proto}}}\big] - E\big[\cost_{\party_i, \execution_{\profile_{\proto}}}\big] \leq (1 - \alpha) \cdot \universalRewardFunc(\miningpower_{\party_i}) \cdot \totalReward - \cost^\bot_{\party_i}\;.\]

    Now assume that $\party_i$ unilaterally deviates by following the ``always
    abstain" strategy, $\strategy_\mathsf{abs}$, which is of course not
    $\infractionPredicate_{abs}$-compliant. Then, $\party_i$ makes no queries
    to $\oracle_\proto$ and, by Assumption~\ref{ass:zero-cost}, its cost is
    $0$. Let $\profile_i$ be the strategy profile where $\party_i$ follows
    $\strategy_\mathsf{abs}$ and every party $\party \neq \party_i$ follows
    $\proto$. By the definition of $\beta_i$, we have that
    $\Pr[\reward_{\party_i, \execution_{\profile_i}} = \universalRewardFunc(\miningpower_{\party_i}) \cdot \totalReward] = 1 - \beta_i$ and $\Pr[\reward_{\party_i, \execution_{\profile_i}} = 0] = \beta_i$.

    By the definition of $\utility_i$, it holds that
    \[\utility_i(\profile_i) = E\big[\reward_{\party_i, \execution_{\profile_i}}\big] - E\big[\cost_{\party_i, \execution_{\profile_i}}\big] = (1 - \beta_i) \cdot \universalRewardFunc(\miningpower_{\party_i}) \cdot \totalReward - 0\;.\]
    Assume that
    $\cost^\bot_{\party_i} - (\beta_i - \alpha) \cdot \universalRewardFunc(\miningpower_{\party_i}) \cdot \totalReward > 0$. Then, for $\epsilon_i < \cost^\bot_{\party_i} - (\beta_i - \alpha) \cdot \universalRewardFunc(\miningpower_{\party_i}) \cdot \totalReward$,
    we have that
    \begin{equation*}
        \begin{split}
            \utility_i(\profile_i) & = (1 - \beta_i) \cdot \universalRewardFunc(\miningpower_{\party_i}) \cdot \totalReward = \\
            & = (1 - \alpha) \cdot \universalRewardFunc(\miningpower_{\party_i}) \cdot \totalReward + (\alpha - \beta_i) \cdot \universalRewardFunc(\miningpower_{\party_i}) \cdot \totalReward = \\
            & = (1 - \alpha) \cdot \universalRewardFunc(\miningpower_{\party_i}) \cdot \totalReward - \cost^\bot_{\party_i} + \cost^\bot_{\party_i} - (\beta_i - \alpha) \cdot \universalRewardFunc(\miningpower_{\party_i}) \cdot \totalReward \geq \\
            & \geq \utility_i(\profile_{\proto}) + \cost^\bot_{\party_i} - (\beta_i - \alpha) \cdot \universalRewardFunc(\miningpower_{\party_i}) \cdot \totalReward > \\
            & > \utility_i(\profile_{\proto}) + \epsilon_i
        \end{split}
    \end{equation*}

    Let $i^*\in[n]$ be such that
    $\cost^\bot_{\party_{i^*}} - (\beta_{i^*} - \alpha) \cdot \universalRewardFunc(\miningpower_{\party_{i^*}}) \cdot \totalReward \} = \underset{i \in [\totalParties]}{\mathsf{max}}\{ \cost^\bot_{\party_i} - (\beta_i - \alpha) \cdot \universalRewardFunc(\miningpower_{\party_i}) \cdot \totalReward \}$.
    By the above, for every
    $0\leq\epsilon < \cost^\bot_{\party_{i^*}} - (\beta_{i^*} - \alpha) \cdot \universalRewardFunc(\miningpower_{\party_{i^*}}) \cdot \totalReward \}$,
    we have that $\profile_{i^*}$ is directly $\epsilon$-reachable from
    $\profile_\proto$ \wrt $\bar{\utility}$.

    Thus, $\profile_{i^*}$ is a non $\infractionPredicate_{abs}$-compliant
    strategy profile that is in $\mathsf{Cone}_{\epsilon,
    \bar{\utility}}(\proto)$. Consequently, for
    $0 \leq\epsilon < \underset{i \in [\totalParties]}{\mathsf{max}}\{ \cost^\bot_{\party_i} - (\beta_i - \alpha) \cdot \universalRewardFunc(\miningpower_{\party_i}) \cdot \totalReward \}$,
    it holds that
    $\mathsf{Cone}_{\epsilon, \bar{\utility}}(\proto) \subsetneq (\strategySet_{\infractionPredicate_{abs}})^\totalParties$,
    \ie the protocol $\proto$ is not $(\epsilon,
    \infractionPredicate_{abs})$-compliant \wrt $\bar{\utility}$.
\end{proof}

Before concluding, let us examine the variables of the bound
$\underset{i \in [\totalParties]}{\mathsf{max}}\{ \cost^\bot_{\party_i} - (\beta_i - \alpha) \cdot \universalRewardFunc(\miningpower_{\party_i}) \cdot \totalReward \}$
of Theorem~\ref{thm:universal-profit}. We note that, in the context of
blockchain systems, a ``party'' is equivalent to a unit of power; therefore, a
party $\party$ that controls $\miningpower_{\party}$ of the total power, in
effect controls $\miningpower_{\party}$ of all ``parties'' that participate in
the blockchain protocol.

To discuss $\alpha$ and $\beta_i$, we first consider the liveness
property~\cite{RSA:GarKia20} of blockchain protocols. Briefly, if a protocol
guarantees liveness with parameter $u$, then a transaction which is diffused on
slot $\slot$ is part of the (finalized) ledger of every honest party on round
$\slot + u$. Therefore, assuming that the environment gives at least one
transaction to the parties, if a protocol $\proto$ guarantees liveness unless
with negligible probability $\mathsf{negl}(\secparam)$,\footnote{Recall that $\secparam$ is
$\proto$'s security parameter, while $\mathsf{negl}(\cdot)$ is a negligible function.}
then at least one block is created during the execution with overwhelming
probability (in $\secparam$).

Now, we consider the probabilities $\alpha$ and  $\beta_i$.  The former is
negligible, since consensus protocols typically guarantee liveness against a
number of crash (or Byzantine) faults, let alone if all parties are honest. The
latter, however, depends on $\party_i$'s percentage of power
$\miningpower_{\party_i}$. For instance, consider Ouroboros, which is secure if
a deviating party $\party_i$ controls less than $\frac{1}{2}$ of the staking
power and all others employ $\proto$. Thus, if $\miningpower_{\party_i} =
\frac{2}{3}$ and $\party_i$ abstains, the protocol cannot guarantee liveness,
\ie it is probable that no blocks are created. However, if
$\miningpower_{\party_i} = \frac{1}{4}$, then liveness is guaranteed with
overwhelming probability; hence, even if $\party_i$ abstains, at least one
block is typically created.
Corollary~\ref{col:universal-profit-non-compliance} generalizes this argument,
by showing that, if enough parties participate, then at least one of them is
small enough, such that its abstaining does not result in a system halt, hence
it is incentivized to be non-compliant.

\begin{corollary}\label{col:universal-profit-non-compliance}
    Let $\proto$ be a blockchain protocol, with security parameter $\secparam$,
    which is run by $\totalParties$ parties, under the same considerations of
    Theorem~\ref{thm:universal-profit}.
    Additionally, assume that $\proto$ has liveness with security threshold
    $\frac{1}{x}$ in the following sense: for every strategy profile
    $\profile$, if $\underset{\party \in \partySet_{-\profile}}{\sum}
    \miningpower_{\party} < \frac{1}{x}$, where $\partySet_{-\profile}$ is the
    set of parties that deviate from $\proto$ when $\profile$ is followed, then
    $\proto$ guarantees liveness with overwhelming (i.e.,
    $1-\mathsf{negl}(\secparam)$) probability.

    If $x < \totalParties$, then for (non-negligible) values
    $\epsilon < \underset{i \in [\totalParties]}{\mathsf{max}}\{\cost^\bot_{\party_i}\} - \mathsf{negl}(\secparam)$,
    $\proto$ is \emph{not} $(\epsilon, \infractionPredicate_{abs})$-compliant \wrt $\bar{\utility}$.
\end{corollary}
\begin{proof}
    First, since $\proto$ guarantees liveness even under some byzantine faults,
    $\alpha=\mathsf{negl}(\secparam)$.

    Second, if $x < \totalParties$, then there exists $i\in[n]$ such that
    $\miningpower_{\party_i} < \frac{1}{x}$. To prove this, if $\totalParties >
    x$ and $\forall j\in[n]: \miningpower_{\party_j} \geq \frac{1}{x}$ then
    $\underset{j \in [\totalParties]}{\sum} \miningpower_{\party_i} \geq \frac{n}{x} > 1$.
    This contradicts to the definition of the parties' participating power (cf.
    Section~\ref{sec:model}), where it holds that $\underset{j \in
    [\totalParties]}{\sum} \miningpower_{\party_j}=1$.

    Now consider the strategy profile $\profile$ where $\party_i$ abstains and
    all the other parties honestly follow $\proto$. Then, by definition,
    $\partySet_{-\profile}=\{\party_i\}$ and therefore,
    \[\underset{\party \in \partySet_{-\profile}}{\sum} \miningpower_{\party}=\miningpower_{\party_i} < \frac{1}{x}\;.\]
    Thus, by the assumption for $\proto$, we have that if the parties follow
    $\profile$, then $\proto$ guarantees liveness with
    $1-\mathsf{negl}(\secparam)$ probability. Hence, $\beta_i =
    \mathsf{negl}(\secparam)$.  Finally, since
    $\universalRewardFunc(\miningpower_{\party_i}) \in [0, 1]$ and
    $\totalReward$ is a finite value irrespective of the parties' strategy
    profile, the value $(\beta_i - \alpha) \cdot
    \universalRewardFunc(\miningpower_{\party_i}) \cdot \totalReward$ is also
    negligible in $\secparam$.
\end{proof}

The minimal cost $\cost^\bot_{\party_i}$ of honest participation for
$\party_i$ depends on the blockchain system's details. In PoW systems, where
participation consists of repeatedly performing computations, cost increases
with the percentage of mining power; for instance, controlling $51$\% of
Bitcoin's mining power for $1$ hour costs
\$$700,000$.\footnote{\url{https://www.crypto51.app} [May 2021]} In PoS
systems, cost is typically irrespective of staking power, since participation
consists only of monitoring the network and regularly signing messages; for
example, running a production-grade Cardano node costs \$$140$ per
month\footnote{https://forum.cardano.org/t/realistic-cost-to-operate-stake-pool/40056
[September 2020]}. Therefore, taking
Corollary~\ref{col:universal-profit-non-compliance} into account, the upper
bound $\underset{i \in [\totalParties]}{\mathsf{max}}\{\cost^\bot_{\party_i}\}
- \mathsf{negl}(\secparam)$ of $\epsilon$ is typically rather large for PoS
systems.

The free-rider hazard is manifested in
Algorand\footnote{\url{https://algorand.foundation}}, a cryptocurrency system
that follows the Algorand consensus
protocol~\cite{EPRINT:CGMV18,EPRINT:GHMVZ17} and employs fair rewards, as
defined above. Its users own ``Algo'' tokens and transact over a ledger
maintained by ``participation nodes'', which run the Algorand protocol and
extend the ledger via blocks. Each user receives a fixed reward\footnote{For a
weekly execution, the reward per owned Algo is $0.001094$ Algos.
[\url{https://algoexplorer.io/rewards-calculator} [October 2020]} per Algo
token they own~\cite{algorand-rewards}, awarded with every block.  Users
may run a participation node, but are not
rewarded~\cite{DBLP:conf/dsn/FooladgarMJR20} for doing so, and participation is
proportional to the amount of Algos that the user owns. Therefore, a party that
owns a few Algos will expectedly abstain from participation in the consensus
protocol.

In conclusion, fair rewards in PoS protocols may incentivize users to abstain.
In turn, this can impact the system's performance, \eg delaying block
production and transaction finalization. In the extreme case, when multiple
parties can change their strategy simultaneously, it could result in a
``tragedy of the commons'' situation~\cite{lloyd1833two}, where \emph{all}
users abstain and the system grinds to a halt.

\begin{remark*}
This section illustrates a difference between PoW and PoS. In
PoS systems, each party's power is registered on the ledger (in the form of
assets), without requiring any action from each party. In PoW, a party's power
becomes evident only after the party puts their hardware to work. This is
demonstrated in Fruitchains~\cite{PODC:PasShi17}, a PoW protocol where each
party receives a reward (approximately) proportional to their mining power.
However, the PoW protocol cannot identify each party's power, unless the party
participates and produces some blocks (or ``fruits''). Therefore, since the
proof of Theorem~\ref{thm:universal-profit} relies on abstaining, this result
does not directly translate to Fruitchains, or similar PoW protocols.
\end{remark*}
