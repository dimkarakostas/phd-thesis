\subsection{Attacks and Market Response}\label{sec:attacks}

To estimate the exchange rate's behavior vis-à-vis external infraction rewards,
we turn to historical data from the cryptocurrency market. Although no
infractions of the type considered in this work have been observed in
deployed PoS systems, we extrapolate data from similar attacks against PoW
cryptocurrencies (Table~\ref{tab:attacks}).

In the considered attacks, the perpetrator $\adversary$ performed double
spending. Specifically, $\adversary$ created a fork and two conflicting
transactions, each of which is published on the two chains of the fork, the
main and the adversarial chain. The main chain's transaction is redeemed for
external rewards, \eg a payment in USD, while the adversarial chain's
transaction simply transfers the assets between two accounts of $\adversary$.
Therefore, $\adversary$ both receives external rewards and retains its
cryptocurrency rewards.

The adversarial chain contains a number of blocks created by $\adversary$.
After this chain becomes longest and is adopted by the network,
$\adversary$ sells its block rewards for USD.  To evaluate the exchange rate at
this point, we set the period between the launch of the attack and the
(presumable) selling of the block rewards to $5$ days. This value depends on
various parameters. For instance, in Bitcoin, the rewards for a block $\block$
can be redeemed after a ``coinbase maturity'' period of $100$ confirmations,
\ie after at least $100$ blocks have been mined on top of $\block$ (equiv. $17$
hours).\footnote{A Bitcoin
block is created on expectation every $10$ minutes.}
Furthermore, transactions are typically not finalized immediately; for
instance, most parties finalize a Bitcoin transaction after $6$ confirmations
and an Ethereum transaction after $240$ confirmations (equiv. approximately $1$
hour).
Usually this restraint is tightened~\cite{etc-attack-4} after an
attack is revealed.

To estimate the difference in rewards that an infraction effects, we use
cryptocurrency prices from
CoinMarketCap\footnote{\url{https://coinmarketcap.com/}}. First, we obtain the
price $P_{C}$ of each cryptocurrency $C$ $5$ days after the attack. Second, we
compute the percentage difference $p_{BTC}$ of Bitcoin's price, between the end
and the beginning of the $5$ day period. The value $P_{C} \cdot p_{BTC}$
expresses the \emph{expected} price of the cryptocurrency, assuming no attack
had occurred.\footnote{Historically, the prices of Bitcoin and alternative
cryptocurrencies are strongly correlated~\cite{btc-price-correlation}.} Next,
we find the number of blocks $b$ created during the attack and the reward
$\reward$ per block. Therefore, the reward difference is computed as $P_{C}
\cdot p_{BTC} \cdot b \cdot \reward$.

As shown in~\cite{CCS:GazKiaRus20,CCS:DKTTVWZ20}, this attack is
optimal. Therefore, using the computations in~\cite{nakamoto2008bitcoin} and
the reorganized blocks during each attack, we approximate the
necessary percentage of adversarial power to successfully mount the attack with
at least $0.5$ probability.

\begin{table}[ht]
    \centering \def\arraystretch{1.5}

    \begin{center}
      \footnotesize
        \begin{tabular}{|c|c|c|c|c|c|}
            \hline

              System
            & Date
            & \begin{tabular}[c]{@{}c@{}} External \\ Utility \end{tabular}
            & \begin{tabular}[c]{@{}c@{}} Rewards \end{tabular}
            & \begin{tabular}[c]{@{}c@{}} Reward \\ Difference \end{tabular}
            & \begin{tabular}[c]{@{}c@{}} Attacker \\ Hash Rate \% \end{tabular} \\
            \hline

            \multirow{4}{*}{\begin{tabular}[c]{@{}c@{}} Ethereum \\ Classic \end{tabular}}   & 5/1/19 \cite{etc-attack-2}        & \$$1.1$M     & \$$12.410$       & \$$-2,646$       & $0.48026$  \\
                                                                                             & 1/8/20 \cite{etc-attack}          & \$$5.6$M     & \$$84,059$      & \$$-11,806$     & $0.4913$  \\
                                                                                             & 6/8/20 \cite{etc-attack-3}        & \$$1.68$M    & \$$91,715$      & \$$-5,761$      & $0.4913$  \\
                                                                                             & 30/8/20 \cite{etc-attack-4}       & unknown      & \$$120,131$      & \$$-12,992$      & $0.5$  \\
            \hline

            Horizen                             & 8/6/18 \cite{zencash-attack}      & \$$550,000$   & \$$5,756$      & \$$-752$  & $0.461373$ \\
            \hline

            \multirow{1}{*}{Vertcoin}           & 2/12/18 \cite{vertcoin-attack}     & \$$100,000$   & \$$3,978$       & \$$-879$  & $0.487124$ \\
            \hline

            \multirow{2}{*}{\begin{tabular}[c]{@{}c@{}} Bitcoin \\ Gold \end{tabular}}       & 16/5/18 \cite{btg-attack-2}       & \$$17.5$M     & \$$11,447$     & \$$-1,404$    & $0.441631$ \\
                                                & 23/1/20 \cite{btg-attack}         & \$$72,000$    & \$$4,247$      & \$$814$   & $0.43991$ \\
            \hline

            Feathercoin                         & 1/6/13 \cite{ftc-attack}         & \$$63,800$    & \$$1,203$      & \$$-95.73$   & $0.48283$ \\
            \hline

            \begin{tabular}[c]{@{}c@{}} Litecoin \\ Cash \end{tabular}                       & 4/7/19 \cite{lcc-attack}          & \$$5,511$     & \$$80.66$         & \$$-0.15$    & $0.47167$ \\
            \hline

        \end{tabular}
      \normalsize
    \end{center}
    \caption{
        Double spending attacks and the market's response to them.
        External utility is estimated as the reward from double-spent
        transactions. To compute the reward difference, we multiply the rewards
        from reorganized blocks with the exchange rate difference, \ie the
        asset's price $5$ days after the attack minus the expected price, if an
        attack had not occurred (following Bitcoin's price in the same period).
    }
    \label{tab:attacks}
\end{table}
