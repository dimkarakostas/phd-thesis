\subsection{Bitcoin}\label{subsec:bitcoin}

In this section, we study the Bitcoin~\cite{nakamoto2008bitcoin} blockchain
protocol. Bitcoin is a prime example of a family of protocols that links the
amount of valid blocks, that each party can produce per execution, with the
party's hardware capabilities. This family includes:
\begin{inparaenum}[i)]
    \item Proof-of-Work-based protocols like Ethereum~\cite{wood2014ethereum},
        Bitcoin NG~\cite{eyal2016bitcoin}, or Zerocash~\cite{SP:BCGGMT14};
    \item Proof-of-Space~\cite{C:DFKP15} and
        Proof-of-Space-Time~\cite{C:MorOrl19} protocols like
        SpaceMint~\cite{FC:PKFGAP18} and Chia~\cite{AC:AACKPR17,EC:CohPie18}.
\end{inparaenum}

\paragraph{Execution Details}
Typically, protocols from the aforementioned family enforce that, when all parties follow the protocol honestly, the expected percentage of blocks
created by a party $\party$ is $\miningpower_{\party}$ of the total blocks
produced by all parties during the execution. Along the lines of the formulation in~\cite{EC:GarKiaLeo15,C:GarKiaLeo17}, in the Bitcoin
protocol, each party $\party$ can make at most $\miningpower_{\party}
\cdot \powQueryNum$ queries to the hashing oracle $\oracle_\proto$ per time slot, where $q$ is the total number of queries that all parties can make to $\oracle_\proto$ during a time slot. We note that when $\party$ follows the Bitcoin protocol, they perform exactly $\miningpower_{\party} \cdot\powQueryNum$ queries to the hashing oracle.
Each query can be seen as an independent block production trial and is
successful with probability $\difficulty$, which is a
protocol-specific ``mining difficulty'' parameter.

From the point of view of the observer $\observer$, a party $\party$ is rewarded a
fixed amount $\reward$ for each block they contribute to the chain output by
$\observer$.
Then, Bitcoin implements a special case of block-proportional rewards (cf.
Definition~\ref{def:proportional-rewards}), such that:
\begin{itemize}
    \item The total number of rewards for $\executionTrace$ is
        \[\totalReward_{\observer,\executionTrace} = \big|\chain_{\observer,\executionTrace}\big| \cdot \reward=\Big(\sum_{\hat{\party} \in \partySet} \msgOutputSet_{\hat{\party},\executionTrace}\Big)\cdot \reward\;,\]
        where $|\cdot|$ denotes the length of a chain in blocks.
    \item The proportional reward function $\proportionalRewardFunc(\cdot,\cdot)$ is defined as
        \[\proportionalRewardFunc\big(\chain_{\observer,\executionTrace}, \party\big) =\frac{\msgOutputSet_{\party,\executionTrace}}{\big|\chain_{\observer,\executionTrace}\big|}=\frac{\msgOutputSet_{\party,\executionTrace}}{\sum_{\hat{\party} \in \partySet} \msgOutputSet_{\hat{\party},\executionTrace}}\]
\end{itemize}
\noindent Thus, by Definition~\ref{def:proportional-rewards}, we have that
\begin{equation}\label{eq:bitcoin_reward}
    \forall \executionTrace \; \forall \party \in \partySet: \reward_{\party, \executionTrace} = \proportionalRewardFunc(\chain_{\observer,\executionTrace}, \party) \cdot \totalReward_{\observer,\executionTrace}=\msgOutputSet_{\party,\executionTrace}\cdot \reward\;.
\end{equation}

In Bitcoin, on each time slot a party keeps a local chain,
which is the longest among all available chains. If multiple longest chains
exist, the party follows the chronological ordering of messages.

Following, we assume that none of the participating parties has complete control
over message delivery, \ie apart from the preference index (cf. Section~\ref{sec:network-model}). Therefore, when two parties $\party, \party'$
produce blocks for the same index on the same time slot, it may be unclear
which is adopted by third parties that follow the protocol.

Furthermore, the \emph{index} of each block $\mesg$ is an integer that identifies
the distance of $\mesg$ from $\genesis$, \ie its height in the tree of
blocks. Blocks on the same height, but different branches, have the same
index but are non-equivalent (recall, that two messages are equivalent if their
hash is equal).

\paragraph{Bitcoin is a Compliant Protocol \wrt Reward}
We prove that Bitcoin is a $(\Theta(\delta^2), \infractionPredicate)$-compliant
protocol under the utility \emph{Reward}, where $\infractionPredicate$ is any
associated infraction predicate. By Definition~\ref{def:utility} and
Eq.~\eqref{eq:bitcoin_reward}, we have that when parties follow the strategy
profile $\sigma$, the utility $U_{\party}$ of party $\party$ is
\begin{equation}\label{eq:bitcoin_utility}
    U_{\party}(\sigma)=E\big[\msgOutputSet_{\party,\execution_\sigma}\big]\cdot \reward\;,
\end{equation}
where $\msgOutputSet_{\party,\execution_\sigma}$ is the (random variable)
number of blocks produced by $\party$ in the chain output by $\observer$ and
$\reward$ is the fixed amount of rewards per block. Our analysis considers
typical values of the success probability $\delta$, sufficiently small such
that $\delta\cdot q < 1$ (recall that $q$ is the total number of oracle queries
available to all parties per slot).

We say that party $\party$ is \emph{successful} during time slot $\slot$, if
$\party$ manages to produce at least one block, \ie at least one oracle query
submitted by $\party$ during $\slot$ was successful. The time slot $\slot$ is
\emph{uniquely successful} for $\party$, if no other party than $\party$
manages to produce a block in $\slot$. We prove the following lemma.

\begin{lemma}\label{lem:unique_succ}
    Assume an execution trace $\executionTrace$ of the Bitcoin protocol where
    all parties follow the honest strategy. Let $\block_1,\ldots,\block_k$ be a
    sequence of blocks produced by party $\party\in\mathbb{P}$ during a time
    slot $\slot$ that was uniquely successful for $\party$ in
    $\executionTrace$. Then, $\block_1,\ldots,\block_k$ will be included in the
    chain output by the observer $\observer$.
\end{lemma}
\begin{proof}
    Let $h$ be the height of $\block_1$. Then, for every $j\in\{1,\ldots,k\}$,
    the height of the block $\block_j$ is $h+j-1$.  Assume for the sake of
    contradiction that there is a $j^*\in[k]$ such that $\block_{j^*}$ is not
    in the observer's chain. Since each block contains the hash of the previous
    block in the chain of $\observer$, the latter implies that the subsequence
    $\block_{j^*},\ldots,\block_k$ is not in the observer's chain. There are
    two reasons that $\block_{j^*}$ is missing from the observer's chain.
    \begin{enumerate}
    \item The observer $\observer$ never received $\block_{j^*}$. However,
        after the end of time slot $\slot$, $\observer$ will be activated and
            fetch the messages included in its $\textsc{Receive}_\observer()$
            string. Therefore, the case that $\observer$ never received
            $\block_{j^*}$ cannot happen.

    \item The observer has another block $\block'_{j^*}$ included in its chain
        that has the same height, $h+j^*-1$, as $\block_{j^*}$. Since $\slot$
            was uniquely successful for $\party$ in $\executionTrace$, the
            block $\block'_{j^*}$ must have been produced in a time slot
            $\slot'$ that is different than $\slot$. Assume that $\party$
            produced the block sequence $\block'_{j^*},\ldots,\block'_{k'}$
            during $\slot'$. We examine the following two cases:
            \begin{enumerate}
                \item $\slot>\slot'$: then $\block'_{j^*}$ was produced before
                    $\block_{j^*}$, so in time slot $\slot'+1$ all parties
                    received (at least) the sequence
                    $\block'_{j^*},\ldots,\block'_{k'}$. All parties select the
                    longest chain, so the chain that they will select will have
                    at least $h+k'-1\geq h+j^*-1$ number of blocks in
                    $\slot'+1$. Thus, for time slot $\slot\geq \slot'+1$ the
                    parties submit queries for producing blocks which height is
                    at least $h+k'>h+j^*-1$. So at time slot $\slot$, the party
                    $\party$ cannot have produced a block which height is $\leq
                    h+j^*-1$.
                \item $r<r'$: then $\block_{j^*}$ was produced before
                    $\block'_{j^*}$. So in time slot $r+1$ all parties receive
                    the sequence $\block_1,\ldots,\block_k$. Thus, they will
                    adopt a chain with at least $h+k-1\geq h+j^*-1$ number of
                    blocks. Thus, for time slot $r'\geq r+1$ the parties submit
                    queries for producing blocks with height at least
                    $h+k>h+j^*-1$. Therefore, $\party$ cannot have produced a
                    block of height $\leq h+j^*-1$ during time slot $r'$.
            \end{enumerate}
    \end{enumerate}
By the above, $\block_{j^*}$ is a block with height $h+j^*-1$ received by
    $\observer$ and no other block with height $h+j^*-1$ is included in
    $\observer$'s chain. Since $\observer$ adopts the longest chain, there must
    be a block with height $h+j^*-1$ that is included in its chain. It is
    straightforward that this block will be $\block_{j^*}$, which leads to
    contradiction.
\end{proof}
Subsequently, we apply Lemma~\ref{lem:unique_succ} to the proof of the main theorem of this subsection, stated below.

\begin{theorem}\label{thm:bitcoin_eq_approx}
Let $N$ be the number of time slots of the execution and $\infractionPredicate$
    be any associated infraction predicate. Let $\bar{U}$ be the utility vector
    where each party employs the utility Reward. Then, the Bitcoin protocol is
    $(\epsilon,\infractionPredicate)$-compliant \wrt $\bar{U}$, for
    $\epsilon:=\frac{NRq^2}{2}\delta^2$ .
\end{theorem}

\begin{proof}
We will show that the Bitcoin protocol is an $\epsilon$-Nash equilibrium \wrt $\bar{\utility}$, for $\epsilon:=\frac{NRq^2}{2}\delta^2$. By Proposition~\ref{prop:eq_comp}, the latter implies that the Bitcoin protocol is $(\epsilon,\infractionPredicate)$-compliant \wrt $\bar{\utility}$.

    Consider a protocol execution where all parties follow the honest strategy $\Pi$, with $\profile_\proto$ denoting the profile $\langle\Pi,\ldots,\Pi\rangle$. Let $\party$ be a party and $\mu_{\party}$ be its mining power.
For $\slot\in[N]$, let $X_{\party,\slot}^{\profile_\proto}$ be the random variable that is $1$ if the time slot $\slot$ is uniquely successful for $\party$ and $0$ otherwise. By protocol description, a party $\party'$ makes $\mu_{\party'} \cdot q$ oracle queries during $\slot$, each with success probability $\delta$. Thus:
\begin{equation}\label{eq:pr_unique}
\begin{split}
&\Pr[X_{\party,r}^{\profile_\proto}=1]=\\
=&\Pr[\party\mbox{ is successful during }\slot]\cdot\\
&\quad\quad\cdot\Pr[\mbox{all the other parties produce no blocks in } \slot]=\\
    =&\Big(1-(1-\delta)^{\mu_{\party} q}\Big)\cdot\prod_{\party'\neq\party}(1-\delta)^{\mu_{\party'} q}=\\
    =&\Big(1-(1-\delta)^{\mu_{\party} q}\Big)\cdot(1-\delta)^{(1-\mu_{\party}) q}
    =(1-\delta)^{(1-\mu_{\party}) q}-(1-\delta)^q\;.
\end{split}
\end{equation}

    The random variable $X_{\party,\execution_{\profile_\proto}}:=\sum_{\slot\in[N]}X_{\party,r}^{\profile_\proto}$ expresses the number of uniquely successful time slots for $\party$. By Eq.~\eqref{eq:pr_unique}, $X_{\party}^\Pi$ follows the binomial distribution with $N$ trials and probability of success $(1-\delta)^{(1-\mu_{\party}) q}-(1-\delta)^q$. Therefore:
    $E\big[X_{\party,\execution_{\profile_\proto}}\big]=N\Big((1-\delta)^{(1-\mu_{\party}) q}-(1-\delta)^q\Big)$.

Let $\msgOutputSet_{\party,\execution_{\profile_\proto}}$ be the number of blocks produced by $\party$ included in the chain output by the observer $\observer$. In a uniquely successful time slot $\slot$, $\party$ produces at least one block, and by Lemma~\ref{lem:unique_succ}, all the blocks that $\party$ produces during $\slot$ will be included in the chain output by the observer. Therefore, for all random coins $\msgOutputSet_{\party,\execution_{\profile_\proto}}\geq X_{\party,\execution_{\profile_\proto}}$ and so it holds that:
\begin{equation}\label{eq:lower_bound}
    E\big[\msgOutputSet_{\party,\execution_{\profile_\proto}}\big]\geq E\big[X_{\party,\execution_{\profile_\proto}}\big]=N\Big((1-\delta)^{(1-\mu_{\party}) q}-(1-\delta)^q\Big)\;.
\end{equation}

Now assume that $\party$ decides to unilaterally deviate from the protocol, following a strategy $S$. Let $\profile$ denote the respective strategy profile.
    Let $Z_{\party,\execution_\profile}$ be the number of blocks that $\party$ produces by following $S$ and $\msgOutputSet_{\party,\execution_\profile}$ be the number of blocks produced by $\party$ that will be included in the chain output by $\observer$. Clearly, for all random coins $\msgOutputSet_{\party,\execution_\profile}\leq Z_{\party,\execution_\profile}$. Without loss of generality, we may assume that $\party$ makes all of their $N\mu_{\party} q$ available oracle queries (indeed, if $\party$ made less than $N\mu_{\party} q$ queries then on average it would produce less blocks). Thus, we observe that $Z_{\party,\execution_\profile}$ follows the binomial distribution with $N\mu_{\party} q$ trials and probability of success $\delta$. Thus:
\begin{equation}\label{eq:upper_bound}
    E\big[\msgOutputSet_{\party,\execution_\profile}\big]\leq\big[Z_{\party,\execution_{\profile}}\big]=N\mu_{\party} q\delta\;.
\end{equation}

    By definition of $\utility_{\party}$ in Eq.~\eqref{eq:bitcoin_utility} and Eq.~\eqref{eq:lower_bound} and~\eqref{eq:upper_bound}, for fixed block rewards $R$ and for every strategy $S$ that $\party$ may follow, we have:
\begin{equation}\label{eq:upper-lower}
\begin{split}
    \utility_{\party}(\profile)&-\utility_{\party}(\profile_\proto)=E\big[\msgOutputSet_{\party,\execution_\profile}\big]\cdot R-E\big[\msgOutputSet_{\party,\execution_{\profile_\proto}}\big]\cdot R\leq\\
    &\leq N\mu_{\party} q\delta R- N\Big((1-\delta)^{(1-\mu_{\party}) q}-(1-\delta)^q\Big)R\;.
\end{split}
\end{equation}
By Bernoulli's inequality, we have that:
    $(1-\delta)^{(1-\mu_{\party}) q}\geq1-(1-\mu_{\party}) q\delta$.
Besides, by binomial expansion, and the assumption that $\delta\cdot q<1$ we have that:
$(1-\delta)^q\leq1-q\delta+\frac{q^2}{2}\delta^2$.
Thus, by applying the two above inequalities in Eq.~\eqref{eq:upper-lower}, we get that:
\begin{equation}\label{eq:btc_approx}
\begin{split}
    &\utility_{\party}(\profile)-\utility_{\party}(\profile_\proto)\leq N\mu_{\party} q\delta R- N\Big((1-\delta)^{(1-\mu_{\party}) q}-(1-\delta)^q\Big)R\leq\\
    \leq&N\mu_{\party} q\delta R- N\Big(1-(1-\mu_{\party}) q\delta-\Big(1-q\delta+\frac{q^2}{2}\delta^2\Big)\Big)R=\\
    =&N\mu_{\party} q\delta R-N\Big(\mu_{\party} q\delta-\frac{q^2}{2}\delta^2\Big)R= \frac{NRq^2}{2}\delta^2
\end{split}
\end{equation}

By Eq.~\eqref{eq:btc_approx}, Bitcoin is an $\epsilon$-Nash equilibrium for $\epsilon:=\frac{NRq^2}{2}\delta^2$.
\end{proof}

The result of Theorem~\ref{thm:bitcoin_eq_approx} shows that Bitcoin \wrt
rewards is compliant, by proving that it is an approximate Nash equilibrium.
This is in agreement with previous
results~\cite{KrollDaveyFeltenWEIS2013,kiayias2019coalitionsafe}. Similar results
exist for Bitcoin \wrt profit~\cite{EC:BGMTZ18,kiayias2019coalitionsafe},
therefore compliance results for this utility are expected to also be achievable.

\begin{remark*}
A well-known result from the Selfish Mining
attack~\cite{FC:EyaSir14,FC:SapSomZoh16} is that Bitcoin is not an equilibrium
with respect to relative rewards. Therefore, we stress that the above analysis
concerns the utility of \emph{absolute} rewards, although evaluating compliance
under relative rewards is an interesting open question.
\end{remark*}
