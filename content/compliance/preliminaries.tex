\section{The Setting}

We consider a distributed protocol $\proto$, which is executed by all parties in
a set $\partySet$, and which is performed over a number of time slots. Our
analysis is restricted on executions where every party $\party \in \partySet$
is activated on each time slot. The activation is scheduled by an environment
$\env$, which also provides the parties with inputs.

We use $\secparam$ to denote $\proto$'s security parameter, and
$\mathsf{negl}(\cdot)$ to denote that a function is negligible, i.e.,
asymptotically smaller than the inverse of any polynomial. By $[n]$, we denote
the set $\{1,\ldots,n\}$. The expectation of a random variable $X$ is denoted
by $E[X]$.

\subsection{Network Model}\label{sec:network-model}

We assume a peer-to-peer network. Specifically, parties do not communicate via
point-to-point connections, but rather use a variant of the \emph{diffuse}
functionality defined in~\cite{EC:GarKiaLeo15}, described as follows.

\paragraph{Diffuse Functionality}\label{par:diffuse}
The functionality initializes a variable $\emph{slot}$ to $1$, which is
readable from all parties. In addition, it maintains a string
$\textsc{Receive}_{\party}()$ for each party $\party$. Each party $\party$ is
allowed to fetch the contents of $\textsc{Receive}_{\party}()$ at the beginning
of each time slot. To diffuse a (possibly empty) message $\mesg$, $\party$
sends to the functionality $\mesg$ and an integer index $i_\mesg$, which
indicates the message delivery priority of $\mesg$ (see below). In turn,
the functionality records $\mesg$ and $i_\mesg$. On each slot, every party completes its
activity by sending a special $\textsc{Complete}$ message to the functionality.
When all parties submit $\textsc{Complete}$, the functionality delivers the
messages, which are diffused during this slot, as follows.  First, it groups
all messages based on their associated index and sorts them, with messages with
smaller index having higher priority. Then, it randomizes the order of each
group's messages, \ie which are associated with the same index.  Therefore,
eventually all messages are sorted in a well-defined order.  Subsequently, the
functionality includes all messages in the $\textsc{Receive}_{\party}()$ string
of every party $\party \in \partySet$, following the aforementioned order.
Hence, all parties receive all diffused messages, without any information
on each message's creator. Finally, the functionality increases the value of $\emph{slot}$
by $1$.

\paragraph{Lossy Diffuse Functionality}\label{sec:lossy-network}
The lossy diffuse functionality is similar to the above variant, with one
difference: it is parameterized with a probability $\networkLossProb$, which
defines the probability that a message $\mesg$ is dropped, \ie it is not
delivered to any recipient. This functionality aims to model the setting where
a network with stochastic delays is used by an application, where users reject
messages which are delivered with delay above a (protocol-specific) limit. For
example, various protocols, like Bitcoin~\cite{nakamoto2008bitcoin}, resolve
message conflicts based on the order of delivery; thus, delaying a message for
long enough, such that a competing message is delivered beforehand, is
equivalent to dropping the message altogether.

\subsection{Approximate Nash Equilibrium}\label{sec:equilibrium}

An approximate Nash equilibrium is a common tool for expressing a solution to a
non-cooperative game involving $\totalParties$ parties $\party_1,\ldots,\party_n$. Each party $\party_i$
employs a strategy $\strategy_i$. The strategy is a set of rules and actions
the party makes, depending on what has happened up to any point in the game,
\ie it defines the part of the entire distributed protocol $\proto$ performed
by $\party_i$. There exists an ``honest'' strategy, defined by $\proto$, which
parties may employ; for ease of notation,
$\proto$ denotes both the distributed protocol and the honest strategy.
A \emph{strategy profile} is a vector of all players' strategies.

Each party $\party_i$ has a game \emph{utility} $\utility_i$, which is a real
function that takes as input a strategy profile. A strategy profile is an
$\epsilon$-Nash equilibrium when no party can increase its utility more than
$\epsilon$ by \emph{unilaterally} changing its strategy
(Definition~\ref{def:equilibrium}).

\begin{definition}\label{def:equilibrium}
    Let $\epsilon$ be a non-negative real number and $\strategySet$ be the set of all strategies a party may employ.
    Also let $\profile^* = (\strategy^*_i, \strategy^*_{-i})$ be a strategy profile of
    $\partySet$, where $\strategy^*_i$ is the strategy followed by $\party_i$ and $\strategy^*_{-i}$ denotes the $\totalParties - 1$
    strategies employed by all parties except $\party_i$. We say that
    $\profile^*$ is an \emph{$\epsilon$-Nash equilibrium} \wrt a utility vector $\bar{\utility} = \langle \utility_1, \ldots, \utility_\totalParties \rangle$ if:
    $$\forall \party_i \in \partySet \; \forall \strategy_i \in \strategySet \setminus \{ \strategy^*_i \} : \utility_i(\strategy^*_i, \strategy^*_{-i}) \geq \utility_i(\strategy_i, \strategy^*_{-i}) - \epsilon$$
\end{definition}
