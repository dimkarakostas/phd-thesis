\subsubsection{Multi-Leader Proof-of-Stake}\label{subsec:multi-leader-pos}

We now turn to Ouroboros Praos~\cite{EC:DGKR18}, the representative protocol of
a family that includes Ouroboros Genesis~\cite{CCS:BGKRZ18},
Peercoin~\cite{king2012ppcoin}, and Tezos' baking system~\cite{tezos-pos}. These
are similar to the previous PoS family (cf.
Section~\ref{subsec:single-leader-pos}), but with one difference: it is
possible that multiple parties are chosen as leaders for the same time slot.
As Theorem~\ref{thm:compliant-praos} shows, Ouroboros Praos and the other
members of this family are not compliant. The core argument of the proof is the
same as with Ouroboros under a lossy network
(Theorem~\ref{thm:compliant-ouroboros-lossy}). Specifically, assuming
a randomized message delivery, a party is incentivized to produce multiple
blocks to decrease the probability that a conflicting block, produced by a
fellow slot leader, is adopted over their own.
We note that, neither Ouroboros nor Ouroboros Praos enforce a choice policy in
case of conflicting messages. However, parties typically utilize network
delivery, opting for the message that arrives first.

\begin{theorem}\label{thm:compliant-praos}
    Assume
    \begin{inparaenum}[i)]
        \item a synchronous network (cf.
            Section~\ref{sec:network-model}),
        \item the conflicting infraction predicate $\infractionPredicate_{conf}$
            (Definition~\ref{def:blockchain-infraction}),
        \item that $\forall \party' \in \partySet: \miningpower_{\party'} <
            \frac{1}{2}$, and
        \item that $\party$ is the party with maximum $\sum_{j \in [1, \epoch]} \miningpower_{\party, j}$ over all parties.
    \end{inparaenum}

    Ouroboros Praos with block-proportional rewards (Definition~\ref{def:proportional-rewards}, for fixed block reward $\reward$) is \emph{not} $(\epsilon, \infractionPredicate_{conf})$-compliant (Definition~\ref{def:compliant}) \wrt utility \emph{Reward} for (non-negligible)
    $\epsilon < \frac{1}{2} p_l \cdot \reward \cdot \sum_{j \in [1, \epoch]} \epochLength \cdot \miningpower_{\party,j}$,
    where $p_l$ is the (protocol-dependent) probability that $2$ leaders are
    elected for the same slot, and is \emph{not} $(\epsilon,
    \infractionPredicate_{conf})$-compliant \wrt utility \emph{Profit} for any
    $\epsilon < (\frac{1}{6} p_l \cdot \reward - \cost) \cdot \sum_{j \in [1, \epoch]} \epochLength \cdot \miningpower_{\party,j}$.
\end{theorem}
\begin{proof}
    As with Theorem~\ref{thm:compliant-ouroboros-lossy}, we will define a bad
    event $E$, during which the expected rewards of party $\party$ are less
    if following $\proto$, compared to a non-compliant strategy.

    Let $\slot$ be a slot during which $\party$ is leader. Additionally, a
    different party $\party'$ is also a leader of $\slot$. Both parties create
    two different messages and both set the network priority
    to the maximum (cf. Section~\ref{sec:network-model}); we note that typically the
    protocol instructs the parties to diffuse the messages as soon as possible,
    which is equivalent to setting maximum priority.
    Following, the
    diffuse functionality delivers $\block'$ before $\block$, therefore all
    parties (possibly except $\party$) adopt $\block'$ and ignore $\block$.

    Let $p_E$ be the probability that $E$ occurs. First, $p_E$ depends on the
    probability $p_l$ that multiple leaders exist alongside $\party$; $p_l$
    depends on the protocol's leader schedule functionality, but should
    typically be non-negligible. Second, it depends on the order delivery of
    $\block, \block'$; since both have the same priority and the delivery is
    randomized, (cf. Section~\ref{sec:network-model}), the probability $p_n$
    that $\block'$ is delivered before $\block$ is $p_n = \frac{1}{2}$.
    Therefore, it holds $p_E = p_l \cdot p_n = \frac{1}{2} p_l$, which is
    non-negligible.

    Additionally, if $\sum_{j \in [1, \epoch]} \epochLength \cdot
    \miningpower_{\party,j}$ is the number of slots during which $\party$ is
    leader, the total expected rewards of $\party$, when everybody (including
    $\party$) follows $\proto$, are at most
    $(1 - \frac{1}{2} p_l) \cdot \reward \cdot \sum_{j \in [1, \epoch]} \epochLength \cdot \miningpower_{\party,j}$.

    Now, consider the following (non-compliant) strategy, which is employed only by $\party$. When $\party$ is slot
    leader, it produces $t > 1$ blocks, which it diffuses with maximum
    priority. If every party $\party' \neq \party$ follows $\proto$, the
    probability $p_l$ remains the same as before. Therefore, if $\party, \party'$ are both leaders,
    the probability that the following slot's leader adopts the block of $\party'$ is $\frac{1}{t+1}$.
    So, the total expected
    rewards of $\party$ under the new strategy are
    $(1 - \frac{1}{t+1} p_l) \cdot \reward \cdot \sum_{j \in [1, \epoch]} \epochLength \cdot \miningpower_{\party,j}$.
    Since $\frac{1}{t+1} < \frac{1}{2}$, the new strategy profile is directly
    reachable from the $\profile_\proto$, when every party follows $\proto$.
    Additionally, as $t \rightarrow \infty$, the expected rewards tend to the
    (maximum) value $\reward \cdot \sum_{j \in [1, \epoch]} \epochLength \cdot
    \miningpower_{\party,j}$.

    Regarding \emph{Profit}, for $t = 2$ the increase in expected rewards is:
    $((1 - \frac{1}{3} p_l) - (1 - \frac{1}{2} p_l)) \cdot \reward \cdot \sum_{j \in [1, \epoch]} \epochLength \cdot \miningpower_{\party,j} \Rightarrow
    \frac{1}{6} p_l \cdot \reward \cdot \sum_{j \in [1, \epoch]} \epochLength \cdot \miningpower_{\party,j}$,
    thus the bound for $\epsilon$ follows with the same reasoning as
    Theorem~\ref{thm:compliant-ouroboros-lossy}.
\end{proof}
