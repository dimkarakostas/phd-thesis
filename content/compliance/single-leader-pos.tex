\subsubsection{Single-Leader Proof-of-Stake}\label{subsec:single-leader-pos}

As before, we analyze a representative of a family of protocols, in
this case the PoS protocol Ouroboros~\cite{C:KRDO17}.
This family includes systems like
EOS\footnote{\url{https://developers.eos.io/welcome/latest/protocol/consensus_protocol}},
Steem\footnote{\url{https://steem.com/}}, and Ouroboros
BFT~\cite{EPRINT:KiaRus18}.
We again utilize the blockchain infraction predicates (cf.
Definition~\ref{def:blockchain-infraction}).
As shown in Section~\ref{sec:universal}, fair rewards do not necessarily
guarantee compliance; as Ouroboros itself does not define rewards, we now
consider the same block-proportional rewards (cf.
Definition~\ref{def:proportional-rewards}) as Bitcoin (\ie fixed rewards per block).

On each slot, Ouroboros defines a single party, the ``slot leader'', as
eligible to create a valid message. Specifically, the protocol restricts that
a leader cannot extend the chain with multiple blocks for the same slot,
therefore all honest parties extend their chain by at most $1$ block per slot.
The leader schedule is public and is computed at the
beginning of each epoch via a secure, publicly verifiable Multi-Party
Computation (MPC) sub-protocol, which cannot be biased by any single party. To
prevent long-range attacks~\cite{buterin2014stake}, Ouroboros employs a form of
rolling checkpoints (``a bounded-depth longest-chain rule''~\cite{C:KRDO17}),
\ie a party ignores forks that stem from a block older than a
(protocol-specific) limit from the adopted chain's head.

Ouroboros is a representative of a family of protocols that demonstrates the following properties:
\begin{itemize}
    \item the execution is organized in epochs;
    \item within each epoch, a single party (the \emph{leader}) is eligible to
        produce a message per index;
    \item a party which is online considers the blocks of each past epoch
        finalized (\ie does not remove them in favor of a competing, albeit
        possibly longer, chain);
    \item no single party with power less than $\frac{1}{2}$ can bias the
        epoch's leader schedule.
\end{itemize}


\paragraph{Synchronous network}
First, we assume a synchronous network (cf.
Section~\ref{sec:network-model}).
Theorem~\ref{thm:compliant-ouroboros-synchronous} shows that Ouroboros with
block-proportional rewards under synchronous networks is $(\epsilon,\infractionPredicate)$-compliant, $\infractionPredicate$ being any associated infraction predicate, for negligible $\epsilon$; this result is in line with the informal
incentives' analysis of Ouroboros~\cite{C:KRDO17}.

\begin{theorem}\label{thm:compliant-ouroboros-synchronous}
    Assume
    \begin{inparaenum}[i)]
        \item a synchronous network (cf.
            Section~\ref{sec:network-model}),
        \item any associated infraction predicate $\infractionPredicate$, and
        \item that $\forall \party \in \partySet: \miningpower_{\party} <$~$\frac{1}{2}$.
    \end{inparaenum}
    Ouroboros with block-proportional rewards (cf.
    Definition~\ref{def:proportional-rewards}, for fixed block reward $\reward$) is $(\epsilon, \infractionPredicate)$-compliant (cf.
    Definition~\ref{def:compliant}) \wrt utility \emph{Reward} (cf.
    Definition~\ref{def:utility}) and,
    if $\reward > \cost$, it is also $(\epsilon, \infractionPredicate)$-compliant
    \wrt utility \emph{Profit}, in both cases for negligible $\epsilon$.
\end{theorem}
\begin{proof}
    To prove the theorem, it suffices to show that, if the assumptions hold,
    Ouroboros is an $\epsilon$-Nash equilibrium (cf. Proposition~\ref{prop:eq_comp}), \ie no
    party can increase its reward more than $\epsilon$ by unilaterally deviating from the protocol, where $\epsilon=\mathsf{negl}(\secparam)$.

    First, if all parties control a minority of staking power, no single
    party can bias the slot leader schedule for any epoch (unless with $\mathsf{negl}(\secparam)$ probability). Therefore,
    the (maximum) expected number of slots for which each party $\party$ is
    leader is $\sum_{j \in [1, \epoch]} \epochLength \cdot
    \miningpower_{\party,j}$, where $\miningpower_{\party,j}$ is the
    percentage of staking power of $\party$ during the $j$-th epoch.

    Second, if all parties follow $\proto$, then the total expected rewards for
    each party $\party$ are $\reward \cdot \sum_{j \in [1, \epoch]}
    \epochLength \cdot \miningpower_{\party,j}$. This is a direct consequence
    of the network synchronicity assumption. Specifically, on slot $\slot$ the
    (single) leader $\party$ creates exactly one block $\block$, which extends the longest
    chain (adopted by $\party$).
    At the beginning of slot $\slot + 1$, all other parties receive
    $\block$ and, since $\block$ is now part of the (unique) longest chain, all
    parties adopt it. Consequently, all following leaders will extend the chain
    that contains $\block$, so eventually $\block$ will be in the chain output
    by $\observer$.
    Therefore, if all parties follow the protocol and no party can bias the
    leader schedule, then no party can increase its expected rewards by
    deviating from the protocol.

    Regarding profit, a leader creates a block by performing a single
    query to $\oracle_\proto$. Additionally, cost depends only on the number of
    such queries. Therefore, if the cost of performing a single query is less
    than $\reward$, then the profit per slot is larger than $0$, so abstaining
    from even a single slot reduces the expected aggregate profit; therefore,
    all parties are incentivized to participate in all slots.
\end{proof}

\paragraph{Lossy network}
Second, we assume a lossy network (cf. Section~\ref{sec:lossy-network}).
Theorem~\ref{thm:compliant-ouroboros-lossy} shows that Ouroboros with block
proportional rewards is \emph{not} compliant \wrt the conflicting infraction
predicate $\infractionPredicate_{conf}$; specifically, it shows that $\epsilon$
is upper-bounded by a large value, which is typically non-negligible.

\begin{theorem}\label{thm:compliant-ouroboros-lossy}
    Assume
    \begin{inparaenum}[i)]
        \item a lossy network with (non-negligible) parameter
            $\networkLossProb$ (cf. Section~\ref{sec:lossy-network}),
        \item the conflicting infraction predicate $\infractionPredicate_{conf}$
            (cf. Definition~\ref{def:blockchain-infraction}),
        \item that $\forall \party' \in \partySet: \miningpower_{\party'} <
            \frac{1}{2}$, and
        \item that $\party$ is the party with maximum $\sum_{j \in [1, \epoch]} \miningpower_{\party, j}$ over all parties.
    \end{inparaenum}

    Ouroboros with block-proportional rewards (cf.
    Definition~\ref{def:proportional-rewards}, for fixed block reward $\reward$) is \emph{not} $(\epsilon, \infractionPredicate_{conf})$-compliant (cf.
    Definition~\ref{def:compliant}), \wrt utility \emph{Reward} for any
    $\epsilon < \networkLossProb \cdot (1 - \networkLossProb)^2 \cdot \reward \cdot \sum_{j \in [1, \epoch]} \epochLength \cdot \miningpower_{\party,j}$, and is also
    \emph{not} $(\epsilon,
    \infractionPredicate_{conf})$-compliant, \wrt utility \emph{Profit} for any
    $\epsilon < (\networkLossProb \cdot (1 - \networkLossProb)^3 \cdot \reward - \cost) \cdot \sum_{j \in [1, \epoch]} \epochLength \cdot \miningpower_{\party, j}$.
\end{theorem}
\begin{proof}
    To prove the statement we define the following event $E$ for party
    $\party$, when all parties follow $\proto$ and $\slot$ is a slot where
    $\party$ is the leader. $\party$ extends its adopted longest chain $\chain$
    with a new block $\block$. However, $\block$ is dropped, \ie it is not
    delivered to any party $\party' \neq \party$; this occurs with probability
    $\networkLossProb$, a parameter of the lossy network. Therefore, the leader
    of slot $\slot+1$, who does not adopt $\block$, creates a different,
    ``competing'' block $\block'$, which is not dropped (with probability
    $1 - \networkLossProb$), thus all parties, except $\party$, adopt
    $\block'$. Finally, on round $\slot+2$, the slot leader $\party'' \neq
    \party$ produces a block $\block''$ which extends $\block'$ and $\block''$
    is not dropped, therefore all parties -- including $\party$ adopt, on round
    $\slot+3$, the -- now longest -- chain $\chain || \block' || \block''$.
    The probability that $E$ occurs is $\networkLossProb \cdot (1 -
    \networkLossProb)^2$, which is non-negligible. This follows from the fact
    that the probability that each block is dropped is independent.
    So, the expected rewards of
    $\party$ when everyone follows $\proto$ are at most
    $(1 - \networkLossProb \cdot (1 - \networkLossProb)^2) \cdot \reward \cdot \sum_{j \in [1, \epoch]} \epochLength \cdot \miningpower_{\party,j}$.
    However, as described above, the maximum rewards are
    $\reward \cdot \sum_{j \in [1, \epoch]} \epochLength \cdot \miningpower_{\party,j}$, \ie they are strictly larger.

    Now, consider the following strategy $\strategy$. When party $\party$
    is the slot leader, it creates $t$ messages, all extending its
    adopted longest chain; in other words, $\party$ forks the chain by
    producing $t$ competing blocks. Clearly, as long as at least one of the $t$
    blocks is delivered, event $E$ will not occur, the leader of slot $\slot+1$
    will adopt one of the blocks created by $\party$ and, eventually, $\party$
    will receive the rewards that correspond to slot $\slot$. The probability
    that all $t$ blocks are dropped is $\networkLossProb^t$, while the expected rewards are
    $(1 - \networkLossProb^t \cdot (1 - \networkLossProb)^2) \cdot \reward \cdot \sum_{j \in [1, \epoch]} \epochLength \cdot \miningpower_{\party,j}$.
    As $t$ increases, this probability tends to $0$ and the party's expected
    rewards tend to the maximum value $\reward \cdot \sum_{j \in [1, \epoch]}
    \epochLength \cdot \miningpower_{\party,j}$.

    Therefore, for utility \emph{Reward}, the profile
    $\profile_{\party, \strategy}$, under which all parties except $\party$
    employ $\proto$ and $\party$ employs $\strategy$, is directly reachable
    (cf. Definition~\ref{def:reach}) from the ``all honest'' profile
    $\profile_\proto$. Hence, since $\strategy$ is non-compliant, so is
    $\proto$, for values of $\epsilon < $.
    $\reward \cdot \sum_{j \in [1, \epoch]} \epochLength \cdot \miningpower_{\party,j} - (1 - \networkLossProb \cdot (1 - \networkLossProb)^2) \cdot \reward \cdot \sum_{j \in [1, \epoch]} \epochLength \cdot \miningpower_{\party,j} = \networkLossProb \cdot (1 - \networkLossProb)^2 \cdot \reward \cdot \sum_{j \in [1, \epoch]} \epochLength \cdot \miningpower_{\party,j}$.

    Regarding cost, we consider the party $\party$ with the maximum power
    across the execution and the case when $t = 2$, \ie the simplest case of
    non-compliant deviation. The expected rewards of $\party$ are
    $(1 - \networkLossProb^2 \cdot (1 - \networkLossProb)^2) \cdot \reward \cdot \sum_{j \in [1, \epoch]} \epochLength \cdot \miningpower_{\party,j}$,
    therefore its reward increase (instead of employing $\proto$) is
    $( (1 - \networkLossProb^2 \cdot (1 - \networkLossProb)^2) - (1 - \networkLossProb \cdot (1 - \networkLossProb)^2)) \cdot \reward \cdot \sum_{j \in [1, \epoch]} \epochLength \cdot \miningpower_{\party,j} =
    \networkLossProb \cdot (1 - \networkLossProb)^3 \cdot \reward \cdot \sum_{j \in [1, \epoch]} \epochLength \cdot \miningpower_{\party,j}$.
    Hence, given a cost $\cost$ per query, if the aggregate cost for performing
    these extra queries to $\oracle_\proto$ is less than this gain, $\party$ is
    incentivized to deviate so.
\end{proof}

The lossy network analysis is particularly of interest, because it manifested
in practice. On December 2019, Cardano released its Incentivized Testnet
(ITN)\footnote{\url{https://staking.cardano.org/}}. On the ITN, Cardano
stakeholders, \ie users who owned Cardano's currency, participated in PoS by
forming stake pools which produced blocks. The ITN used proportional rewards
and the execution followed the Ouroboros model of epochs and slots.
Specifically, each pool was elected as a slot leader proportionally to its
stake and received its proportional share of an epoch's rewards based on its
\emph{performance}, \ie the number of produced blocks compared to the number of
\emph{expected} blocks it should have produced. This incentivized pool
operators to actively avoid abstaining, \ie failing to produce a block when
elected as slot leader. However, forks started to form while the network was
unstable and lossy. This incentivized pools\footnote{For a (heated) discussion
on this issue see Reddit:
\url{https://www.reddit.com/r/cardano/comments/ekncza}} to ``clone'' their
nodes, \ie run multiple node instances in parallel, thus increasing network
connectivity, reducing packet loss, but also extending all possible forks. To
make matters worse, this solution not only perpetuated forks but also created
new ones, as clones would not coordinate and produced different blocks, even
when extending the same chain.

\begin{remark*}
Although a lossy network may render a PoS protocol
non-compliant, the same does not hold for PoW ledgers. As described in the
proof of Theorem~\ref{thm:compliant-ouroboros-lossy}, a party produces multiple
blocks per slot to maximize the probability that one of them is eventually
output by $\observer$. Notably, since the PoS protocol restricts that at most
one block extends the longest chain per slot, these blocks are necessarily
conflicting. However, PoW ledgers do not enforce such restriction; therefore, a
party would instead create multiple consecutive (instead of parallel,
conflicting) blocks, as covered in the proof of
Theorem~\ref{thm:bitcoin_eq_approx}, which yields maximal expected rewards even
under a lossy network.
\end{remark*}
