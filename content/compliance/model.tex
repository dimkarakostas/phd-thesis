\section{Compliance Model}\label{sec:model}

We assume a distributed protocol $\proto$, which is performed by a set of
parties $\partySet$ over a number of time slots.
In our analysis, each party $\party \in \partySet$ is associated with
a number $\miningpower_{\party} \in [0, 1]$. $\miningpower_{\party}$ identifies
$\party$'s percentage of participation power in the protocol, \eg its votes,
hashing power, staking power, \etc; consequently, $\sum_{\party \in \partySet}
\miningpower_{\party} = 1$.

\subsection{Basic Notions}\label{subsec:basic}
A protocol's execution $\execution_{\env, \sigma, \slot}$ at a given time slot
$\slot$ is probabilistic and parameterized by the environment $\env$ and the
strategy profile $\sigma$ of the participating parties. As discussed, $\env$
provides the parties with inputs and schedules their activation. For notation
simplicity, when $\slot$ is omitted, $\execution_{\env, \sigma}$ refers to the
end of the execution, which occurs after polynomially many time slots.

An \emph{execution trace} $\executionTrace_{\env, \sigma, \slot}$ on a time slot
$\slot$ is the value that the random variable $\execution_{\env, \sigma, \slot}$ takes for a fixed
environment $\env$ and strategy profile $\sigma$, and for fixed random coins of
$\env$, each party $\party \in \partySet$, and every
protocol-specific oracle (see below). A party $\party$'s view of an
execution trace $\executionTrace_{\env, \sigma, \slot}^{\party}$ consists of
the messages that $\party$ has sent and received until slot $\slot$. For
notation simplicity, we will omit the subscripts $\{ \env,
\sigma, \slot \}$ from both $\execution$ and $\executionTrace$, unless
required for clarity.\smallskip

The protocol $\proto$ defines two components, which are related to our
analysis: (1) the oracle $\oracle_\proto$, and (2) the ``infraction'' predicate
$\infractionPredicate$. We present these two components below.

\paragraph{The Oracle $\oracle_\proto$}
The oracle $\oracle_\proto$ provides the parties with the core functionality
needed to participate in $\proto$. For example, in a Proof-of-Work (PoW) system
$\oracle_\proto$ is the random or hashing oracle, whereas in an authenticated
Byzantine Agreement protocol $\oracle_\proto$ is a signing oracle. On each time
slot, a party can perform at most a polynomial number of queries to
$\oracle_\proto$; in the simplest case, each party can submit a single query
per slot. Finally, $\oracle_\proto$ is \emph{stateless}, \ie its random coins
are decided upon the beginning of the execution and its responses do not depend
on the order of the queries.

\paragraph{The Infraction Predicate $\infractionPredicate$}
The infraction predicate $\infractionPredicate$ abstracts the
deviant behavior that the analysis aims to capture. Specifically,
$\infractionPredicate$, when given the execution trace and a party $\party$,
responds with $1$ only if $\party$ deviates from the protocol in some
well-defined manner. Definition~\ref{def:infraction-predicate} provides the
core generic property of $\infractionPredicate$, \ie that honest
parties never deviate. For ease of notation, we will next simply write $\infractionPredicate$ unless required for clarity.
With hindsight, our analysis will
focus on infraction predicates that capture either producing conflicting
messages or abstaining from the protocol.

\begin{definition}[Infraction Predicate Property]\label{def:infraction-predicate}
    The infraction predicate $\infractionPredicate$ has the property that, for
    every execution trace $\executionTrace$ and for every party $\party \in
    \partySet$, if $\party$ employs the (honest) strategy $\proto$ then
    $\infractionPredicate(\executionTrace, \party) = 0$.
\end{definition}

We stress that Definition~\ref{def:infraction-predicate} implies that
$\infractionPredicate$ being $0$ is a necessary, but \emph{not} a sufficient
condition, for a party to be honest. Specifically, for all honest parties
$\infractionPredicate$ is always $0$, but $\infractionPredicate$ might
also be $0$ for a party that deviates from $\proto$, in such way that is not
captured by $\infractionPredicate$. In that case, we say that the party employs
an $\compliant$ strategy (Definition~\ref{def:compliant-strategy}). A strategy
profile is $\compliant$ if all of its strategies are $\compliant$; naturally,
the ``all honest'' profile $\profile_\proto$, \ie when all parties employ
$\proto$, is -- by definition -- $\compliant$.

\begin{definition}[Compliant Strategy]\label{def:compliant-strategy}
    Let $\infractionPredicate$ be an infraction predicate. A strategy $\strategy$ is
    \emph{$\mathcal{X}$-compliant} if and only if $\infractionPredicate(\executionTrace, \party) =
    0$ for every party $\party$ and for every trace $\executionTrace$ where $\party$ employs $\strategy$.
\end{definition}

\paragraph{The observer $\observer$}
We assume a special party $\observer$, the \emph{(passive)
observer}. This party does not actively participate in the execution, but it
runs $\proto$ and observes the protocol's execution. Notably, $\observer$ is
\emph{always online}, \ie it bootstraps at the beginning of the execution and
is activated on every slot, in order to receive diffused messages. Therefore,
the observer models a user of the system, who frequently uses the system but
does not actively participate in its maintenance. Additionally, at the last
round of the execution, the environment $\env$ activates only $\observer$, in
order to receive the diffused messages of the penultimate round and have a
complete point of view.

As mentioned in Section~\ref{sec:equilibrium}, we assume a utility $\utility_i$ for every party $\party_i$, so the utility vector defines the payoff values for
each party. In our examples, a party's utility depends on two parameters:
\begin{inparaenum}[i)]
    \item the party's rewards, which are distributed by the protocol, from the
        point of view of $\observer$;
    \item the cost of participation.
\end{inparaenum}

\subsection{Compliant Protocols}\label{subsec:compliant}

To define the notion of an $(\epsilon, \infractionPredicate)$-compliant protocol
$\proto$, we require two parameters:
\begin{inparaenum}[(i)]
    \item the associated infraction predicate $\infractionPredicate$ and
    \item a non-negative real number $\epsilon$.
\end{inparaenum}
Following Definition~\ref{def:compliant-strategy}, $\infractionPredicate$
determines the set of compliant strategies that the parties may follow in
$\proto$. Intuitively, $\epsilon$ specifies the sufficient gain threshold after which a party
switches strategies. In particular, $\epsilon$ is
used to define when a strategy profile $\sigma'$ is \emph{directly reachable}
from a strategy profile $\sigma$, in the sense that $\sigma'$ results from the
unilateral deviation of a party $\party_i$ from $\sigma$ and, by this
deviation, the utility of $\party_i$ increases more than $\epsilon$.
Generally, $\sigma'$ is \emph{reachable} from $\sigma$, if $\sigma'$ results
from a ``path'' of strategy profiles, starting from $\sigma$, which are
sequentially related via direct reachability. Finally, we define the \emph{cone}
of a profile $\sigma$ as the set of all strategies that are
reachable from $\sigma$, including $\sigma$ itself.

Given the above definitions, we say that $\proto$ is
$(\epsilon,\mathcal{X})$-compliant if the cone of the ``all honest'' strategy
profile $\profile_\proto$ contains only profiles that consist of $\compliant$
strategies. Thus, if a protocol is compliant, then the parties may
(unilaterally) deviate from the honest strategy only in a compliant manner, as
dictated by $\infractionPredicate$.

Formally, first we define ``reachability'' between two strategy profiles, as
well as the notion of a ``cone'' of a strategy profile \wrt the reachability
relation, and then we define a compliant protocol \wrt its associated
infraction predicate.

\begin{definition}\label{def:reach}
    Let:
    \begin{inparaenum}[i)]
        \item $\epsilon$ be a non-negative real number;
        \item $\proto$ be a protocol run by parties $\party_1, \ldots, \party_\totalParties$;
        \item $\bar{\utility}=\langle \utility_1, \ldots, \utility_\totalParties \rangle$ be a utility vector, where $\utility_i$ is the utility of $\party_i$;
        \item $\strategySet$ be the set of all strategies a party may employ.
    \end{inparaenum}
    We provide the following definitions.
    \begin{enumerate}
        \item Let $\profile, \profile' \in \strategySet^\totalParties$ be two strategy profiles where $\profile = \langle \strategy_1, \ldots, \strategy_\totalParties \rangle$ and $\profile'= \langle \strategy'_1, \ldots, \strategy'_\totalParties \rangle$. We say that $\profile'$ is \emph{directly $\epsilon$-reachable from $\profile$ \wrt $\bar{\utility}$}, if there exists $i \in [\totalParties]$ \st (i) $\forall j \in [\totalParties] \backslash \{i\}: \strategy_j = \strategy'_j$ and (ii) $\utility_i(\profile') > \utility_i(\profile) + \epsilon$.
        \item Let $\profile, \profile' \in \strategySet^\totalParties$ be two distinct profiles. We say that $\profile'$ is \emph{$\epsilon$-reachable from $\profile$ \wrt $\bar{\utility}$}, if there exist profiles $\profile_1, \ldots, \profile_k$ such that (i) $\profile_1 = \profile$, (ii) $\profile_k = \profile'$, and (iii) $\forall j \in [2, k]$ it holds that $\profile_j$ is directly $\epsilon$-reachable from $\profile_{j-1}$ \wrt $\bar{\utility}$.
        \item For every strategy profile $\profile \in \strategySet^\totalParties$ we define the \emph{$(\epsilon, \bar{\utility})$-cone of $\profile$} as the set:
        \[\mathsf{Cone}_{\epsilon, \bar{\utility}}(\profile) := \{\profile' \in \strategySet^\totalParties\;|\;(\profile' = \profile) \lor (\profile'\mbox{ is $\epsilon$-reachable from }\profile\mbox{ \wrt }\bar{\utility})\}\;.\]
    \end{enumerate}
\end{definition}

\begin{definition}\label{def:compliant}
    Let:
    \begin{inparaenum}[i)]
        \item $\epsilon$ be a non-negative real number;
        \item $\proto$ be a protocol run by the parties $\party_1, \ldots, \party_\totalParties$;
        \item $\infractionPredicate$ be an infraction predicate;
        \item $\bar{\utility}=\langle \utility_1, \ldots, \utility_\totalParties \rangle$ be a utility vector, where $\utility_i$ is the utility of party $\party_i$;
        \item $\strategySet$ be the set of all strategies a party may employ;
        \item $\strategySet_{\infractionPredicate}$ be the set of $\compliant$ strategies.
    \end{inparaenum}

    A strategy profile $\profile \in \strategySet^n$ is $\compliant$ if $\profile \in (\strategySet_{\infractionPredicate})^\totalParties$.

    The \emph{$(\epsilon,\bar{U})$-cone of $\proto$}, denoted by $\mathsf{Cone}_{\epsilon,\bar{U}}(\proto)$, is the set $\mathsf{Cone}_{\epsilon, \bar{\utility}}(\profile_\proto)$, \ie the set of all strategies that are $\epsilon$-reachable from the ``all honest'' strategy profile $\profile_\proto = \langle \proto, \ldots, \proto \rangle$ \wrt $\bar{\utility}$, also including $\profile_\proto$.

    $\proto$ is \emph{$(\epsilon,\mathcal{X})$-compliant \wrt $\bar{\utility}$} if $\mathsf{Cone}_{\epsilon,\bar{U}}(\proto) \subseteq (\strategySet_{\infractionPredicate})^\totalParties$, \ie all strategy profiles in the $(\epsilon,\bar{\utility})$-cone of $\proto$ are $\compliant$.
\end{definition}

Proposition~\ref{prop:eq_comp} shows that Definition~\ref{def:compliant} expresses a relaxation of the standard approximation Nash equilibrium (Definition~\ref{def:equilibrium}).

\begin{proposition}\label{prop:eq_comp}
    Let:
    \begin{inparaenum}[i)]
        \item $\epsilon$ be a non-negative real number;
        \item $\proto$ be a protocol run by the parties $\party_1, \ldots, \party_\totalParties$;
        \item $\infractionPredicate$ be an infraction predicate;
        \item $\bar{\utility}=\langle \utility_1, \ldots, \utility_\totalParties \rangle$ be a utility vector, with $\utility_i$ the utility of $\party_i$.
    \end{inparaenum}
    If $\proto$ is an $\epsilon$-Nash equilibrium \wrt $\bar{\utility}$ (\ie $\profile_\proto=\langle \proto, \ldots, \proto \rangle$ is an $\epsilon$-Nash equilibrium \wrt $\bar{\utility}$), then $\proto$ is $(\epsilon, \infractionPredicate)$-compliant \wrt $\bar{\utility}$.
\end{proposition}
\begin{proof}
Assume that $\profile_\proto$ is an $\epsilon$-Nash equilibrium \wrt $\bar{\utility}$ and let $\profile'=(\strategy'_1, \ldots, \strategy'_\totalParties)$ be a strategy profile. We will show that $\profile'$ is not directly $\epsilon$-reachable from $\profile_\proto$ \wrt $\bar{\utility}$.

Assume that there exists $i \in [\totalParties]$ s.t. $\forall j \in [\totalParties] \backslash\{i\}: \strategy'_j = \proto$. Since $\profile_\proto$ is an $\epsilon$-Nash equilibrium \wrt $\bar{\utility}$, it holds that $\utility_i(\profile') \leq \utility_i(\profile_\proto) + \epsilon$. Therefore, Definition~\ref{def:reach} is not satisfied and $\profile'$ is not directly $\epsilon$-reachable from $\profile_\proto$ \wrt $\bar{\utility}$.

Since no profiles are directly $\epsilon$-reachable from $\profile_\proto$ \wrt $\bar{\utility}$, it is straightforward that there are no $\epsilon$-reachable strategy profiles from $\profile_\proto$ \wrt $\bar{\utility}$. The latter implies that the $(\epsilon, \bar{\utility})$-cone of $\proto$ contains only $\profile_\proto$, \ie $\mathsf{Cone}_{\epsilon,\bar{\utility}}(\proto) = \{ \profile_\proto\}$. By Definitions~\ref{def:infraction-predicate} and~\ref{def:compliant-strategy}, $\profile_\proto$ is $\compliant$, so we deduce that $\proto$ is $(\epsilon,\infractionPredicate)$-compliant \wrt $\bar{\utility}$.
\end{proof}
