\section{Externalities}\label{sec:externalities}

In this section, we enhance our analysis with parameters external to
the distributed system. First, we introduce an exchange rate,
to model the rewards' price in the same unit of account as the cost. We
analyze how the exchange rate should behave to ensure compliance, assuming
an external utility which is awarded for an infraction. Following,
we identify historical patterns to approximate this behavior and, finally,
introduce penalties, which disincentivize infraction when the exchange rate
behavior does not suffice.

\subsection{Utility under Externalities}

In distributed ledger systems, rewards are denominated in the ledger's native
currency (\ie satoshi in Bitcoin, wei in Ethereum, etc). However, cost is
typically denominated in fiat (USD, GBP, etc). Therefore, we introduce
an \emph{exchange rate}, between the ledger's native currency and USD, to
denominate the rewards and cost in the same unit of account and precisely
estimate a party's utility.

The exchange rate $\exchangeRate_{\execution}$ is a random variable,
parameterized by a strategy profile $\profile$. For a trace
$\executionTrace_{\profile}$ under $\profile$, the exchange rate takes a
non-negative real value. The exchange rate is applied once, at the end of the
execution. Intuitively, this implies that a party eventually sells their
rewards at the end of the execution. Therefore, its utility depends on the
accumulated rewards, during the execution, and the exchange rate at the end.

The infraction predicate expresses a deviant behavior that parties may exhibit.
So far, we considered distributed protocols in a standalone fashion, analyzing
whether they incentivize parties to avoid infractions. In reality, a ledger
exists alongside other systems, and a party's utility may depend on parameters
external to the distributed ledger. For instance, double spending against
Bitcoin is a common hazard, which does not increase an attacker's \emph{Bitcoin
rewards}, but awards them external rewards, \eg goods that are purchased with
the double-spent coins.

The external -- to the ledger -- reward is modeled as a random variable
$\utilityBoost_{\party, \execution_\profile}$, which takes non-negative integer
values. Similarly to the rewards' random variable, it is
parameterized by a
party $\party$ and a strategy profile $\profile$.  The infraction utility is
applied once when computing a party's utility and has the property that, for
every trace $\executionTrace$ during which a party $\party$ performs no
infraction, it holds that $\utilityBoost_{\party, \executionTrace} = 0$.
Therefore, a party receives these external rewards only by performing an
infraction.

Taking into account the exchange rate and the external infraction-based
utility, we now define a new utility. As with
Definition~\ref{def:utility}, the new utility $\utility$ takes two forms,
\emph{Reward} and \emph{Profit}. For the former, $\utility$ normalizes the protocol
rewards by applying the exchange rate and adds the external, infraction
rewards. For \emph{Profit}, it also subtracts the cost (which is already
denominated in the base currency). Definition~\ref{def:utility-external} defines the utility under
externalities. For ease of reading, we use the following notation:
\begin{itemize}
    \item $\rewardVal_{\party, \profile} = E[\reward_{\party, \execution_\profile}]$;
    \item $\exchangeRateVal_\profile = E[\exchangeRate_{\execution_\profile}]$;
    \item $\utilityBoostVal_{\party, \profile} = E[\utilityBoost_{\party, \execution_{\profile}}]$;
    \item $\costVal_{\party, \profile} = E[\cost_{\party, \execution_\profile}]$.
\end{itemize}

\begin{definition}\label{def:utility-external}
    Let:
    \begin{inparaenum}[i)]
        \item $\profile$ be a strategy profile;
        \item $\execution_\profile$ be an execution under $\profile$;
        \item $\exchangeRateVal_{\profile}$ be the (expected) exchange rate of $\execution_\profile$;
        \item $\utilityBoostVal_{\party, \profile}$ be the (expected) external rewards of $\party$ under $\profile$.
    \end{inparaenum}
    We define two types of utility $\utility_{\party}$ of a party $\party$ for
    $\profile$ under externalities:
    \begin{enumerate}
        \item \emph{Reward}: $\utility_{\party}(\profile) = \rewardVal_{\party, \profile} \cdot \exchangeRateVal_\profile + \utilityBoostVal_{\party, \profile}$
        \item \emph{Profit}: $\utility_{\party}(\profile) = \rewardVal_{\party, \profile} \cdot \exchangeRateVal_\profile + \utilityBoostVal_{\party, \profile} - \costVal_{\party, \profile}$
    \end{enumerate}
\end{definition}
