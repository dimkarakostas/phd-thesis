\subsection{Proof-of-Stake}\label{subsec:pos}

Proof-of-Stake (PoS) systems differ from Bitcoin in a few points. Typically, the
execution of such systems is organized in \emph{epochs}, each consisting of a
fixed number $\epochLength$ of time slots. On each slot, a specified set of
parties is eligible to participate in the protocol. Depending on the protocol,
the leader schedule of each epoch may or may not be a priori public.

The core difference from PoW concerns the power $\miningpower_{\party}$ of each
party. In PoS, $\miningpower_{\party}$ represents their \emph{stake} in the
system, \ie the number of coins, that $\party$ owns. Stake is dynamic,
therefore the system's coins may change hands and the leader schedule of each
epoch depends on the stake distribution at the beginning of the
epoch.\footnote{In reality, the snapshot of the stake distribution is retrieved
at an earlier point of the previous epoch, but we can employ this simplified
version without loss of generality} As with Bitcoin, each party participates
proportionally to their power, so, on expectation, the ratio of slots for which
$\party$ is leader, over the total number of the epoch's slots, is
$\miningpower_{\party}$.

Moreover, in PoS protocols, the oracle $\oracle_\proto$ does not perform hashing
as in Bitcoin. Instead, it is parameterized by the leader schedule and
typically performs signing. The signature output by $\oracle_\proto$ is valid
if and only if the input message is submitted by the leader of the slot, during
which $\oracle_\proto$ is responding. This introduces two important
consequences: i) only the slot leader can produce valid messages during a
given slot; ii) the leader can produce as many valid messages as the number
of possible queries to $\oracle_\proto$.

In the upcoming paragraphs we use the following notation:
\begin{itemize}
    \item $\cost$: the cost of a single query to $\oracle_\proto$;
    \item $\reward$: the (fixed) reward per block;
    \item $\epoch$: the number of epochs in an execution;
    \item $\epochLength$: the number of slots per epoch;
    \item $\miningpower_{\party, j}$: the power of party $\party$ on epoch $j$.
\end{itemize}
