\subsection{Ethereum's State Efficiency}\label{sec:ethereum-fees}
\comment{TODO: fix}

Ethereum's fee model is more well-defined than Bitcoin's. In Ethereum, each
operation is associated with a fee, which is evaluated in gas\comment{TODO: preliminaries}. Different
operations cost different amounts of gas, with operations that use storage in
the ledger being more expensive.  Ethereum partially incentivizes proper state
management by offering discounts for cleaning unused variables. Specifically,
an operation that removes a variable or a smart contract from the global state
increases the ``refund balance'' of the transaction. At the end of the
transaction's execution, the account get a refund in gas fees equal to the
refund balance, capped to a maximum of half the total gas
used~\cite{wood2014ethereum}.

Ethereum's fee policy incentivizes proper state management in two ways.  First,
it disincentivizes unnecessary increase of the global state by enforcing high
fees on operations that do so. Second, it incentivizes removing items from the
global state, \ie reducing it, via the gas refund mechanism. However, this
policy is not ideal, as showcased in the following example.

Assume an account $A$ creates a smart contract $S$. To deploy $S$, $A$ needs to
create and publish a transaction which contains $S$'s code. The gas fees of
this transaction depend on both
\begin{inparaenum}[a)]
    \item the length of the contract's code (in bytes) and
    \item the operations performed by the contract's constructor, \eg creating
        global contract variables.
\end{inparaenum}

A contract can be permanently deleted via the $\mathsf{selfdestruct}$ opcode.
This opcode removes the contract from the state, transfers its balance to a
provided address, and refunds the user who calls it with $19.000$
gas\footnote{The refund is given only if the contract's balance is transferred
to an existing account. If it is transferred to an account with $0$ preexisting
balance, then the $\mathsf{selfdestruct}$ operation costs $30.000$ gas.
Finally, the $\mathsf{selfdestruct}$ operation costs less than the
$\mathsf{transfer}$ operation to incentivize its usage.}.

However, including $\mathsf{selfdestruct}$ incurs a cost on the contract's
creator. Specifically, this single line of code costs $15.000$ gas. Therefore,
if the contract's creator is not the party that runs $\mathsf{selfdestruct}$,
then the former has no incentive to include this operation and incur the cost
of this line of code.

Eventually $S$ reaches the end of its lifetime. Assume that, after its
functionality is completed, the only remaining operation that can be performed
is $\mathsf{selfdestruct}$. Observe that the $\mathsf{selfdestruct}$ refund is
at most half of the transaction's cost. In that case, no party has incentive to
destroy the contract and incur the (discounted) cost of the transaction.
