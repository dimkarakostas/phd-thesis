\section{State Efficiency in Bitcoin}\label{sec:fee_model}

We now define the \emph{state efficiency} property. Our goal is to incentivize
users to minimize the global state, without impacting the system's
functionality. In that case, if all users are rational, \ie operate following
the incentives, then the state will be minimized as much as possible. Future
work will explore the actual impact of deploying such incentives in real-world
systems.

To achieve state efficiency, a transaction's fee should be proportional to the
incurred state cost. In other words, the more a transaction increases the
ledger's state, the higher its fees should be. Specifically, a transaction's
fee should reflect:
\begin{inparaenum}[i)]
    \item the transaction's size, \ie the cost of storing a transaction
        permanently on the ledger and
    \item the transaction's state cost.
\end{inparaenum}
A distributed ledger's fee model should aim at incentivizing users
to minimizing both storage types, \ie the distributed ledger and the global
state.

First, we define the \emph{fee function} $\feeFunction$, \ie the function that
assigns an (integer) fee on a transaction, given a ledger state:
$\feeFunction: \mathit{Transaction} \rightarrow \mathit{LedgerState} \rightarrow \mathit{Int}$.
Following, Definition~\ref{def:state-efficiency} describes \emph{state
efficiency}. This property instructs the fee function to (monotonically)
increase fees, if a transaction increases the state. Intuitively, between two
equivalent transactions, the transaction that incurs greater state cost should
also incur a larger fee.

\begin{definition}\label{def:state-efficiency}
    A fee function $\feeFunction$ is \emph{state efficient} if
    \begin{align*}
        \forall \ledgerState \in \ledgerStateSet \forall \tx_1, \tx_2 \in \txSet \mid \tx_1 \equiv \tx_2 \land \mathit{costTx}(\tx_1, \ledgerState) > \mathit{costTx}(\tx_2, \ledgerState):
        \feeFunction(\tx_1, \ledgerState) > \feeFunction(\tx_2, \ledgerState)
    \end{align*}
    for transaction cost function (cf. Definition~\ref{def:tx-cost}) and
    equivalence (cf. Definition~\ref{def:equiv_txs}).
\end{definition}

Evidently, if the utility of users is to minimize transaction fees, a state
efficient fee function ensures that they are also incentivized to minimize the
global state. Finally, Definition~\ref{def:state-efficiency-same-out} sets
\emph{narrow state efficiency}, a special case of state efficiency which
compares equivalent transactions that differ only in their inputs.

\begin{definition}\label{def:state-efficiency-same-out}
    A fee function $\feeFunction$ is \emph{narrow state efficient} if
    \begin{align*}
        \forall \ledgerState \in \ledgerStateSet \forall \tx_1, \tx_2 \in \txSet \mid \\
        \tx_1 \equiv \tx_2 \land \tx_1.\mathsf{outputs} = \tx_2.\mathsf{outputs} \land \mathit{costTx}(\tx_1, \ledgerState) > \mathit{costTx}(\tx_2, \ledgerState): \\
        \feeFunction(\tx_1, \ledgerState) > \feeFunction(\tx_2, \ledgerState)
    \end{align*}
    for transaction cost function (cf. Definition~\ref{def:tx-cost}) and
    equivalence (cf. Definition~\ref{def:equiv_txs}).
\end{definition}
