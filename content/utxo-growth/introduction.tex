Distributed ledgers implement a storage layer, on top of which a shared state
is maintained in a decentralized manner. In ledger systems, this state
consists of three objects:
\begin{inparaenum}[i)]
    \item the public ledger, \ie the list of transactions which form the
        system's history;
    \item the mempool, \ie the set of, yet unpublished, transactions;
    \item the active state which, in systems like Bitcoin, consists of the UTxO
        set.
\end{inparaenum}
To support thousands (or millions) of participants, a decentralized system's
state management should be well-designed, primarily focused on minimizing the
shared state. Our work focuses on the third type, as poorly designed management
often leads to performance issues and even Denial-of-Service (DoS) attacks. In
Ethereum, during a 2016 DoS attack, an attacker added 18 million accounts to
the state, increasing its size by 18 times~\cite{ethereum-dos}. Bitcoin saw
similar spam attacks in 2013~\cite{FCW:VasThoMoo14} and
2015~\cite{bitcoin-flooding}, when millions of outputs were added to the UTxO
set.

Mining nodes and full nodes incur costs for maintaining the shared state in the
Bitcoin network. This cost pertains to the resources (\ie CPU, disk, network
bandwidth, memory) that are consumed with every transaction transmitted,
validated, and stored. An expensive part of a transaction is the newly created
outputs, which are added to the in-memory UTxO set. As the system's scale
increases, the cost of maintaining the UTxO set gradually leads to a
shared-state bloat, which makes the cost of running a full node prohibiting.

Moreover, the system's incentives, which are promoted via transaction fees,
only deteriorate the problem. For example, assume two transactions $\tx_A$ and
$\tx_B$: $\tx_A$ spends $5$ inputs and creates $1$ output, while $\tx_B$ spends
$1$ input and creates $2$ outputs. Assuming the size of a UTxO is equal to the
size of consuming it (200 bytes) and that transaction fees are $30$ satoshi per
byte, $\tx_A$ costs $30 \times 200 \times (5+1) =  36000$ satoshi and $\tx_B$
costs $30 \times 200 \times (1+2) = 18000$ satoshi. Although $\tx_B$ burdens
the UTxO set by creating a net delta of ($2 - 1 = 1$) new UTxO, while $\tx_A$
reduces the shared state by consuming ($1 - 5 = -4$) UTxOs, $\tx_B$ is cheaper
in terms of fees. Clearly, the existing fee scheme penalizes the consumption of
multiple inputs, disincentivizing minimizing the shared state.
