\paragraph{Bitcoin's State Management}\label{sec:bitcoin-fees}
Bitcoin's consensus model does not consider fees. Specifically, the user
decides a transaction's fees and the miners choose whether to include a
transaction in a block. Therefore, it has been stipulated that the level of
fees is the balance between the rational choices of miners, who supply the
market with block space, and users, who demand part of said
space~\cite{bitcoin-fees}.

In practice, users follow the client software's choice even when not
needed~\cite{FCW:MosBoh15}, \eg when blocks are not full. Similarly, miners
usually follow the hard-coded software rules and may accept even zero-fee
transactions. The reference rules of the Bitcoin Wiki~\cite{bitcoin-fees}
define the \emph{fee rate} $x$, which is the fraction of fees per transaction
size, Miners sort transactions based on this metric and solve the Knapsack
problem to fill a new block with transactions that maximize it. Some notable
alternatives also focus on fee
rate~\cite{santos2019miner,rizun2015transaction}, while reference
rules~\cite{bitcoin-fees} used to also take into account the UTxO age.

As before, a transaction consists of inputs and outputs, \ie old UTxOs which
are spent and newly-created UTxOs. Inputs and UTxOs have a fixed size
$\txInputSize$ and $\utxoSize$ respectively.\footnote{This assumption slightly
diverges from the real-world, where UTxOs are typically of varying size
depending on the operations in the ScriptPubKey.} The size of a transaction is
the sum of its inputs and outputs, \ie is a linear combination of
$\txInputSize$ and $\utxoSize$, while a transaction's cost is the difference
between the number of its UTxOs minus its inputs.
Bitcoin's fee function is $\feeFunction = \sizeByteFee \cdot
\mathsf{size}(\tx)$, where $\mathsf{size}(\tx)$ is $\tx$'s size in bytes and
$\sizeByteFee$ is a fixed fee per byte.\footnote{$\sizeByteFee = 0.0067$\$/byte
[September 2020] (\url{https://bitinfocharts.com})}

We break the fee efficiency of $\feeFunction$ via a counterexample. Assume two
transactions which are applied on the same ledger state $\ledgerState$; for
ease of notation, in the rest of the section $\feeFunction(\tx)$ denotes
$\feeFunction(\tx, \ledgerState)$. First, $\tx_1$ has $1$ input and $1$ output,
so its state cost is $\costFunc(\tx_1, \ledgerState) = 0$ and its fee is
$\feeFunction(\tx_1) = \sizeByteFee \cdot (\txInputSize + \utxoSize)$. Second,
$\tx_2$ has $2$ inputs and $1$ output, \ie its state cost is $\costFunc(\tx_2,
\ledgerState) = -1$, since it decreases the state; however, its fee is
$\feeFunction(\tx_2) = \sizeByteFee \cdot (2 \cdot \txInputSize + \utxoSize) =
\feeFunction(\tx_1) + \sizeByteFee \cdot \txInputSize$.  Thus, although
$\costFunc(\tx_1) > \costFunc(\tx_2)$, $\tx_2$'s fee is higher, since it is
larger.

A better alternative fee function is the following:
$\feeFunction' = \sizeByteFee \cdot
\mathsf{size}(\tx) + \utxoFee \cdot \costFunc(\tx, \ledgerState)$.
Note that this is state-efficient in our model  for a
sufficiently small value of $\beta$ (cf. Section~\ref{sec:state-efficient-bitcoin}).
Observe with this function, when
increasing the UTxO set, a user needs to pay an extra fee $\utxoFee$ per UTxO.
Given this change, the reference rules are updated so that, instead of the fee rate, miners use the scoring function:
$\mathsf{score}(\tx) = \frac{\mathsf{fees}(\tx) - \utxoFee \cdot \costFunc(\tx,
\ledgerState)}{\mathsf{size}(\tx)}$, where $\mathsf{fees}(\tx)$ are $\tx$'s
total fees. In market prices, $1$ byte of RAM
costs approximately $\$3.35 \cdot 10^{-9}$~\cite{ram-price}. The average size of a Bitcoin
UTxO is $61$ Bytes~\cite{FCW:DPNH18}, so a single Bitcoin UTxO costs
$\utxoFee = 61 \cdot 3.35 \cdot 10^{-9} = \$2 \cdot 10^{-7}$. Given $10000$
full nodes\footnote{\url{https://bitnodes.io} [July 2020]}, which maintain the
ledger and keep the UTxO in memory, the cost becomes $\utxoFee =
\$0.002$; equivalently, denominated in Bitcoin\footnote{$1 BTC = \$9000$
[\url{https://coinmarketcap.com}; July 2020]}, the cost of creating a UTxO is
$\utxoFee = 22$ satoshi.

This solution incorporates the operational costs of miners, thus it is the
rational choice for miners who aim at maximizing their profit. Observe that,
after subtracting the fees that relate to UTxO costs, the scoring mechanism
behaves the same as the one currently used by Bitcoin miners. Therefore, if
users wish to prioritize their transactions, they would again simply increase
their transaction's fees; in that case, the UTxO portion of the fees (\ie
$\utxoFee \cdot \costFunc(\tx, \ledgerState)$) remains the same, hence higher
$\mathsf{fees}$ result in a higher $\mathsf{score}$, similar to the existing
mechanism. Also we note that this mechanism is directly enforceable on Bitcoin
without the need of a fork.

\subsection{A State Efficient Bitcoin}\label{sec:state-efficient-bitcoin}

Intuitively, to make $\feeFunction$ state efficient we force the creator of
a UTxO to subsidize its consumption, \ie to pay the user who later
consumes it. Our fee function is again: $\feeFunction' =
\sizeByteFee \cdot \mathsf{size}(\tx) + \utxoFee \cdot \costFunc(\tx,
\ledgerState)$. Assume two transactions $\tx_1, \tx_2$ with $i_1, i_2$
inputs and $o_1, o_2$ outputs respectively:
\begin{align}
    \costFunc(\tx_1) > \costFunc(\tx_2) \Leftrightarrow
    o_1 - i_1 > o_2 - i_2 \Leftrightarrow
    o_2 - o_1 < i_2 - i_1 \label{eq:tx-cost-diff}
\end{align}
$\feeFunction'$ is state efficient (cf.
Definition~\ref{def:state-efficiency}) if:
\begin{align}
    \feeFunction'(\tx_1) > \feeFunction'(\tx_2) \Rightarrow \nonumber \\
    \mathsf{size}(\tx_1) \cdot \sizeByteFee + \costFunc(\tx_1) \cdot \utxoFee > \mathsf{size}(\tx_2) \cdot \sizeByteFee + \costFunc(\tx_2) \cdot \utxoFee \Rightarrow \nonumber \\
    (i_1 \cdot \txInputSize + o_1 \cdot \utxoSize) \cdot \sizeByteFee + (o_1 - i_1) \cdot \utxoFee > (i_2 \cdot \txInputSize + o_2 \cdot \utxoSize) \cdot \sizeByteFee + (o_2 - i_2) \cdot \utxoFee \Rightarrow \nonumber \\
     (o_1 - i_1) \cdot \utxoFee - (o_2 - i_2) \cdot \utxoFee > (i_2 \cdot \txInputSize + o_2 \cdot \utxoSize) \cdot \sizeByteFee - (i_1 \cdot \txInputSize + o_1 \cdot \utxoSize) \cdot \sizeByteFee \Rightarrow \nonumber \\
     (i_2 - i_1 + o_1 - o_2) \cdot \utxoFee > ((i_2 - i_1) \cdot \txInputSize + (o_2 - o_1) \cdot \utxoSize) \cdot \sizeByteFee \xRightarrow{(\ref{eq:tx-cost-diff})} \nonumber \\
    \utxoFee > \frac{(i_2 - i_1) \cdot \txInputSize + (o_2 - o_1) \cdot \utxoSize}{(i_2 - i_1) - (o_2 - o_1)} \cdot \sizeByteFee \label{eq:state-efficiency}
\end{align}
If $\feeFunction'$ is narrow state efficient, then $o_1 = o_2$ and the inequality is simplified:
\begin{align}
    \utxoFee > \txInputSize \cdot \sizeByteFee \label{eq:narrow-state-efficiency}
\end{align}

We turn again to the previous example. For transaction $\tx_1$, with $1$ input
and $1$ output, $\feeFunction'(\tx_1) = (\txInputSize + \utxoSize) \cdot
\sizeByteFee$ and for transaction $\tx_2$, with $2$ inputs and $1$ output,
$\feeFunction'(\tx_2) = (2 \cdot \txInputSize + \utxoSize) \cdot \sizeByteFee -
\utxoFee = \feeFunction'(\tx_1) + \sizeByteFee \cdot \txInputSize - \utxoFee$.
Since Inequalities~\ref{eq:state-efficiency}
and~\ref{eq:narrow-state-efficiency} ensure that $\utxoFee > \txInputSize \cdot
\sizeByteFee$, the size fee of the extra input in $\tx_2$ is offset by the
extra fee $\utxoFee$, which is paid by the user who creates it.
Again to evaluate these variables we consider market prices. The size of a
typical, pay-to-script-hash or pay-to-public-key-hash, UTxO is
$34$ Bytes~\cite{btc-tx}, while the size of consuming it is $146$ bytes.
Therefore, to make and make present-day Bitcoin (narrow) state efficient, we can
set $\utxoSize = 34$, $\txInputSize = 146$, $\sizeByteFee = 0.0067$\$, and thus
$\utxoFee > 0.0978$\$.

However, this approach presents a number of challenges. To enforce
$\feeFunction'$, the fee policy should be incorporated in the consensus
protocol and a transaction's validity will depend on its amount of fees. As
long as $\feeFunction'(\tx) > 0$, \ie a transaction cannot have negative fees,
the fee function can be enforced via a \emph{soft} fork. Specifically, this
change is backwards compatible, as miners that do not adopt this change will
still accept transactions that follow the new fee scheme. However, if
$\costFunc(\tx) \ll 0$ and possibly $\feeFunction'(\tx) < 0$, to implement
$\feeFunction'$ we need to establish a ``pot'' of fees. When a user creates
$\tx$ with fee $\feeFunction' = \sizeByteFee \cdot \mathsf{size}(\tx) +
\utxoFee \cdot \costFunc(\tx, \ledgerState)$, the first part ($\sizeByteFee
\cdot \mathsf{size}(\tx)$) is awarded to the miners as before. The second part
($\utxoFee \cdot \costFunc(\tx, \ledgerState)$) is deposited to (or, in case of
negative cost, withdrawn from) the pot. In case of negative cost, the
transaction defines a special UTxO for receiving the reimbursement. At any
point in time, the size of the pot is directly proportional to the UTxO set.
Observe that the miners receive the same rewards as before, so their business
model is not affected by this change. Finally, the cost of flooding the system
with UTxOs increases by $\utxoFee$ per UTxO which, depending on $\utxoFee$, can
render attacks ineffective.
