\section{Transaction Optimization}\label{sec:tx_optimization}

The purpose of a distributed ledger is to execute payments, \ie transfer value
from one party to another via transactions.  Multiple transactions can perform
the same transfer of value between two parties. Such transactions are
\emph{equivalent} in terms of their final result, \ie transferring some value
between parties A and B, but may vary in their cost to the ledger state.
\emph{Transaction optimization} is the problem of finding the equivalent
transaction with minimum cost; our work is heavily inspired by the seminal
research on database query optimization \cite{ioannidis}.

The cost difference between equivalent transactions may be significant.
For example, assume that Alice wants to give Bob $100$ coins and owns a UTxO of
$100$ coins and $100$ UTxOs of $1$ coin each. Consider the two equivalent
plans:
\begin{inparaenum}[1)]
    \item Alice spends the single UTxO of value $100$ and
        creates $100$ outputs of value $1$ for Bob;
    \item Alice spends the $100$ UTxOs of $1$ coin value and
        defines a single UTxO of value $100$ to transfer to Bob.
\end{inparaenum}
The cost of the two approaches exemplifies the ledger state impact that
equivalent transactions may have. The first plan increases the ledger's state
by $99$ UTxOs, while the second decreases it by the same amount.

Following, we use the terms plan and transaction interchangeably, \ie an
alternative plan that achieves the same goal is expressed as an alternative,
equivalent transaction. Definition~\ref{def:equiv_txs} describes transaction
equivalency, while Definition~\ref{def:equiv_total_orders_of_txs} defines
equivalency between two ordered lists of transactions.

\begin{definition}\label{def:equiv_txs}
    Transactions $\tx_1, \tx_2$ are equivalent (denoted $\tx_1
    \equiv \tx_2$) if, when applied to the same state $\ledgerState_A$
    of a ledger $\ledger$, they result in states $\ledgerState_1$ and
    $\ledgerState_2$ respectively, with the same total accumulated value per unique
    address $\addr$:
    \[
    \forall \addr \in A
        \sum_{\substack{i \in \ledgerState_1 \\ o_i = \utxo(i, \ledger) \\
        o_i.\mathit{address} =
        \addr }}^{} o_i.\mathit{value} =
        \sum_{\substack{ j \in \ledgerState_2 \\ o_j = \utxo(j, \ledger) \\
        o_j.\mathit{address} =
        \addr }}^{} o_j.\mathit{value}
    \]
    where $A$ is the set of all addresses of the parties participating in the
    ledger system.
\end{definition}

\begin{definition}\label{def:equiv_total_orders_of_txs}
    Two different totally ordered sets of the same $N$ transactions $[T_i]$ and
    $[T_j]$ are equivalent (denoted as $[T_i]
    \equiv [T_j]$) if, when applied to the same ledger state $\ledgerState_A$
    of a ledger $\ledger$, they result in states $\ledgerState_1$ and
    $\ledgerState_2$ respectively, where the total accumulated value per unique
    address $\addr$ is the same in both states:
    \[
    \forall \addr \in A
    \sum_{\substack{i \in \ledgerState_1 \\ o_i = \utxo(i, \ledger) \\
        o_i.\mathit{address} =
        \addr }}^{} o_i.\mathit{value} =
    \sum_{\substack{ j \in \ledgerState_2 \\ o_j = \utxo(j, \ledger) \\
        o_j.\mathit{address} =
        \addr }}^{} o_j.\mathit{value}
    \]
    where $A$ is the set of addresses of all participants in the distributed
    ledger system.
\end{definition}

Following, we define the basic logical operators for expressing a transaction
and explore optimization techniques for compiling the optimal transaction plan.
