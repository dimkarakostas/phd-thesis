\paragraph{Contributions}
Our work investigates techniques that minimize the shared state of the
distributed ledger, \ie the in-memory UTxO set. Our approach is twofold:
\begin{inparaenum}[a)]
    \item we propose transaction optimization techniques which, when employed
        by wallets, help reduce the shared state's cost;
    \item propose a novel fee scheme that incentivizes ``shared
        state-friendly'' transactions.
\end{inparaenum}

In particular, we propose a UTxO model, which abstracts UTxO ledgers and
enables evaluating the cost of a ledger’s shared state. We then
propose a transaction optimization framework, based on three levels of
optimization:
\begin{inparaenum}[a)]
    \item a declarative (rule-based) level,
    \item a logical/algebraic (cost-based) level, and
    \item a physical/algorithmic (cost-based) level.
\end{inparaenum}
Following, we propose three transaction optimization techniques based on the
aforementioned optimization levels:
\begin{inparaenum}[a)]
    \item a rule-driven optimal total order of transactions (the
        \emph{last-payer rule}),
    \item a logical transaction transformation (the \emph{2-for-1
        transformation}), and
    \item a novel \emph{input selection} algorithm that minimizes the UTxO set
        increase, \ie favors consumption over creation of UTxOs.
\end{inparaenum}
We then define the transaction optimization problem and propose a 3-step
dynamic programming algorithm to approximate the optimal solution.  Finally, we
define the state efficiency property that a fee function should have, in order
to correctly reflect a transaction's shared-state cost, and propose a state
efficient fee function for Bitcoin.
