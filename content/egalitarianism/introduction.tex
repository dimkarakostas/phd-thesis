In almost all blockchain cryptocurrency systems, block generators are
incentivized to participate via block rewards, \ie for each block they
successfully produce and which is subsequently adopted by all other
participants. In many cryptocurrencies, the rewards serve a dual purpose:
incentivise the the miners/minters but also create and distribute the
underlying cryptocurrency to the system's maintainers. These rewards follow
various schedules that are designed based on the macroeconomic desiderata
envisioned by the architects of the cryptocurrency. For example, the rate of
coin production is halved every $210,000$ blocks in Bitcoin. Ethereum and
Litecoin follow similar schedules.  On the contrary, Monero has a smooth
emission schedule in which the rewards are gradually reduced at every new block
generated. The question of what this schedule should be can have significant
impact on the variance of stake ownership after an execution of a sufficient
number of protocol rounds~\cite{FC:FKORVW19}. Taking this into account, in this
chapter we consider the block generators as investors and focus on the
comparison of the expected returns of investors with different purchasing
power.

A central economic property that arises from this line of thought is
\emph{cryptocurrency egalitarianism} (also ``crypto-egalitarianism''). This
property states that rewards should be proportional to the invested capital.
Therefore, wealthy investors should not be disproportionately rewarded, but
everybody should have equal opportunity to both participate and earn rewards.
Until now, the term crypto-egalitarianism has been left undefined, although
several cryptocurrencies claim to be more egalitarian than
others~\cite{van2013cryptonote,mcmillan2013}. However, lacking a quantifiable
metric, the discussion around egalitarianism remains ill-posed.  The core
contribution of this chapter is to put forth the first concrete definition of
egalitarianism, in a way which is generic, practically measurable, and
applicable to any cryptocurrency. Additionally, we show that wealth
redistribution, from the rich to the poor, is impossible in decentralized,
anonymous (or pseudonymous) systems; thus, rewarding everybody proportionately
to their capital is the best achievable setting.
