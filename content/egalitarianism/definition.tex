\subsection{A Formal Model of Crypto-Egalitarianism}\label{sec:definition}

The core contribution of this chapter is a formal definition of an economic
measure of \emph{egalitarianism} in cryptocurrencies.

Before we present our definition, let us first state the \emph{desiderata} of
such a definition. First, we want to enable concrete measurements on
cryptocurrencies, in a manner that is quantitative and not vague. Thus far,
egalitarianism claims have been rather informal, failing to include exact
data~\cite{van2013cryptonote,mcmillan2013}. As such, different cryptocurrencies
claim egalitarianism over the others, without demonstrating the claims or
provide conclusive arguments. Second, an egalitarianism definition must measure
the protocol maintenance returns of smaller vs. larger investors.  We thus
desire a metric which extracts a smaller value to indicate a \emph{lack of
egalitarianism} (\eg when large wealth generates blocks disproportionately
faster than small wealth), or a larger value to indicate \emph{perfect
egalitarianism} (\ie when every invested dollar has exactly equal power in
terms of cryptocurrency generation).

The first step in establishing our crypto-egalitarianism definition is to
define the \emph{egalitarian curve} $f$. The horizontal axis of this curve
plots the financial capital, which is available for investment, denominated in
a fiat currency (USD).\footnote{Given that we explore a small investment
duration, it makes little difference whether these are nominal USD or real USD,
as long as they are the same when applying comparisons.} The vertical axis
plots the Return On Investment (ROI), which measures the amount of
cryptocurrency that is created during the investment period and remains unspent
at the end of it, given an optimal allocation of the initial capital. We
require that ROI is computed over \emph{freshly generated} cryptocurrency;
thus, it must be newly-mined or minted, and not purchased from existing
investors. Finally, the curve is plotted with a fixed investment duration in
mind; naturally, curves of different cryptocurrencies can be compared only if
they use the same duration.

\begin{definition}[Egalitarian curve]
    Given a cryptocurrency $c$ and the set of all possible investment
    strategies $\mbb{B}$, we define the \emph{egalitarian curve} $f_{c,d}:
    \mbb{R}^+ \longrightarrow \mbb{R}^+$ of $c$ for an investment period $d$
    as:
    \begin{align}
        f_{c, d}(v) = \frac{\underset{B \in \mbb{B}}{\max}{E[B(v)]} - v}{v}
    \end{align}
\end{definition}

The value $\underset{B \in \mbb{B}}{\max}{E[B(v)]}$ identifies the maximum
expectation of returns across all investment strategies $\mbb{B}$, \ie the
amount of returns which the \emph{optimal} strategy ensures for a given initial
capital $v$. The blockchain execution is modeled as a random variable, since
returns vary by execution; specifically, the randomness of the execution can
affect returns, as a participant may be ``lucky'', \ie produce more blocks than
expected~\cite{FC:FKORVW19}.

We remark that we \emph{do} allow strategies to reinvest capital. For instance,
returns earned from mining rewards can be reinvested in electricity costs for
future mining. Furthermore, for unit consistency, we assume the strategy $B(v)$
returns the freshly generated coins denominated in the same units as the
capital $v$. Second, we assume participants act independently and follow the
protocol according to its specifications.

Using our definition of the egalitarian curve, we now define
(Definition~\ref{def:egalitarianism}) egalitarianism as a concrete number.
Considering the initial capital $v$ as a random variable, which follows a
certain distribution $\mc{D}$, egalitarianism is the variance of the expected
ROI, when the capital is chosen from the given distribution.

\begin{definition}[Egalitarianism]\label{def:egalitarianism}
    Given a cryptocurrency $c$ and its egalitarian curve $f$, we define the
    \emph{egalitarianism} $e$ of $c$, for investment duration $d$ under initial
    capital distribution $\mc{D}$, as follows:
    \begin{align}
      e_{c, d, \mc{D}} = -\msf{Var}_{v \gets \mc{D}}[f_{c, d}(v)]
    \end{align}
\end{definition}

The intuition behind this definition is that, to have egalitarianism, the ROI
must remain the same across different capital investments. As such, any
deviation from the mean is non-egalitarian. Naturally, if a system's
egalitarianism is \emph{higher} than another, we say that the former is
\emph{more egalitarian} than the latter. Of course, to be accurate, such
comparisons must be made after fixing the parameters $c$ and $d$, as well as
the initial capital distribution $\mc{D}$. Fixing $\mathcal{D}$ to be the
uniform distribution between a minimum and a maximum capital, the returns are
the same for all initial capitals alike.

Based on the above, we can define the \emph{ideal egalitarian curve}. First,
as an interesting thought experiment, we consider the egalitarian curve which
is decreasing (and is, arguably, \emph{the} ideal curve). In this case, small
investors would receive proportionally more newly created cryptocurrencies for
every dollar they invest, \ie the system would redistribute wealth from the
rich to the poor. However, one can quickly see that, in decentralized
cryptocurrencies where the identities of the participants are unknown, this is
impossible. Indeed, the fact that decentralized cryptocurrencies allow
anonymous generation of new identities~\cite{douceur2002sybil} allows a wealthy
investor to split their capital into smaller ones. Thus, if the curve were ever
to have a negative slope, the sum of the smaller splits of the rich investment
would achieve a higher gain. By the definition of the curve, which mandates
that it depicts the ROI of an \emph{optimal} investment, this would be a
contradiction. Corollary~\ref{cor:sybil} makes this intuition more precise.

\begin{corollary}[Sybil strategies]\label{cor:sybil}
    Fix a cryptocurrency $c$ and an investment period interval $d$. Given
    capital $v$, for every natural number $i \in \mathbb{N}^\star$, it holds
    that $f_{c,d}(v) \leq f_{c, d}(i \cdot v)$.
\end{corollary}
\begin{proof}
    We prove the statement by contradiction. Assume that, for some capital $v$,
    there exists a natural number $i \in \mathbb{N}^\star$ such that
    $f_{c,d}(v) > f_{c,d}(i \cdot v)$. Also assume that, for capital $v$, the
    optimal strategy is $B'$, so $\underset{B \in
    \mathbb{B}}{\max}{\mathbb{E}[B(v)]} = \mathbb{E}[B'(v)]$. For capital $i
    \cdot v$, there exists a strategy $B''$, such that the capital is split
    into $i$ equally-sized parts, with the strategy $B'$ applied on each part.
    Given that the execution of each such sub-strategy is independent, the
    expected returns for $B''$ are:
    \begin{align}\label{eq:break-strategy}
        \mathbb{E}[B''(i \cdot v)] = i \cdot \mathbb{E}[B'(v)]  = i \cdot \underset{B \in \mathbb{B}}{\max}{\mathbb{E}[B(v)]}
    \end{align}
    Additionally, $B''$ is at best the optimal strategy, so:
    \begin{align}\label{eq:multi-strategy}
        \underset{B \in \mathbb{B}}{\max}{\mathbb{E}[B(i \cdot v)]} \geq \mathbb{E}[B''(i \cdot v)] \xRightarrow{\text{(\ref{eq:break-strategy})}}
        \underset{B \in \mathbb{B}}{\max}{\mathbb{E}[B(i \cdot v)]} \geq i \cdot \underset{B \in \mathbb{B}}{\max}{\mathbb{E}[B(v)]}
    \end{align}
    However, it should also hold that:
    \begin{alignat}{2}
        f_{c,d}(v) &> f_{c,d}(i \cdot v) \Rightarrow \notag\\
        \frac{\underset{B \in \mathbb{B}}{\max}{\mathbb{E}[B(v)]} - v}{v} &> \frac{\underset{B \in \mathbb{B}}{\max}{\mathbb{E}[B(i \cdot v)]} - i \cdot v}{i \cdot v} \xRightarrow{\text{(\ref{eq:multi-strategy})}} \notag\\
        \frac{\underset{B \in \mathbb{B}}{\max}{\mathbb{E}[B(v)]} - v}{v} &> \frac{i \cdot \underset{B \in \mathbb{B}}{\max}{\mathbb{E}[B(v)]} - i \cdot v}{i \cdot v} \Rightarrow \notag\\
        \frac{\underset{B \in \mathbb{B}}{\max}{\mathbb{E}[B(v)]} - v}{v} &> \frac{\underset{B \in \mathbb{B}}{\max}{\mathbb{E}[B(v)]} - v}{v} \notag
    \end{alignat}
    which is impossible.
\end{proof}

Corollary~\ref{cor:sybil} shows that, in purely decentralized systems, a
decreasing egalitarian curve is impossible. Therefore, the next-best ideal
curve is a constant one, where the ROI is stable regardless of capital
invested. Under this condition, the amount of freshly generated cryptocurrency
is exactly proportional to the money invested. Consequently, a cryptocurrency
with an ideal egalitarian curve is perfectly egalitarian
(Definition~\ref{def:perfect-egalitarianism}).

\begin{definition}[Perfect egalitarianism]\label{def:perfect-egalitarianism}
    A cryptocurrency $c$ is \emph{perfectly egalitarian}, for investment
    duration $d$ and initial capital distribution $\mc{D}$, if $e_{c, d, \mc{D}}
    = 0$.
\end{definition}
