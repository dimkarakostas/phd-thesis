\subsection{Discussion}

In this chapter, in providing a concrete definition of crypto-egalitarianism,
we enable an evidence-based discussion to substitute folklore arguments.
The first application of this metric was provided
in~\cite{karakostas2019cryptocurrency}, which compared some of the largest
cryptocurrency systems to date.  Using our model, the egalitarianism of four
indicative PoW-based cryptocurrencies (Bitcoin,
Litecoin~\cite{lee2011litecoin}, Ethereum~\cite{buterin,wood2014ethereum}, and
Monero~\cite{van2013cryptonote}) was measured.  The assessed claims of these
projects were found in agreement with our data, thus presenting for the first
time economic comparisons which quantify them precisely. On the pure PoS side,
it was shown that egalitarian behavior is similar across all coins,
independently of externalities such as hardware characteristics. Therefore, it
suffices to perform a case study of an indicative PoS protocol (in this case,
Ouroboros~\cite{C:KRDO17}).  It was then shown that pure PoS coins can be
perfectly egalitarian, contrary to their PoW counterparts.
These results were very optimistic in terms of usability of our metric, as they
provide concrete figures which measure the egalitarianism of several popular
cryptocurrencies. The most unexpected result arised from the comparison between
the PoW and PoS mechanisms. Although blockchain folklore argued in favour of
PoW systems in terms of egalitarianism, these results show that, in fact, it is
PoS systems which are more egalitarian under our proposed model.

Another interesting property that arises from this work is the impossibility of
wealth redistribution in cryptocurrency systems. As shown in
Corollary~\ref{cor:sybil}, in a purely decentralized setting, where no
real-world identity checks exist (and, arguably, cannot exist), a wealthy
participant can always pose as multiple poor users. Therefore, any attempt to
redistribute wealth from the rich to the poor, based on entirely technological
tools and without taking into account real-world social structures, seems
doomed to fail. In conclusion, all decentralized cryptocurrencies are ``rich
get richer'' schemes; the pertinent question, which this chapter aimed at
resolving, is \emph{how fast} this takes place.
