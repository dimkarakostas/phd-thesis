\subsection{PoW vs. PoS}

Before studying the egalitarianism of different cryptocurrency consensus
mechanisms, we consider the leader election process, to establish an
understanding of the differences in egalitarianism between the two models of
PoW and PoS.

\paragraph{Proof-of-Work}
The number of hash evaluations is one of the several critical parameters to
consider when purchasing mining hardware. Other important parameters include
the price of a mining unit, as well as its electricity consumption. Mining
hardware is divided in various tiers based on performance, namely CPU miners,
GPU miners, FPGA miners, and specialized ASIC miners~\cite{taylor2013bitcoin}.
Although the pricing of such devices may be similar, the hashing rate and, in
turn, the Return on Investment, is highly dependent on the hardware's tier.
For example, as of December 2018, the mining hardware ``Whatsminer M10''
produced by the company ``MicroBT'' cost $\$1{,}022.00$ per unit and produces
$\$0.104266$ per hour of operation in net gains, \ie average mined Bitcoins per
hour denominated in US dollars minus the electricity costs. On the other hand,
the mining hardware ``8 Nano Pro'', produced by the company ``ASICMiner'', cost
$\$6{,}000.00$ per unit, but produces $\$0.315327$ per hour of operation in net
gains, \ie almost three times the hourly net gains of its cheaper competitor.
Thus, if one can afford to purchase the more expensive hardware, each of their
subsequent dollar invested in electricity returns more mined coins.

It has long been folklore knowledge in the blockchain community that mining
becomes more egalitarian by using a memory-hard PoW function. This intuition is
correct, the core reason being the difficulty to construct specialized hardware
for memory-hard functions. For example, no ASICs currently exist for Monero
mining. Therefore, the only way to scale mining operations is by purchasing
more general purpose hardware. However, since the mining hardware in this case
varies little, both in terms of cost and performance, scaling returns become
proportional to investments.

In this chapter, we only analyze the scaling of the economics of mining with
respect to hardware. We also do not take into account basic costs such as
shipping and the availability of a basic machine to co-ordinate mining (such as
a personal computer not performing mining itself). A multitude of additional
factors play important roles for mining operations, such as space rental costs,
machine cooling and maintenance costs, or bulk electricity purchase. As is
common in economies of scale, these relative costs are reduced for large-scale
operations, although they are similar for all PoW cryptocurrencies and thus do
not affect relative comparisons between them. We also remark that we analyze
mining costs for small capital investments. If larger capital, \eg above a few
million US dollars, is available, corporations can develop their own
specialized hardware and gain a competitive advantage by treating it as a trade
secret~\cite{taylor2013bitcoin}. These details make the comparison in favour of
PoS \emph{more pronounced}, as PoS operations do not incur such types of costs
and do not lend themselves to specialized mining hardware research.

\paragraph{Proof-of-Stake}
PoS is often criticized for its lack of egalitarianism. The rationale is that,
in PoS, the more money one stakes, the more money one generates. Thus, ``the
rich get richer'', which is precisely the opposite of egalitarianism.
Additionally, in PoS systems, the money owners could constitute a \emph{closed,
rich club}, refusing to share the assets with any outsiders. In contrast, this
argument claims, PoW is naturally egalitarian; everyone is paid based not on
the money they own, but on the computational power they put to work. In this
case, since computational power is a \emph{natural} resource and cannot be
exclusively owned, a closed rich club cannot be formed. Although this argument
seems agreeable at first, the results of this chapter contradict it. In fact,
correctly parameterized stake-based systems are much more egalitarian than
work-based ones.\footnote{Variations of PoS, such as
delegated PoS, may not be perfectly egalitarian, since the delegates, \ie the
leaders of the stake pools, typically earn extra profits for managing the stake
pools~\cite{DBLP:conf/eurosp/BrunjesKKS20}.}

It is instructive to dispel the above argument intuitively, before we support
our position with data. First, the argument that money can be exclusively
owned, but computational power cannot, is rather misguided. Indeed, this may be
true in the case of a peculiar oligopoly, where a small faction of parties
mutually agrees to never sell to outsiders, despite external demand. However,
in an open market, both money and computational power can be freely purchased
and, in fact, any non-negligible amount of computational power must be
necessarily purchased that way.  In this work, we assume an open market for
both mining hardware and financial capital, which allows open participation.
Therefore, given that both money and computational power are purchasable, we
consider the funds one invests, either in technology or in financial capital,
in order to maximize the returns from a cryptocurrency's block generation
mechanisms. The amount of cryptocurrency generated by a given investment can be
concretely measured and compared, thus the question can now be analyzed
quantitatively and answered concretely.
