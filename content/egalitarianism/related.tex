\paragraph{Related work}
The macro and microeconomics of blockchain design have been studied from
several perspectives, but remain an active area of research with a number of
open questions. \emph{Egalitarianism} in particular has been studied in PoW
systems from the perspective of \emph{memory-hard
functions}~\cite{EC:AlwBloPie17,USENIX:BirKho16}. These works operate under the
premise that memory hardness provides egalitarianism, in the sense that the
cost of one computational step is roughly the same irrespective of the
underlying computational platform (typically ASIC vs. generic). In this chapter
we generalize this question, by asking whether computational power grows
proportionally to capital invested, \ie whether larger wealth results
disproportionately more rewards. Additionally, a notable work by Fanti
\etal~\cite{FC:FKORVW19} introduces the complementary notion of
\emph{equitability}.  That work studies the evolution of a system after a
series of rounds, putting forth the property that stake ownership remains in
proportion \emph{before} and \emph{after} rewards have been awarded. By
studying the behavior of the returns' \emph{variance} under the randomness of
executions, they show that the distribution of capital follows a Pólya process.
This chapter can be seen as complementary to their results, by quantifying the
\emph{expectation} of rewards and then studying the variance under the
randomness of initial capital allocation. Therefore, a cryptocurrency can be
perfectly egalitarian and poorly equitable and vice versa; notably, it is
possible to obtain a cryptocurrency both egalitarian and equitable, by adopting
correctly parameterized PoS under a geometric reward function.
